\documentclass[10pt]{article}

\usepackage{fullpage}
\usepackage{setspace}
\usepackage{parskip}
\usepackage{titlesec}
\usepackage[section]{placeins}
\usepackage{xcolor}
\usepackage{breakcites}
\usepackage{lineno}
\usepackage{hyphenat}





\PassOptionsToPackage{hyphens}{url}
\usepackage[colorlinks = true,
            linkcolor = blue,
            urlcolor  = blue,
            citecolor = blue,
            anchorcolor = blue]{hyperref}
\usepackage{etoolbox}
\makeatletter
\patchcmd\@combinedblfloats{\box\@outputbox}{\unvbox\@outputbox}{}{%
  \errmessage{\noexpand\@combinedblfloats could not be patched}%
}%
\makeatother


\usepackage[round]{natbib}
\let\cite\citep




\renewenvironment{abstract}
  {{\bfseries\noindent{\abstractname}\par\nobreak}\footnotesize}
  {\bigskip}

\titlespacing{\section}{0pt}{*3}{*1}
\titlespacing{\subsection}{0pt}{*2}{*0.5}
\titlespacing{\subsubsection}{0pt}{*1.5}{0pt}


\usepackage{authblk}


\usepackage{graphicx}
\usepackage[space]{grffile}
\usepackage{latexsym}
\usepackage{textcomp}
\usepackage{longtable}
\usepackage{tabulary}
\usepackage{booktabs,array,multirow}
\usepackage{amsfonts,amsmath,amssymb}
\providecommand\citet{\cite}
\providecommand\citep{\cite}
\providecommand\citealt{\cite}
% You can conditionalize code for latexml or normal latex using this.
\newif\iflatexml\latexmlfalse
\AtBeginDocument{\DeclareGraphicsExtensions{.pdf,.PDF,.eps,.EPS,.png,.PNG,.tif,.TIF,.jpg,.JPG,.jpeg,.JPEG}}

\usepackage[utf8]{inputenc}
\usepackage[ngerman,english]{babel}








\newcommand{\bra}[1]{\ensuremath{\left\langle#1\right|}}
\newcommand{\ket}[1]{\ensuremath{\left|#1\right\rangle}}
\newcommand{\rhohat}{\hat{\rho}}
\newcommand{\eexp}[1]{\mathrm{e}^{#1}}
\newcommand{\trarb}[2]{\mathrm{tr}_{#1}(#2)}
\newcommand{\tr}[1]{\mathrm{tr}(#1)}

\begin{document}

\title{Lecture 1 - Some cooking recipes for Quantum Mechanics}



\author[1]{Fred Jendrzejewski}%
\author[2]{Selim Jochim}%
\affil[1]{Kirchhoff-Institut für Physik}%
\affil[2]{Physikalisches Institut der Universität Heidelberg}%


\vspace{-1em}



  
  \date{January 03, 2025}


\begingroup
\let\center\flushleft
\let\endcenter\endflushleft
\maketitle
\endgroup





\selectlanguage{english}
\begin{abstract}
In this first lecture we will review the foundations of quantum mechanics at the level of a cooking recipe. This will enable us to use them later for the discussion of the atomic structure and interaction between atoms and light.%
\end{abstract}%



\sloppy


This is the first lecture of the Advanced Atomic Physics course at Heidelberg University, as tought in the wintersemester 2019/2020.
It is intended for master students, which have a basic understanding of quantum mechanics and electromagnetism. In total, we will study multiple topics of modern atomic, molecular and optical physics over a total of 24 lectures, where each lectures is approximately 90 minutes. 
\begin{itemize}
\item We will start the series with some basics on quantum mechanics.
\item Then work our way into the harmonic oscillator and the hydrogen atom.
\item We will then leave the path of increasingly complex atoms for a moment to have some fun with light-propagation, lasers and discussion of the Bell inequalities.
\item A discussion of more complex atoms gives us the acutual tools at hand that are in the lab.
\item This sets up a discussion of di-atomic molecules, which ends the old-school AMO.
\item We move on to quantized atom-light interaction, the Jaynes Cummings model and strong-field lasers.
\item We will finally finish with modern ways to implement quantum simulators and quantum computers.
\end{itemize}

The topics of the lectures will be discussed in more details in the associated tutorials. 

\section{Introduction}
In AMO physics we will encounter the consequences of quantum mechanics all the time. So we will start out with a review of the basic ingredients to facilitate the later discussion of the experiments. 

Some good introductions on the traditional approach can be found in \cite{2002, 2006} \cite{1}\cite{2}. Previously, we mostly followed the discussion of Ref. \cite{2006}. Nowadays, I also recommend the works by Scott Aaronson \cite{quantum,holes}. There is also a good \href{https://www.quantamagazine.org/quantum-theory-rebuilt-from-simple-physical-principles-20170830/#}{article by Quanta-Magazine} on the whole effort to derive quantum mechanics from some simple principles. This effort started with Ref. \cite{axioms}, which actually makes for a nice read.

Before we start with the detailled cooking recipe let us give you some examples of  quantum systems, which are of major importance throughout the lecture:
\begin{enumerate}
\item \textit{Orbit in an atom, molecule etc}. Most of you might have studied this during the introduction into quantum mechanics.
\item \textit{Occupation number of a photon mode}. Any person working on quantum optics has to understand the quantum properties of photons.
\item \textit{Position of an atom} is of great importance for double slit experiments, the quantum simulation of condensed matter systems with atoms,  or matterwave experiments.
\item The \textit{spin degree of freedom} of an atom like in the historical Stern-Gerlach experiment. 
\item The classical coin-toss or bit, which connects us nicely to simple classical probability theory or computing
\end{enumerate}

\section{The possible outcomes  (the Hilbert Space) for the Problem in Question}

The first step is to identify the right Hilbert space for your problem. For a classical problem, we would simply list all the different possible outcomes in a list $(p_1, \cdots, p_N)$ of \textit{real} numbers. As one of the outcomes has to happen, we obtain the normalization condition:
\begin{align}
\sum_i p_i = 1
\end{align}

In quantum mechanics, we follow a similar approach of first identifying the possible outcomes. But instead of describing the outcomes with real numbers, we now associate a complex number $\alpha_i$ to each outcome $(\alpha_1, \cdots, \alpha_N)$, with $\alpha_i \in \mathbb{C}$. Given that they should also describe some probability they have to be normalized to one, but now we have the condition:
\begin{align}
\sum_i |\alpha_i|^2 = 1
\end{align}

Aaronson claims that this is just measuring probabilities in in $L_2$ norm. I would highly recommend his discussions on his blog for a more instructive derivation\cite{quantum}. Next we will not use the traditional lists, but the bra-ket notation, by writing:
\begin{align}
\ket{\psi}&= \sum_i \alpha_i \ket{i}
\end{align}

And given that these are complex vectors, we will measure their overlap through a Hermitian scalar product
\begin{align}
\langle\psi_1 \psi_2\rangle=(\langle{\psi_2}| \psi_1\rangle)^*.
\end{align}
\subsection{The coin toss}
The situation becomes particularly nice to follow for the two level system or the coin toss. In classical systems, we will get heads up $\uparrow$ with a certain probability p. So the inverse $\downarrow$ arrives with likelyhood $1-p$. We would then classical list the probabilities with $(p,1-p)$. In the quantum world we achieve such a coin for example in spin 1/2 systems or qubits in general. We will then describe the system through the state:
\begin{align}
\ket{\psi} = \alpha_\uparrow \ket{\uparrow} + \alpha_\downarrow \ket{\downarrow}  \qquad \text{with} \; \langle\psi | \psi\rangle = 1.
\end{align}

The next problem is how to act on the system in the classical world or in the quantum world.


\subsection{Quantum rules}
Having set up the space on which we want to act we have to follow the rules of quantum mechanics. The informal way of describing is actually nicely described by Chris Monroe \href{https://youtu.be/CC7nlBM2cSM}{in this video}. We might summarize them as follows:
\begin{enumerate}
\item Quantum objects can be in several states at the same time.
\item Rule number one only works when you are not looking.
\end{enumerate}

The more methematical fashion is two say that there two ways of manipulating quantum states:
\begin{enumerate}
\item Unitary transformations $\hat{U}$.
\item Measurements.
\end{enumerate}

\section{Unitary transformations}
As states change and evolve, we know that the total probability should be conserved. So we transform the state by some operator $\hat{U}$, which just maps the state $\ket{\psi}\xrightarrow[]{U}\ket{\psi'}$. This should not change the norm, and we obtain the condition:
\begin{align}
\bra{\psi}\hat{U}^\dag\hat{U} \ket{\psi} = 1\\
\hat{U}^\dag\hat{U}  = \mathbb{1}
\end{align}
That's the very definition of unitary operators and unitary matrices.
Going back to the case of a coin toss, we see that we can then transform our qubit through the unitary operator:
\begin{align}
\hat{U}=\frac{1}{\sqrt{2}}\left(\begin{array}{cc}
1 & -1\\
1 & 1
\end{array}\right)
\end{align}
Applying it on the previously defined states $\uparrow$ and $\downarrow$, we get the superposition state:
\begin{align}
\hat{U}\ket{\uparrow} &= \frac{\ket{\uparrow}-\ket{\downarrow}}{\sqrt{2}}\\
\hat{U}\ket{\downarrow} &= \frac{\ket{\uparrow}+\ket{\downarrow}}{\sqrt{2}}
\end{align}
As we use the unitary matrices we also see why we might one to use complex numbers. Imagine that we would like to do something that is roughly the square root of the unitary, which often just means that the system should evolve for half the time as we will see later. If we then have negative nummbers, they will immediately become imaginary.

Such superposition would not be possible in the classical case, as non-negative values are forbidden there. Actually, operations on classical propability distributions are only possible if every entry of the matrix is non-negative (probabilities are never negative right ?) and each column adds up to one (we cannot loose something in a transformation). Such matrices are called \mathit{stochastic matrices}.

\section{Observables and Measurements}
As much fun as it might be to manipulate a quantum state, we also have to measure it and how it connects to the properties of the system at hand. Any given physical quantity $A$ is associated with a Hermitian operator $\hat{A} = \hat{A}^\dag$ acting in the Hilbert space of the system, which we defined previously. Please, be utterly aware that those Hermitian operators have absolutely no need to be unitary. However, any unitary operator might be written as $\hat{U}= e^{i\hat{A}}$.

In a \emph{measurement} , the possible outcomes are then the eigenvalues $a_\alpha$ \index{eigenvalue} of the operator $\hat{A}$:

\begin{align}
\hat{A}\ket{\alpha}=a_{\alpha}\ket{\alpha}.
\end{align}
The system will collapse to the corresponding eigenvector and the probability of finding the system in state $\ket{\alpha}$ is
\begin{align}
P(\ket{\alpha})=||\hat{P}_{\ket{\alpha}} \ket{\psi}||^2 = \bra{\psi} \hat{P}^{\dag}_{\ket{\alpha}} \hat{P}_{\ket{\alpha}} \ket{\psi},
\end{align}
where $\hat{P}_{\ket{\alpha}}= \ket{\alpha} \bra{\alpha}$.

As for our previous examples, how would you measure them typically, i.e. what would be the operator ?
\begin{enumerate}
\item In atoms the operators will be angular moment, radius, vibrations etc.
\item For the occupation number we have nowadays number counting photodectors.
\item The position of an atom might be detected through high-resolution CCD cameras.
\item For the \textit{measurement of the spin}, we typically correlate the internal degree of freedom to the spatial degree of freedom. This is done by applying a magnetic field gradient acting on the magnetic moment $\hat{\vec{\mu}}$ \index{magnetic moment}, which in turn is associated with the spin via $\hat{\vec{\mu}} = g \mu_B \hat{\vec{s}}/\hbar$, where $g$ is the Land\selectlanguage{ngerman}é $g$-factor \index{Land\'e $g$-factor} and $\mu_B$ is the Bohr magneton \index{Bohr magneton}. The energy of the system is $\hat{H} = -\hat{\vec{\mu}} \cdot \vec{B}$.
\end{enumerate}

\section{Time Evolution}
Being able to access the operator values and intialize the wavefunction in some way, we also want to have a prediction on its time-evolution. For most cases of this lecture we can simply describe the system by the non-relativistic \textbf{Schrödinger Equation.} It reads
\begin{align}
i\hbar\partial_t\ket{\psi(t)}=\hat{H}(t)\ket{\psi(t)}.
\end{align}
In general, the Hamilton operator $\hat{H}$ is time-dependent. For a time-independent Hamilton operator $\hat{H}$, we can find eigenstates $\ket{\phi_n}$ with corresponding eigenenergies $E_n$ \index{eigenenergy}:

\begin{align}
\hat{H}\ket{\phi_n}=E_n\ket{\phi_n}.
\end{align}
The eigenstates $\ket{\phi_n}$ in turn have a simple time evolution:

\begin{align}
    \ket{\phi_n(t)}=\ket{\phi_n(0)}\cdot \exp{-i E_nt/\hbar}.
\end{align}
If we know the initial state of a system
\begin{align}
\ket{\psi(0)}=\sum_n \alpha_n\ket{\phi_n},
\end{align}
where $\alpha_n=\langle\phi_n | \psi(0)\rangle$, we will know the full dimension time evolution
\begin{align}
\ket{\psi(t)}=\sum_n\alpha_n\ket{\phi_n}\exp{-i E_n t/\hbar}. \;\, \text{(Schrödinger picture)}
\end{align}
\textbf{Note.} Sometimes it is beneficial to work in the Heisenberg picture, which works with static ket vectors $\ket{\psi}^{(H)}$ and incorporates the time evolution in the operators. \footnote{We will follow this route in the discussion of the two-level system and the Bloch sphere.}
In certain cases one would have to have access to relativistic dynamics, which are then described by the \textbf{Dirac equation}. However, we will only touch on this topic very briefly, as it directly leads us into the intruiging problems of \textbf{quantum electrodynamics}.

\subsection{The Heisenberg picture}
As mentionned in the first lecture it can benefitial to work in the Heisenberg picture instead of the Schrödinger picture. This approach is widely used in the field of many-body physics, as it underlies the formalism of the second quantization. To make the connection with  the Schrödinger picture we should remember that we have the formal solution of
\begin{align}
\ket{\psi(t)} = \eexp{-i\hat{H}t}\ket{\psi(0)}
\end{align}
So, if we would like to look into the expectation value of some operator, we have:
\begin{align}
\langle\hat{A}(t)\rangle = \bra{\psi(0)}\eexp{i\hat{H}t}\hat{A}_S\eexp{-i\hat{H}t}\ket{\psi(0)}
\end{align}
This motivates the following definition of the operator in the Heisenberg picture:
\begin{align}
    \hat{A}_H=\eexp{i{\hat{H} t}/{\hbar}} \hat{A}_S \eexp{-i{\hat{H} t}/{\hbar}}
\end{align}
where $\exp{-i{\hat{H} t}/{\hbar}}$ is a time evolution operator (N.B.: $\hat{H}_S = \hat{H}_H$). The time evolution of $\hat{A}_H$ is:
\begin{align}
    \notag \frac{d}{dt} \hat{A}_H &=&& \frac{i}{\hbar}\hat{H}\eexp{i{\hat{H}t}/{\hbar}}\hat{A}_S \eexp{-i{\hat{H} t}/{\hbar}}\\ 
    &&-&\frac{i}{\hbar} \eexp{i{\hat{H} t}/{\hbar}}\hat{A}_S \eexp{-i{\hat{H}t}/{\hbar}}\hat{H}+\partial_t \hat{A}_H\\
    &=&& \frac{i}{\hbar}\left[\hat{H},\hat{A}_H\right] + \eexp{i{\hat{H}t}/{\hbar}}\partial_t\hat{A}_S\eexp{-i{\hat{H}t}/{\hbar}}
\end{align}
\textbf{Note.} In the Heisenberg picture the state vectors are time-independent:
\begin{align}
    \ket{\psi}_H \equiv \ket{\psi(t=0)}=\exp{i{\hat{H}}t/{\hbar}} \ket{\psi(t)}.
\end{align}
Therefore, the results of measurements are the same in both pictures:
\begin{align}
 \bra{\psi(t)}\hat{A}\ket{\psi(t)} = \bra{\psi}_H \hat{A}_H \ket{\psi}_H.
 \end{align}

The next lecture of this series can be found here: \cite{Jendrzejewski}.

\selectlanguage{english}
\FloatBarrier
\bibliographystyle{plainnat}
\bibliography{bibliography/converted_to_latex.bib%
}

\end{document}

