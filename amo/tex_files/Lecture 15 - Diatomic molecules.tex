\documentclass[10pt]{article}

\usepackage{fullpage}
\usepackage{setspace}
\usepackage{parskip}
\usepackage{titlesec}
\usepackage[section]{placeins}
\usepackage{xcolor}
\usepackage{breakcites}
\usepackage{lineno}
\usepackage{hyphenat}





\PassOptionsToPackage{hyphens}{url}
\usepackage[colorlinks = true,
            linkcolor = blue,
            urlcolor  = blue,
            citecolor = blue,
            anchorcolor = blue]{hyperref}
\usepackage{etoolbox}
\makeatletter
\patchcmd\@combinedblfloats{\box\@outputbox}{\unvbox\@outputbox}{}{%
  \errmessage{\noexpand\@combinedblfloats could not be patched}%
}%
\makeatother


\usepackage[round]{natbib}
\let\cite\citep




\renewenvironment{abstract}
  {{\bfseries\noindent{\abstractname}\par\nobreak}\footnotesize}
  {\bigskip}

\titlespacing{\section}{0pt}{*3}{*1}
\titlespacing{\subsection}{0pt}{*2}{*0.5}
\titlespacing{\subsubsection}{0pt}{*1.5}{0pt}


\usepackage{authblk}


\usepackage{graphicx}
\usepackage[space]{grffile}
\usepackage{latexsym}
\usepackage{textcomp}
\usepackage{longtable}
\usepackage{tabulary}
\usepackage{booktabs,array,multirow}
\usepackage{amsfonts,amsmath,amssymb}
\providecommand\citet{\cite}
\providecommand\citep{\cite}
\providecommand\citealt{\cite}
% You can conditionalize code for latexml or normal latex using this.
\newif\iflatexml\latexmlfalse
\AtBeginDocument{\DeclareGraphicsExtensions{.pdf,.PDF,.eps,.EPS,.png,.PNG,.tif,.TIF,.jpg,.JPG,.jpeg,.JPEG}}

\usepackage[utf8]{inputenc}
\usepackage[ngerman,english]{babel}








\usepackage{siunitx}
\newcommand{\vecRA}{\mathbf{R}_\mathrm{A}}
\newcommand{\vecRB}{\mathbf{R}_\mathrm{B}}
\usepackage{amsmath}
\newcommand{\bra}[1]{\ensuremath{\left\langle#1\right|}}
\newcommand{\ket}[1]{\ensuremath{\left|#1\right\rangle}}
\newcommand{\braket}[1]{\ensuremath{\left\langle#1\right\rangle}}

\newcommand{\vecra}{\mathbf{r}_\mathrm{A}}
\newcommand{\vecrb}{\mathbf{r}_\mathrm{B}}

\begin{document}

\title{Lecture 15 - Diatomic molecules}



\author[1]{Fred Jendrzejewski}%
\author[2]{Selim Jochim}%
\author[2]{Matthias Weidemüller}%
\affil[1]{Kirchhoff-Institut für Physik}%
\affil[2]{Physikalisches Institut der Universität Heidelberg}%


\vspace{-1em}



  
  \date{January 07, 2025}


\begingroup
\let\center\flushleft
\let\endcenter\endflushleft
\maketitle
\endgroup





\selectlanguage{english}
\begin{abstract}
In this lecture we will start to put atoms together to build simple molecules. We will first use the Born-Oppenheimer approximation, to eliminate slow processes from the study of the fast electron dynamics. Then, we will study simple mechanisms of binding atoms.%
\end{abstract}%



\sloppy


\section{Introduction}

Molecules add a new layer of complexity to the system. In atoms, we had different combinations of nuclei and electrons, leading to different kinds of atoms. In this lecture, we will use atoms as basic building block of more complex structures, the molecules. While this complexity makes it necessary to introduce new approximations, it also allows us to study new processes in nature. 

So we will start out with the simplest of all molecules, barely a molecule, the $H_2^+$ ion. We start out with a discussion of the Born-Oppenheimer approximation. Detailled discussions can be found in Chapter 8 of \cite{atkins1997molecular}, Chapter 9 of \cite{Demtr_der_2010} and Chapter 10 of \cite{bransden2003physics}. 

\section{Molecular hydrogen ion}
In molecular hydrogen we have only three ingredients. A single electron, which is bound to two nuclei as shown in Fig. \ref{547852}.\selectlanguage{english}
\begin{figure}[h!]
\begin{center}
\includegraphics[width=0.70\columnwidth]{figures/bov2/Bildschirmfoto-2017-12-06-um-11.39.14}
\caption{{The H\textsubscript{2}\textsuperscript{+} ion as discussed in the main
text.
{\label{547852}}%
}}
\end{center}
\end{figure}

The full Hamiltonian of the system at study would read:

\begin{equation}
\hat{H} = - \frac{1}{2}\nabla_\mathbf{r}^2 - \frac{1}{2M}\left(\nabla_{\vecRA}^2 +\nabla_{\vecRB}^2\right) + V(\mathbf{r}, \vecRA, \vecRB )
\end{equation}
We will further introduce the short-hand notations:
\begin{eqnarray}
\hat{T}_e &= - \frac{1}{2}\nabla_\mathbf{r}^2 \\
\hat{T}_n &= - \frac{1}{2M}\left(\nabla_{\vecRA}^2 +\nabla_{\vecRB}^2\right)
\end{eqnarray}

In a stark difference to atoms, we now have two charged nuclei. The relative distance between them and between the electron will be of major importance. Most importantly, we should answer the question, why this configuration should be stable at all given that the two protons repel each other. To handle the problem, we will once again separate out energy scales.

\subsection{The Born-Oppenheimer approximation}

The idea of the \textbf{Born-Oppenheimer approximation} is to separate the fast electronic motion from the slow motion of the heavy nuclueus ($M=1836$). So we will
\begin{enumerate}
\item Solve the electronic motion with the nuclear coordinates fixed.
\item Solve the nuclear motion, assuming that the electron wavefunction adapts instantaneously.
\end{enumerate}

So the ansatz is:
\begin{equation}
\Psi(\vecRA, \vecRB, \mathbf{r}) = \psi_e(\vecRA, \vecRB, \mathbf{r})\cdot \psi_n(\vecRA, \vecRB)
\end{equation}

We will plug this into the Schr\selectlanguage{ngerman}ödinger equation to obtain:
\begin{equation}
\psi_n\hat{T}_e\psi_e +\psi_e\hat{T}_n\psi_n + V(\mathbf{r}, \vecRA, \vecRB)\psi_e \psi_n + W = E \psi_e\psi_n
\end{equation}

This transformation introduced the \textit{non-adiabatic} effects:
\begin{equation}
W = -\frac{1}{2M}\sum_{i=A,B} \left[(\nabla_{\mathbf{R}_i}\psi_e)\cdot(\nabla_{\mathbf{R}_i}\psi_n)+\psi_n \nabla_{\mathbf{R}_i}^2 \psi_e\right]
\end{equation}

In the following we will neglect these effects. And obtain:
\begin{equation}
\psi_e \hat{T}_n\psi_n + \left(\hat{T}_e\psi_e+V\psi_e\right)\psi_n = E \psi_e\psi_n
\end{equation}
So we will first solve the \textit{electronic motion}:
\begin{equation}
\left(\hat{T}_e+\hat{V}\right)\psi_e = E_e(\vecRA, \vecRB) \psi_e
\end{equation}
To be explicit we obtain for the ionic hydrogen:
\begin{equation}\label{Eq:Hamiltonian}
H_e = -\frac{1}{2} \nabla_\mathbf{r}^2-\frac{1}{r_A}-\frac{1}{r_B}+\frac{1}{R}
\end{equation}
At this stage we can just focus on the electronic part to understand the structure of simple diatomic molecules, while assuming that $R$ is an independent parameter. Most importantly, we will focus at usual on symmetries, which will tell us more about the allowed states in the system.

In the second step we will solve the nuclear motion:
\begin{equation}
\hat{T}_n\psi_n + E_e \psi_n = E \psi_n
\end{equation}

This nuclear motion will be at the origin of rotational and vibrational levels, which will be discussed in Lecture \cite{molecules}.




\section{Symmetries of the electronic wavefunction}
This discussion follows along similiar lines as for the hydrogen atom and the helium atom. We basically can categorize the different states by their properties. This will help us later enormously to understand allowed transition etc.

\subsection{Angular momentum}
For any (diatomic) molecule we break the spherical symmetry that we relied on for the atomic systems. This means that angular momentum is not a conserved quantity anymore.

However, the Hamiltonian \eqref{Eq:Hamiltonian} is invariant under the rotation around the axis of the diatomic molecule. One can verify that this implies that:
\begin{eqnarray}
[H_e, L_z] = 0\\
\Rightarrow L_z \psi_e = \pm \Lambda \psi_e (a.u.)
\end{eqnarray}
The reason is that $\hat{L}_z =\frac{1}{i}\partial_\varphi$ depends solely on the angle $\varphi$ and not on $R$. Here the quantum number can have the integer values $\Lambda= 0, 1, 2 , \cdots$. We also note them $\Sigma, \Pi, \Delta, \Phi$ or $\sigma, \pi, \delta, \phi$ for single electrons. 

\subsection{Parity}
We further have symmetry under parity operation for \textit{homo-nuclear, diatomic} molecules $A_2$, see \eqref{Eq:Hamiltonian}. This means that we have once more:
\begin{equation}
\hat{P}\psi_e(\mathbf{r}) = \pm\psi_e(\mathbf{r})
\end{equation}
In the same way as in the lecture on the Helium atom we distinguish the states by \textit{gerade} and \textit{ungerade}. So we then end up with something like $\Lambda_{u,g}^\pm$.

\subsection{Spin}

If the system does not have explicit spin-orbit coupling, the total spin $S$ of the system will be conserved. So the full notation for electronic states is typically:
\begin{equation}
 ^{2S+1}\Lambda^{\pm}_{g,u}
\end{equation}
Most of the time the ground state of the system is $^{1}\Sigma^{+}_{g}$.



\section{Stability of the ground state molecule}

We have now studied the symmetries that the system should have, but until now we did not discuss the most important question: Is this molecule stable ? Within the Born-Oppenheimer approximation, we can actually solve the ionic hydrogen molecule analytically (Chapter 9 of \cite{Demtr_der_2010}). The resulting \textbf{molecular potential curves} are shown in Fig. \ref{632456}.\selectlanguage{english}
\begin{figure}[h!]
\begin{center}
\includegraphics[width=0.70\columnwidth]{figures/Bildschirmfoto-2018-12-05-um-10-17-32/Bildschirmfoto-2018-12-05-um-10-17-32}
\caption{{Molecular potential curves for the molecular hydrogen ion. Figure is
taken from \protect\cite{mechanics}
{\label{632456}}%
}}
\end{center}
\end{figure}

\subsection{Linear combination of atomic orbitals}

The analytical solutions are rather bulky and not particularly instructive. One powerful idea, and very good approximation, is to decompose the molecule wavefunction over the atomic orbitals of its components. Going back to Fig. \ref{547852} we could make the simple Ansatz:
\begin{equation}
\psi_e(\mathbf{r})= c_1 \psi_{1s}(\vecra)+c_2 \psi_{1s}(\vecrb)
\end{equation}

Note, that we made a very simple Ansatz at this stage and we could decompose the system over a much larger set of excited states. But for pedagogical reason we will stick to the simple model at this stage. Going through the symmetry requirements, we find that we can write the full wavefunction as:
\begin{equation}\label{Eq:LCAO_WF}
\psi_{g,u}(\mathbf{r})= \frac{1}{\sqrt{2\pm2S}}\left(\psi_{1s}(\vecra)\pm \psi_{1s}(\vecrb)\right)
\end{equation}
The contribution $S$ describes the overlap of the two atomic orbitals
\begin{equation}
S =  \int d\mathbf{r}\psi_{1s}^*(\vecra)\psi_{1s}(\vecrb)
\end{equation}
We can then evaluate the energy of the two states through the variational principle:
\begin{align}
E_{g,u} &= \bra{\psi_{g,u}}\hat{H}_e\ket{\psi_{g,u}}\\
 &= \frac{1}{2\pm2S}\left(\bra{\psi_A}\pm \bra{\psi_B}\right)\hat{H}_e\left(\ket{\psi_A}\pm \ket{\psi_B}\right)\\
 &= \frac{E_{AA}\pm E_{AB}}{1\pm S}\\
\end{align}
The resulting energy surfaces are shown in Fig. \ref{779212}. In the most simplistic interpretation the gerade state does not have a node in the middle and it is therefore of smaller kinetic energy.\selectlanguage{english}
\begin{figure}[h!]
\begin{center}
\includegraphics[width=0.70\columnwidth]{figures/H2plus-bosurface/H2plus-bosurface}
\caption{{The energy surface of the LCAO for the
H\textsubscript{2}\textsuperscript{+~} molecule.
{\label{779212}}%
}}
\end{center}
\end{figure}

\section{ The neutral hydrogen molecule}

In the previous section we have seen how we can treat the coupling of the nuclei through the exchange of a single shared electron. However, we should now move on to the case of two neutral particles binding together. What is here the relevant mechanism ?\selectlanguage{english}
\begin{figure}[h!]
\begin{center}
\includegraphics[width=0.70\columnwidth]{figures/Bildschirmfoto-2017-12-06-um-08-39-21/HydrogenMolecule}
\caption{{The H\textsubscript{2} molecule.
{\label{546428}}%
}}
\end{center}
\end{figure}



In the following we will only consider the electronic part, which adds up too:

\begin{equation}
\hat{H} = -\frac{1}{2}\left(\nabla_{\mathbf{r}_1}^2+\nabla_{\mathbf{r}_2}^2\right)-\frac{1}{r_{A1}}-\frac{1}{r_{A2}}-\frac{1}{r_{B1}}-\frac{1}{r_{B2}}+\frac{1}{r_{12}}+\frac{1}{R}
\end{equation}

We can now rewrite this Hamiltonian in the more instructive form
\begin{eqnarray}
\hat{H} &= H_{0,1}+H_{0,2}+\frac{1}{r_{12}}+\frac{1}{R}\\
H_{0,i} &= -\frac{1}{2}\nabla_{\mathbf{r}_i}^2-\frac{1}{r_{A,i}}-\frac{1}{r_{B,i}}
\end{eqnarray}

We have can now use the results of the hydrogen ion to understand this system.

\begin{itemize}
\item We have for each electron the solution \eqref{Eq:LCAO_WF}.
\item In the next step, we have to put the two electrons properly within this orbit with $S=0$ and \textit{ignoring} the $e^-$ - $e^-$ interaction.
\end{itemize}
So we can make the Ansatz:
\begin{eqnarray}
\psi(\mathbf{r}_1, \mathbf{r}_2) &= \psi_{g}(\mathbf{r}_1)\cdot\psi_{g}(\mathbf{r}_2)\\
&= \frac{1}{2 + 2S}\left(\psi_{1s}(\mathbf{r}_{A1})+\psi_{1s}(\mathbf{r}_{B1})\right)\left(\psi_{1s}(\mathbf{r}_{A2})+\psi_{1s}(\mathbf{r}_{B2})\right)\\
&= \frac{1}{2 + 2S}\left(\psi_{1s}(\mathbf{r}_{A1})\psi_{1s}(\mathbf{r}_{B2})+\psi_{1s}(\mathbf{r}_{B1})\psi_{1s}(\mathbf{r}_{A2}) +\psi_{1s}(\mathbf{r}_{A1})\psi_{1s}(\mathbf{r}_{A2})+\psi_{1s}(\mathbf{r}_{B1})\psi_{1s}(\mathbf{r}_{B2}) \right)
\end{eqnarray}
The first two terms describe \textbf{kovalent binding}. They describe situations where each electron is associated with one core.  The last two terms describe \textbf{ionic binding} as one associated both electrons with a single atom and then looks one the attraction of another ionic core. This is quite similiar to the interaction in the $H_2^+$ molecule.

Within this approach, one actually finds a binding energy of $E_b = \SI{-2.64}{eV}$ at an equilibrium distance of $R_e = 1.4 a_0$. The experimentally measured values differs quite substantially as we have $E_b = \SI{-4.7}{eV}$. A substantial approximation was here that we neglected the interaction between the electrons, which should repel.



\subsection{The Heitler-London method}

As the two electrons should repel each other, we can assume that the ionic binding is strongly suppressed. So the wavefunction is now assumed to be:

\begin{eqnarray}
\psi_{HL} &= \frac{1}{\sqrt{2 + 2S^2}}\left(\psi_{1s}(\mathbf{r}_{A1})\psi_{1s}(\mathbf{r}_{B2})+\psi_{1s}(\mathbf{r}_{B1})\psi_{1s}(\mathbf{r}_{A2})\right)
\end{eqnarray}
Again, the wavefunction cannot be factorized and the two electrons are entangled because of the interactions. Recognize the common theme with the Helium atom. Calculation of the binding energy within this approximation leads to $E_b = \SI{-3.14}{eV}$ and $R_e=1.6a_0$.

In the next lecture \cite{moleculesa} we will discuss how we can move on from these extremely simple diatomic molecules to the assembly of richer systems.

\selectlanguage{english}
\FloatBarrier
\bibliographystyle{plainnat}
\bibliography{bibliography/converted_to_latex.bib%
}

\end{document}

