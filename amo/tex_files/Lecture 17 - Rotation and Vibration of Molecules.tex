\documentclass[10pt]{article}

\usepackage{fullpage}
\usepackage{setspace}
\usepackage{parskip}
\usepackage{titlesec}
\usepackage[section]{placeins}
\usepackage{xcolor}
\usepackage{breakcites}
\usepackage{lineno}
\usepackage{hyphenat}





\PassOptionsToPackage{hyphens}{url}
\usepackage[colorlinks = true,
            linkcolor = blue,
            urlcolor  = blue,
            citecolor = blue,
            anchorcolor = blue]{hyperref}
\usepackage{etoolbox}
\makeatletter
\patchcmd\@combinedblfloats{\box\@outputbox}{\unvbox\@outputbox}{}{%
  \errmessage{\noexpand\@combinedblfloats could not be patched}%
}%
\makeatother


\usepackage[round]{natbib}
\let\cite\citep




\renewenvironment{abstract}
  {{\bfseries\noindent{\abstractname}\par\nobreak}\footnotesize}
  {\bigskip}

\titlespacing{\section}{0pt}{*3}{*1}
\titlespacing{\subsection}{0pt}{*2}{*0.5}
\titlespacing{\subsubsection}{0pt}{*1.5}{0pt}


\usepackage{authblk}


\usepackage{graphicx}
\usepackage[space]{grffile}
\usepackage{latexsym}
\usepackage{textcomp}
\usepackage{longtable}
\usepackage{tabulary}
\usepackage{booktabs,array,multirow}
\usepackage{amsfonts,amsmath,amssymb}
\providecommand\citet{\cite}
\providecommand\citep{\cite}
\providecommand\citealt{\cite}
% You can conditionalize code for latexml or normal latex using this.
\newif\iflatexml\latexmlfalse
\AtBeginDocument{\DeclareGraphicsExtensions{.pdf,.PDF,.eps,.EPS,.png,.PNG,.tif,.TIF,.jpg,.JPG,.jpeg,.JPEG}}

\usepackage[utf8]{inputenc}
\usepackage[ngerman,english]{babel}








\usepackage{siunitx}
\usepackage{amsmath}
\newcommand{\bra}[1]{\ensuremath{\left\langle#1\right|}}
\newcommand{\ket}[1]{\ensuremath{\left|#1\right\rangle}}
\newcommand{\braket}[1]{\ensuremath{\left\langle#1\right\rangle}}
\newcommand{\rhohat}{\hat{\rho}}
\newcommand{\tr}[1]{\mathrm{tr}(#1)}
\newcommand{\trarb}[2]{\mathrm{tr}_{#1}(#2)}
\newcommand{\vv}[1]{\mathbf{#1}}
\newcommand*\dif{\mathop{}\!\mathrm{d}}
\newcommand{\eexp}[1]{\mathrm{e}^{#1}}
\newcommand*\ch[1]{\ensuremath{\mathrm{#1}}}
\newcommand{\vecRA}{\mathbf{R}_\mathrm{A}}
\newcommand{\vecRB}{\mathbf{R}_\mathrm{B}}


\begin{document}

\title{Lecture 17 - Rotation and Vibration of Molecules}



\author[1]{Fred Jendrzejewski}%
\author[2]{Selim Jochim}%
\author[2]{Matthias Weidemüller}%
\affil[1]{Kirchhoff-Institut für Physik}%
\affil[2]{Physikalisches Institut der Universität Heidelberg}%


\vspace{-1em}



  
  \date{January 07, 2025}


\begingroup
\let\center\flushleft
\let\endcenter\endflushleft
\maketitle
\endgroup





\selectlanguage{english}
\begin{abstract}
We will study the existance of vibrational and rotational levels in molecules. It allows us investigate the transitions of molecules and introduce the Franck-Condon principle. Finally, we will study how such intruiging molecules are used for the study of the permanent electric dipole moment of the electron.%
\end{abstract}%



\sloppy


We studied during the last two lectures the properties the electronic structure. For atoms the next step was the analysis of the transition rules to understand the spectrum. However, for molecules the electronic transition directly couple to the vibrational and rotational motion of the nuclei, which will have to study first.

\section{A short reminder on nuclear motion}
We discussed diatomic molecules, with $N$ electrons bound to the nuclei  \cite{molecules}. The full Hamiltonian of the molecule could be written in the following fashion:
\begin{equation}
\hat{H} = \hat{T}_e + \hat{T}_N + V(\hat{\vecRA}, \hat{\vecRB},\hat{\mathbf{r}}_1,\cdots, \hat{\mathbf{r}}_N)
\end{equation}
$\hat{T}_e$ describes the kinetic energy of the electrons, $\hat{T}_N$ the kinetic energy of the nuclei and $V$ the coupling between them. The we decomposed the full wavefunction over a nuclear part and an electronic part:
\begin{equation}
\Psi(\vecRA, \vecRB, \mathbf{r}_1,\cdots, \mathbf{r}_N) = \psi_e(\vecRA, \vecRB, \mathbf{r}_1,\cdots, \mathbf{r}_N)\cdot \psi_n(\vecRA, \vecRB)
\end{equation}
This allowed us to decouple nicely the two motions and study the properties of the electronic potentials first. In the Born-Oppenheimer approximation we obtained:
\begin{align}\label{Eq:NuclearMotion}
\hat{T}_N \psi_N + E_e (\vec{R}_\textrm{a}, \vec{R}_\textrm{b}) \psi_N = E \psi_N
\end{align}
%
with
\begin{align}
\hat{T}_N = - \frac{1}{2 M_\textrm{a}} \Delta_{\vec{R}_\textrm{a}} - \frac{1}{2 M_\textrm{b}} \Delta_{\vec{R}_\textrm{b}},
\end{align}
the total energy $E$ and the masses of the individual atoms $M_\textrm{a}$ and $M_\textrm{b}$.
For the electronic energy $E_e$, only $\vec{R} = \vec{R}_\textrm{a} - \vec{R}_\textrm{b}$ matters. We then calculated the molecular potential curves $E_e(R)$, which differ for each electronic configuration, discussed in the last lecture \cite{moleculesa} and sketched once more in Fig. \ref{907917}.\selectlanguage{english}
\begin{figure}[h!]
\begin{center}
\includegraphics[width=0.70\columnwidth]{figures/Bildschirmfoto-2018-12-07-um-14-27-34/Bildschirmfoto-2018-12-07-um-14-27-34}
\caption{{The molecular potential curves obtained from the Born-Oppenheimer
Approximation.
{\label{907917}}%
}}
\end{center}
\end{figure}


In the center of mass frame we can tranform \eqref{Eq:NuclearMotion} and get: 
\begin{align}
\left( - \frac{1}{2M} \Delta_{\vec{R}} + E_e (\vec{R}) \right) \psi_\textrm{n} (\vec{R}) = E \psi_\textrm{n} (\vec{R}),
\end{align}
$\vec{R}$ is spherically symmetric, and
\begin{align}
M = \frac{M_\textrm{a} \cdot M_\textrm{b}}{M_\textrm{a} + M_\textrm{b}}
\end{align}
is the reduced mass. This means that we can separate the angular  and radial motion to obtain:
\begin{align}
\psi_\textrm{n} (R,\theta,\varphi) = \frac{1}{R} S(R) \cdot Y_l^m (\theta,\varphi)
\end{align}
They describe the rotational and vibrational levels of the nucleus.


\section{Rotations}
If we assume a ``rigid'' molecule where the distance between the atoms is fixed, the rotational energy is simply given by:
\begin{align}
E_\text{rot} (R) = \frac{J\cdot (J+1)}{2M R^2}\, (\text{a.u.})
\end{align}
where $M$ is the reduced mass of the nuclei in atomic units and $J$ is the angular momentum quantum number. The factor $MR^2$ represents the moment of inertia. For more complex atoms the relationship is not quite as simple and the rotational energy is typically described by the moment of interia $I_{ij}$. The Hamiltonian for this rotation reads then:
\begin{align}
\hat{H}_{rot} =\frac{J_x^2}{2I_{xx}}+\frac{J_y^2}{2I_{yy}}+\frac{J_z^2}{2I_{zz}}
\end{align}
The molecule H\textsubscript{2} has then a rotational frequency $\omega/2\pi =3\cdot 10^{12}\cdot J(J+1) \si{Hz}$.

\section{Vibrations}
As already known from the hydrogen atom we can use the angular solutions to discuss the radial solutions. We have to solve now:
\begin{align}
\left( \frac{1}{2M} \frac{d^2}{dR^2} + E_e (R) + \frac{1}{2M} \frac{J(J+1)}{R^2} \right) S(r) = E_\text{vib} S(R)
\end{align}

If the extension from the minimum (see \ref{584527}) is small, we can approximate it by a harmonic potential.
%
We can then find a vibrational energy $E_\text{vib} = \omega_0 (\nu+\frac{1}{2})\, \nu=0,1,\cdots$
%
The harmonic expansion around the minimum reads:
\begin{align}
E_e \approx E_0 + \frac{1}{2} M \omega_0^2 (R-R_0 )^2
\end{align}

For the example of H\textsubscript{2}, we get $\omega/2\pi \sim 10^{14}$ Hz.\selectlanguage{english}
\begin{figure}[h!]
\begin{center}
\includegraphics[width=0.70\columnwidth]{figures/Bildschirmfoto-2018-10-01-um-14-46-36/Bildschirmfoto-2018-10-01-um-14-46-36}
\caption{{Internuclear potential. In the limit of a harmonic expansion around the
minimum, the vibrationally excited states are equidistant.
{\label{584527}}%
}}
\end{center}
\end{figure}

A better approximation of the vibrational level structure than the simple harmonic oscillator is the \textbf{Morse potential}. 
\subsection{The Morse potential}

In this case we approximate the molecular potential curves by:
\begin{align}
E_e(R) &\approx V_\text{morse}(R)\\
V_\text{morse}(R) &=hcD_e(1-e^{-ax})^2\text{ with }a =\sqrt{\frac{k}{2hcD_e}}\\
x &= R-R_0
\end{align}
Its particular usefulness stems from the fact that it is still analyitically solvable and we obtain:
\begin{align}
E_{vib}=(\nu+\frac{1}{2})\hbar \omega-(\nu+\frac{1}{2})^2 \hbar \omega x_e\\
\omega x_e = \frac{a^2\hbar}{2M}
\end{align}

$x_e$ is then called the anharmonicity parameter.

\section{Molecular transitions}
We are now ready to discuss the different transitions that might appear in the spectrum. And we will work our way through the different levels of energy as we will see that they are all coupled.

\subsection{Rotational transitions}
We will start out with the transitions of the lowest frequency, the rotational transitions. So, we would like to know if it is possible to transition from a state $\ket{\epsilon, J, M_J}$ to another state $\ket{\epsilon, J', M_J'}$, where $\epsilon$ describes the electronic and vibrational degree of freedom. This means that we have to calculate as usual the the electric dipole moment:
\begin{align}
\bra{\epsilon, J', M_J'} \vec{D}\ket{\epsilon, J, M_J}
\end{align}
Within the Born-Oppenheimer approximation electronic and rotational degree of freedom decouple and we can write: 
\begin{align}
\bra{\epsilon, J', M_J'} \vec{D}\ket{\epsilon, J, M_J} = \bra{J', M_J'} \bra{\epsilon}\vec{D}\ket{\epsilon}\ket{ J, M_J}
\end{align}
This electric dipole transitions were forbidden in atoms as they do not have a permanent electric dipole moment. However, hetero-nuclear atoms can have such a permanent electric dipole moment, they are called polar molecules. Examples are alkali-alkali molecules like NaK, NaCs, KRb whose permenanent electric dipole moment can be up $3 ea_0$ \cite{Qu_m_ner_2012}. It follows that:
\begin{itemize}
\item \textbf{Pure rotational transitions exist in polar molecules.}
\end{itemize}
The transition rules are di-atomic molecules: $\Delta J = \pm 1$ and $\Delta M_J= 0, \pm1$. For more complex molecules these transition rules can vary quite substantially as the rotational degree of freedom might have to be described by an additional quantum number.

\subsection{Vibrational transitions}
In the next step, we would like to understand the transitions between different vibrational levels. Hence, we are investigating the electric dipole moment 
\begin{align}
\bra{\epsilon, \nu'} \vec{D}\ket{\epsilon, \nu}= \bra{\nu'} \vec{D}_\epsilon\ket{\nu}
\end{align}
The evaluation is now not quite as simple as for the rotational degree of freedom as both $\nu$ and $\epsilon$ will influence the length of the molecule, they both directly depend on $R$. We can develop the electric dipole moment as a function of distance from the equilibrium and write then:
\begin{align}
\bra{\nu'} \vec{D}_\epsilon\ket{\nu}&= \bra{\nu'} \left(\vec{D}_\epsilon(0)+ \frac{d\vec{D}_\epsilon}{dx}x+\cdots\right)\ket{\nu}\\
&= \frac{d\vec{D}_\epsilon}{dx}\bra{\nu'}x\ket{\nu}+\cdots
\end{align}
So vibrational transistions will only happen in molecules for which the permanent electric dipole changes as a function of distance. Once again they are non-existant in homo-nuclear molecules.

\section{Vibronic transitions}

At this stage, we are ready to discuss electronic transitions. If we are performing an electronic transition this also implies a change on the molecular potential curve as indicated in Fig. \ref{121419}. Imagine now the transition of the ground state molecular branch (called the X branch) to a higher electronic shell (called A, B, C, ...). Such a transition will happen at constant internuclear radius as they are much faster than the nuclei motion. This implies that an electronic transition will typically excite the molecule into a high vibrational branch. The dipole moment is then proportional too:
\begin{align}
\bra{\epsilon', \nu'} \vec{D}\ket{\epsilon, \nu}\approx \vec{D}_{\epsilon, \epsilon'} \bra{\nu'} \ket{\nu}
\end{align}\selectlanguage{english}
\begin{figure}[h!]
\begin{center}
\includegraphics[width=0.70\columnwidth]{figures/Bildschirmfoto-2018-12-11-um-12-47-56/Bildschirmfoto-2018-12-11-um-12.58.06}
\caption{{The Franck-Condon principle for a simple toy model.
{\label{121419}}%
}}
\end{center}
\end{figure}

The factor $S(\nu, \nu')=|\bra{\nu}\ket{\nu'}|^2$ is then called the Franck-Condon factor
and it describes the strength of the transitions.

It is exactly this coupling of different hierarchies that makes the molecular spectra so rich and also extremely tough to control.


\section{Can we get into the groundstate ?}

Given all the complexities of molecules it seems non-trivial to find a scheme that gets them into the ground state. For atoms laser cooling has proven very efficient as we will discuss later. However, it mainly adresses the cooling of external degrees of freedom. In molecules a significant amount of energy its in the rotational and vibrational levels. In this connection, a beautiful solution has been demonstrated in \cite{Ni_2008}. 

The scheme is visualized in Fig. \ref{898773}.\selectlanguage{english}
\begin{figure}[h!]
\begin{center}
\includegraphics[width=0.70\columnwidth]{figures/Bildschirmfoto-2018-12-12-um-10-32-59/Bildschirmfoto-2018-12-12-um-10-32-59}
\caption{{Production of groundstate molecules of K + Rb. Figure is taken from
\protect\cite{Ni_2008}
{\label{898773}}%
}}
\end{center}
\end{figure}

In a first step the atoms are cooled and then associated to a highly excited molecule in the a\textsuperscript{3}$\Sigma$ state. From there the atom has to be transferred down in to the ground state $\ket{g}$. A direct thransfer is not possible as the Franck-Condon factors do not allow for it. Another path is to go through an intermediate level (here the 2$^3 \Sigma$ level), which has overlap with both of them. However, this level has typically overlap with plenty of other levels and a finite lifetime. How can we then optimize the transfer ? The idea is to use the concept of dark states in the triplet of $\{i, e, g\}$.

\subsection{The dark states in three level systems}
We can visualize the idea of the dark state transfer through the following Hamiltonian:
\begin{align}
\hat{H}&= \Omega_1\left(\ket{i}\bra{e}+\ket{e}\bra{i}\right)+\Omega_2\left(\ket{g}\bra{e}+\ket{e}\bra{g}\right)
\end{align}
We can rewrite it as:
\begin{align}
\hat{H}&= (\Omega_1\ket{i}+\Omega_2\ket{g})\bra{e}+\ket{e}(\Omega_1\bra{i}+\Omega_2\bra{g})\\
&\propto\ket{B}\bra{e}+\ket{e}\bra{B}\\
\ket{B}&= \frac{\Omega_1\ket{i}+\Omega_2\ket{g}}{\sqrt{\Omega_1^2+\Omega_2^2}}
\end{align}
So in the three level scheme the excited state is always could to the so-called bright state, which is a coherent superposition of $\ket{g}$ and $\ket{i}$. The orthogonal state is the dark state:
\begin{align}
\ket{D}&= \frac{\Omega_1\ket{g}-\Omega_2\ket{i}}{\sqrt{\Omega_1^2+\Omega_2^2}}\\
\langle B| D\rangle &= 0
\end{align}
Now we can also discuss the transfer sequence non as STIRAP (stimulated Raman adiabatic passage).

\subsection{STIRAP}

STIRAP transfers the loosely bound molecules coherently into the groundstate without ever passing through the lossy excited level. It has the following steps:
\begin{enumerate}
\item The dressing laser $\Omega_2$ is ramped on. The initial $\ket{i}$ is now the dark state.
\item The coupling laser $\Omega_1$ is ramped on, while the laser $\Omega_2$ is ramped down. This transfers the $\ket{i}$ adiabatically into the state $\ket{g}$, which is the dark state for fully switched of $\Omega_2$. 
\end{enumerate}
The molecules are now in the groundstate with a transfer efficiency of roughly $50\%$.

\section{Measurement of the electron electric dipole moment}
Despite their complexity, molecules can be an enormously powerful tool for precision measurements \cite{Safronova_2018,particles}. The test of the existance of a permanent electric dipole moment (electron edm) of the electron is one of these tests. 

What does does the existance of electron edm actually mean ? We have already discussed quite heavily the existance of a permanent edm for polar molecules. The amplitude of their dipole moment is in the order of a few $ea_0$, which is also the natural unit for the induced edm of atoms. One could now also imagine that the electron itself has an edm, which is aligned with its spin $\vec{D}_e = d_e \vec{s}_e$. The standard model actually predicts such a permanent electron edm, but only of the amplitude $d_e \approx 10^{-30}ea_0$, which is fantastically small . However, the search continues as most extensions of the standard model actually predict substantially higher values as summarized in Fig. \ref{421559}. As we can see the most precise measurments are actually performed in very heavy di-atomic molecules. 

In these molecules the electron 'feels' enormous effective electric fields, which can reach the several GV/cm regime \cite{Cairncross_2017}.\selectlanguage{english}
\begin{figure}[h!]
\begin{center}
\includegraphics[width=0.70\columnwidth]{figures/Bildschirmfoto-2018-12-11-um-14-16-30/Bildschirmfoto-2018-12-11-um-14-16-30}
\caption{{Search for the permanent electric dipole moment. Figure is taken from
\protect\cite{experiment}
{\label{421559}}%
}}
\end{center}
\end{figure}

The search for the dipole moment is then testing the dependence of the electron energy:
\begin{align}
E_\pm  = \pm(\mu B_0 + d_e E)
\end{align}
This energy difference can be read out through Ramsey spectroscopy.  Switching the electric field allows then to switch the frequency difference by $\hbar \delta \omega = 4d_e E$. Only an upper limit is known up to now $|d_e|< 8.7 e cm$\cite{2013}.

\selectlanguage{english}
\FloatBarrier
\bibliographystyle{plainnat}
\bibliography{bibliography/converted_to_latex.bib%
}

\end{document}

