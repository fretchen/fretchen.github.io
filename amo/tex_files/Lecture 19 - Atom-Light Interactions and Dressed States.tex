\documentclass[10pt]{article}

\usepackage{fullpage}
\usepackage{setspace}
\usepackage{parskip}
\usepackage{titlesec}
\usepackage[section]{placeins}
\usepackage{xcolor}
\usepackage{breakcites}
\usepackage{lineno}
\usepackage{hyphenat}





\PassOptionsToPackage{hyphens}{url}
\usepackage[colorlinks = true,
            linkcolor = blue,
            urlcolor  = blue,
            citecolor = blue,
            anchorcolor = blue]{hyperref}
\usepackage{etoolbox}
\makeatletter
\patchcmd\@combinedblfloats{\box\@outputbox}{\unvbox\@outputbox}{}{%
  \errmessage{\noexpand\@combinedblfloats could not be patched}%
}%
\makeatother


\usepackage[round]{natbib}
\let\cite\citep




\renewenvironment{abstract}
  {{\bfseries\noindent{\abstractname}\par\nobreak}\footnotesize}
  {\bigskip}

\titlespacing{\section}{0pt}{*3}{*1}
\titlespacing{\subsection}{0pt}{*2}{*0.5}
\titlespacing{\subsubsection}{0pt}{*1.5}{0pt}


\usepackage{authblk}


\usepackage{graphicx}
\usepackage[space]{grffile}
\usepackage{latexsym}
\usepackage{textcomp}
\usepackage{longtable}
\usepackage{tabulary}
\usepackage{booktabs,array,multirow}
\usepackage{amsfonts,amsmath,amssymb}
\providecommand\citet{\cite}
\providecommand\citep{\cite}
\providecommand\citealt{\cite}
% You can conditionalize code for latexml or normal latex using this.
\newif\iflatexml\latexmlfalse
\AtBeginDocument{\DeclareGraphicsExtensions{.pdf,.PDF,.eps,.EPS,.png,.PNG,.tif,.TIF,.jpg,.JPG,.jpeg,.JPEG}}

\usepackage[utf8]{inputenc}
\usepackage[ngerman,english]{babel}








\usepackage{siunitx}
\usepackage{amsmath}
\newcommand{\bra}[1]{\ensuremath{\left\langle#1\right|}}
\newcommand{\ket}[1]{\ensuremath{\left|#1\right\rangle}}
\newcommand{\braket}[1]{\ensuremath{\left\langle#1\right\rangle}}
\newcommand{\rhohat}{\hat{\rho}}
\newcommand{\tr}[1]{\mathrm{tr}(#1)}
\newcommand{\trarb}[2]{\mathrm{tr}_{#1}(#2)}
\newcommand{\vv}[1]{\mathbf{#1}}
\newcommand*\dif{\mathop{}\!\mathrm{d}}
\newcommand{\eexp}[1]{\mathrm{e}^{#1}}
\newcommand*\ch[1]{\ensuremath{\mathrm{#1}}}

\begin{document}

\title{Lecture 19 - Atom-Light Interactions and Dressed States}



\author[1]{Fred Jendrzejewski}%
\author[2]{Selim Jochim}%
\author[2]{Matthias Weidemüller}%
\affil[1]{Kirchhoff-Institut für Physik}%
\affil[2]{Physikalisches Institut der Universität Heidelberg}%


\vspace{-1em}



  
  \date{January 07, 2025}


\begingroup
\let\center\flushleft
\let\endcenter\endflushleft
\maketitle
\endgroup








\sloppy


We have seen that we can understand matter with increasing complexity from the simple two-level system up to molecules. We further studied, how they can be control by classical electromagnetic fields to a very high accuracy.

In the last lecture \cite{field} we also studied how we can understand the electromagnetic field as an ensemble of quantum mechanical photon modes. So in today's lecture we will focus on the interaction between atoms and light, which is in a particularly clean set-up, namely cavity quantum electrodynamics. The fundamental ingredients are sketched in Fig. \ref{305911}.\selectlanguage{english}
\begin{figure}[h!]
\begin{center}
\includegraphics[width=0.70\columnwidth]{figures/Bildschirmfoto-2018-12-18-um-09-15-55/Bildschirmfoto-2018-12-18-um-09-15-55}
\caption{{Rydberg atoms interacting with the photons confined in a high quality
cavity. Picture taken from the Nobel prize announcement for Serge
Haroche \protect\cite{2012a}
{\label{305911}}%
}}
\end{center}
\end{figure}

They are:
\begin{itemize}
\item The electric field confined in a high finesse cavity. It will bescribed by $\hat{H}_\textrm{f}$.
\item An atom transversing the cavity, described by $\hat{H}_\textrm{a}$.
\item The interaction between the atomic charge and the electric field of the cavity described by $\hat{H}_\textrm{af}$.
\end{itemize}

The Hamiltonian reads:

\begin{align} \label{eq:totham}
\hat{H}_0 = \hat{H}_\textrm{a} + \hat{H}_\textrm{f} + \hat{H}_\textrm{af}
\end{align}

\section{The qubit system}

The first ingredient of the Hamiltonian is the qubit system. Several widely studied system that we will come back to exist. The most widely studied are:
\begin{enumerate}
\item The internal qubit states of ions.
\item The transmon qubit in superconducting systems.
\end{enumerate}

Another approach are well isolated states in atoms, namely \textbf{Rydberg} states.

\subsection{Rydberg atoms}
The Rydberg states are highly excited states of Alkali atoms, which have only one electron on the outer shell. As such they are similiar to the  hydrogen atom and hence they can be well described within atomic physics. In the hydrogen atom the energy states are described by the principle quantum number $n$ \cite{Jendrzejewski, Jendrzejewskib}:
\begin{align}
E_n = -E_I \frac{1}{n^2} \text{ with }E_I = 13.6 eV
\end{align}
The typically employed Rydberg states are then in the order of $n\approx 50$, such that the energy difference between two neighboring states is in the order of a few 50 GHz. Focusing only on two of those states we can write the Hamiltonian as:
\begin{align}
\hat{H}_\textrm{a} &= \frac{\hbar\omega_0}{2} \left(\ket{e}\bra{e}-\ket{g}\bra{g} \right)
\end{align}
Through the remainder of the lecture we will frequently switch notations between the spin language and the two-level system:
\begin{align}
\hat{\sigma}_z &= \ket{e}\bra{e}-\ket{g}\bra{g}\\
\hat{\sigma}_+ &= \ket{e}\bra{g}\\
\hat{\sigma}_- &= \ket{g}\bra{e}
\end{align}

Using these highly excited states has several advantages:
\begin{itemize}
\item The energy spacing of a few GHz falls into the regime of microwaves, which are extremely precisely controlled.
\item Given the high quantum number, the electron is typically far away from the nucleus and the induced dipole moments can be rather large.As a such a strong coupling between light-field and qubit seems achievable.
\item The lifetime of the Rydberg states is in the order of a few microseconds, which can be long compared to most other time scales within the experiments.
\end{itemize}

The next step is to couple qubit to a suitable cavity.



\section{The cavity field}
The atom has be coupled to a suitable electric field. The electric field reads in general \cite{field}:
\begin{align}
\hat{E}(\vec{r})&= i \int \frac{d\vec{k}}{(2\pi)^{3/2}}\sum_i \left(\frac{\hbar\omega}{2\epsilon_0}\right)\vec{e}_i\left(\hat{a}_i(\vec{k})e^{i\vec{k}\vec{r}}-\hat{a}^\dag_i(\vec{k})e^{-i\vec{k}\vec{r}}\right)
\end{align}
We can simplify it a lot by working in a suitable cavity. The most important properties of the cavity are the:
\begin{itemize}
\item The resonant frequency $\omega_L$ of the light trapped in the cavity.
\item  The quality factor $Q$, which describes the number of round trips the photon makes within the cavity.
\end{itemize}

The cavities employed for cavity electrodynamics in Paris are made of superconducting material and feature quality factors of up to $10^{10}$. For such high quality factors the electric field can be well reduced to a single relevant mode \footnote{We chose the phase of the electric field such that we can eliminate the minus sign in the Hamiltonian}:
\begin{align}
\hat{E} &\sim (\hat{a} + \hat{a}^\dag)
\end{align}
The full Hamiltonian of the electromagnetic field reads then
\footnote{We ignore the energy of the quantum vacuum as it is not relevant for the following discussions}:
\begin{align}
\hat{H}_\textrm{f} &= \hbar\omega_L \hat{a}^\dag \hat{a}
\end{align}
The $\hat{a}$ is the raising operator for the electro-magnetic field. We typcially describe the electric field in the Fock basis of $\ket{n}$.
\begin{align}
\hat{n} \ket{n} &= n \ket{n}
\end{align}
While is the natural choice for the given Hamiltonian, this is obviously not the natural basis of the raising and lowering operators:
\begin{align}
\hat{a} \ket{n} &= \sqrt{n} \ket{n-1}\\
\hat{a}^\dag \ket{n} &= \sqrt{n+1} \ket{n+1}
\end{align}
From those we can construct any Fock state as:
\begin{align}
\ket{n} &= \frac{\left(\hat{a}^\dag\right)^n}{n!}\ket{0}
\end{align}
But experimentally we rarely manipulate the Hamiltonian directly, we much rather control the electric field, which is proportional to the raising and lowering operators. As such, photon states are widely described in the basis of \textbf{coherent} states:
\begin{align}
\hat{a}\ket{\alpha}&= \alpha\ket{\alpha}
\end{align}
So the eigenvalues are complex numbers corresponding to the complex electric field amplitudes we know from classical optics. To make a connection to the Fock space we can then use the above definitions to write:
\begin{align}
\ket{\alpha} &=e^{-|\alpha|^2/2}\sum_n \frac{\alpha^n}{\sqrt{n!}}\ket{n}
\end{align}
A very useful visualization of the coherent states happens in phase space $X = \frac{a+a^\dag}{2}$ and $P = i\frac{a-a^\dag}{2}$. They are Gaussian wave packages displaced by an amplitude $|\alpha|$ and rotating at speed $\omega_L$. 

\section{The atom-field interaction}
Finally, we have to describe interaction between the atoms and the field. 
%
Interactions between the atoms and the light field are governed by the electric dipole interaction between the atom and the light
\begin{align}
\hat{H}_\textrm{af}&= -\hat{\vec{D}} \cdot \hat{\vec{E}}
\end{align}
We can expand the dipole operator over the two levels of the atom:
\begin{align}
\hat{\vec{D}} = \vec{d}\left(\ket{g}\bra{e}+\ket{e}\bra{g}\right)\\
\hat{\vec{D}} = \vec{d}\left(\hat{\sigma}_- + \hat{\sigma}_+\right)
\end{align}
We can now write:
\begin{align}
\hat{H}_\textrm{af}&= \frac{\hbar \Omega_0}{2} \left(\ket{g}\bra{e}+\ket{e}\bra{g}\right)\left(\hat{a}+\hat{a}^\dag\right)
\end{align}

Multiplying out the different the two brackets leads to two processes of the type:
\begin{itemize}
\item $\hat{\sigma}_- \hat{a}^\dag$, which describes the emission of a photon by deexcitation of the atom.
\item  $\hat{\sigma}_+ \hat{a}$, which describes the absorption of a photon by excitation of the atom.
\end{itemize}
The other two processes are strongly off-resonant and we can typically ignore them. This approximation consists in the rotating wave approximation, discussed in lecture 4 \cite{Jendrzejewski}. The coupling hamiltonian reads then:

\begin{align}
\hat{H}_\textrm{af}&= \frac{\hbar \Omega_0}{2} \left(\hat{\sigma}_- a^\dag +\hat{\sigma}_+ a\right)
\end{align}

We can put the full Hamiltonian together to obtain the Jaynes-Cummings model:
\begin{align}
H_{JC}&=\hbar\omega_0 \ket{e}\bra{e} +\hbar\omega_L \hat{a}^\dag \hat{a} +\frac{\hbar \Omega_0}{2} \left(\hat{\sigma}_- a^\dag +\hat{\sigma}_+ a\right)
\end{align}


\section{Dressed Atom Picture}
We can now analyze the Hamiltonian step-by-step in the dressed atom picture.

\subsection{Optional: Bare States}
Let us first look at the unperturbed ("bare") states, ignoring $H_\textrm{af}$ and depict the ground and excited state $\ket{g}$ and $\ket{e}$ of the atom on an energy scale, as shown in \ref{231072}.\selectlanguage{english}
\begin{figure}[h!]
\begin{center}
\includegraphics[width=0.70\columnwidth]{figures/Bildschirmfoto-2018-10-01-um-15-24-39/Bildschirmfoto-2018-10-01-um-15-24-39}
\caption{{The ground and excited state~\(\left|g\right\rangle\) and~\(\left|e\right\rangle\)
of the atom on an energy scale. Here, the energy of the ground state
is~\(E_g\ =\ 0\). The energy difference between the two states is
\(\hbar\omega_0\). A photon of the surrounding light field has an
energy \(\hbar\omega_L\).
{\label{231072}}%
}}
\end{center}
\end{figure}

The Hamiltonian of the system then reads
\begin{align}
\hat{H} = \hat{H}_\textrm{a} + \hat{H}_\textrm{f} = \hbar \omega_0 \ket{e} \bra{e} + \hbar \omega_\textrm{L} \hat{a}^\dag \hat{a} 
\end{align}

The Hilbert space of this Hamiltonian contains both the state of the atom and the state of the field. We can write them as product states of the form 
\begin{align}
\ket{g/e, n} = \ket{g/e} \otimes \ket{n}
\end{align}
The left substate of the tensor product denotes the state of the atom and the right substate is defined by the number of photons in the external field. We then have:
\begin{align}
\hat{H}\ket{g/e, n} = \left(\hbar\omega_0 \delta_{g/e,e}+\hbar\omega_L n\right)\ket{g/e, n}
\end{align}

We will assume that there is a very small detuning $\delta_\textrm{l}$ between the atom and the light field:
\begin{align}
|\delta_\textrm{l}| = |\omega_\textrm{L}- \omega_0 | \ll \omega_0
\end{align}\selectlanguage{english}
\begin{figure}[h!]
\begin{center}
\includegraphics[width=0.70\columnwidth]{figures/Bildschirmfoto-2018-10-01-um-15-24-47/Bildschirmfoto-2018-10-01-um-15-24-47}
\caption{{An energy diagram of the bare states. Note that the interaction between
atom and light field has not yet been introduced!
{\label{871990}}%
}}
\end{center}
\end{figure}

We can draw another energy diagram (see \ref{871990}), where the state of the atom and the state of the light field are contained in one ``bare'' state. It does \emph{not} yet include the interaction between the atom and the light field. From the diagram one can see that the states forming the manifold
\begin{align}
\Sigma (n) = \left\{ \ket{g,n+1}, \ket{e,n} \right\}
\end{align}
are almost degenerate.

\subsection{Dressed States}

$\hat{H}_\textrm{af}$ couples now only the two states within each manifold $\Sigma(n)= \left\{ \ket{g,n+1}, \ket{e,n} \right\}$. We thus obtain a two-state system (see lecture 3 \cite{Jendrzejewskia})  for which we can write:

\begin{align}
\left( \begin{array}{c} c_1 \\ c_2 \end{array} \right) \equiv c_1 \ket{g,n+1} + c_2 \ket{e,n}
\end{align}
The off-diagonal matric element reads:
\begin{align}
h_n = \braket{e,n|\hat{H}_\textrm{af}|g,n+1} = \frac{\hbar \Omega_0}{2} \sqrt{n+1}.
\end{align}
Note that the square of the matrix element is proportional to $I \propto (n+1) \approx n$ for large $n$.
The three Hamiltonians in \eqref{eq:totham} can then be written in matrix notation and the total Hamiltonian can be constructed:

\begin{align}
H_\textrm{a} &= \left( \begin{array}{cc} 0 & 0 \\ 0 & \hbar \omega_0 \end{array} \right),\\
H_\textrm{f} &= \left( \begin{array}{cc} (n+1)\hbar \omega_\textrm{l} & 0 \\ 0 & n \hbar \omega_\textrm{l} \end{array} \right) =  \left( \begin{array}{cc} \hbar \omega_\textrm{l} & 0 \\ 0 & 0 \end{array} \right) + n \hbar \omega_\textrm{l} \cdot \mathbb{1},\\
H_\textrm{af} &= \left( \begin{array}{cc} 0 & h_n \\ h_n & 0 \end{array} \right), \qquad \text{where} \qquad h_n = \hbar \frac{\Omega_0}{2}\sqrt{n+1},\\
\hat{H} &= \left( \begin{array}{cc} \hbar \omega_\textrm{l} & h_n \\ h_n & \hbar \omega_0 \end{array} \right) + n \hbar \omega_\textrm{l} \cdot \mathbb{1},
\end{align}
The ``dressed states'' are obtained by diagonalizing $\hat{H}$ within $\Sigma(n)$, which es effectively once again a two level system \ref{180224} shows an energy diagram including the bare and the dressed states. The energy difference between the states $\ket{1(n)}$ and $\ket{2(n)}$ is 
\begin{align}
\hbar \Omega = \hbar \sqrt{\delta_\textrm{l}^2 + \Omega_0^2}
\end{align}
with the effective Rabi frequency $\Omega$.\selectlanguage{english}
\begin{figure}[h!]
\begin{center}
\includegraphics[width=0.70\columnwidth]{figures/Bildschirmfoto-2018-10-01-um-15-24-57/Bildschirmfoto-2018-10-01-um-15-24-57}
\caption{{An energy diagram showing the bare and the dressed states.
{\label{180224}}%
}}
\end{center}
\end{figure}

The corresponding eigenvectors are then a mixture of atom and light as visualized in Fig \ref{633447}.\selectlanguage{english}
\begin{figure}[h!]
\begin{center}
\includegraphics[width=0.70\columnwidth]{figures/Bildschirmfoto-2018-10-01-um-15-25-06/Bildschirmfoto-2018-10-01-um-15-25-06}
\caption{{Energies of the bare and dressed states as a function of
the~\(\delta_L\).
{\label{633447}}%
}}
\end{center}
\end{figure}

\section{Quantum Rabi oscillations}

In the resonant case, the Jaynes-Cummings model simply describes a two-level system that is coupled by a quantized Rabi coupling strength
\begin{align}
\Omega_n = \Omega_0 \sqrt{n+1}
\end{align}
So even for an empty cavity the vacuum is predicted to induce Rabi coupling, if it is switched on and off. If more than one photon is in the cavity the oscillation is simply a superposition of several coupling strength:
\begin{align}
P_e(t)=\sum_n p_n\frac{1+\cos\left[\Omega_n t\right]}{2}
\end{align}
This effect has been observed in Ref. \cite{Brune_1996} as summarized in Fig. \ref{910858}.\selectlanguage{english}
\begin{figure}[h!]
\begin{center}
\includegraphics[width=0.70\columnwidth]{figures/Bildschirmfoto-2018-12-18-um-14-07-52/Bildschirmfoto-2018-12-18-um-14-07-52}
\caption{{Observation of quantum Rabi oscillations in \protect\cite{Brune_1996}
{\label{910858}}%
}}
\end{center}
\end{figure}



\selectlanguage{ngerman}\section{Making Schrödingers kitten}

In the previous section the cavity was tuned exactly on resonance with the incoming Rydberg atoms, such that coherent oscillations where possible. On the other hand it is possible to work in the regime, where $\delta_L$ is much larger than the Rabi coupling. In this 'dispersive' regime the atom does not change its internal state, but it only picks up a phase $\Phi_0 =  \frac{\delta_L^2}{4\Omega} T_R$. The inverse of the phase is then imprinted onto the electric field in the cavity. To create a kitten state the experiment goes as follows:
\begin{enumerate}
\item The cavity is filled by a coherent state $\alpha$.
\item The Rydberg atom is prepare in a superposition state $\frac{\ket{e}+\ket{g}}{\sqrt{2}}$
\item The atom now interacts with the cavity for the time $T_R$. At the end, the entangled state is created:
\end{enumerate}
\begin{align}
\ket{\Psi}_1= \frac{e^{-i\Phi_0}\ket{e, \alpha e^{-i\Phi_0}}+\ket{g,\alpha e^{i\Phi_0}}}{\sqrt{2}}
\end{align}
This is the typical situation of Scrödingers cat. Pushing to the extreme case $\Phi_0 = \frac{\pi}{2}$ we entangled the atom with the state $\ket{\pm i\alpha}$. As $\alpha$ is a complex number we entangled a single atom with a large 'cat' state. As the atom is detected it projects the full Schrödinger cat onto the dead or alive state. This projection can be avoided by adding a second Ramsey pulse, which mixes once again the states $\ket{e}$ and $\ket{g}$:
\begin{align}
\ket{e}\rightarrow\frac{\ket{e}+e^{i\varphi}\ket{g}}{\sqrt{2}}\\
\ket{g}\rightarrow\frac{\ket{g}-e^{-i\varphi}\ket{e}}{\sqrt{2}}
\end{align}
The field now becomes:
\begin{align}
\ket{\Psi_2} &= \frac{1}{2}\ket{e}\otimes\left[e^{-i\Phi_0}\ket{\alpha e^{-i\Phi_0}}-e^{-i\varphi}\ket{\alpha e^{i\Phi_0}}\right]+\frac{1}{2}\ket{g}\otimes\left[e^{i(\varphi-\Phi_0)}\ket{\alpha e^{-i\Phi_0}}+\ket{\alpha e^{i\Phi_0}}\right]
\end{align}
The final read-out is then given by:
\begin{align}
P_e &= \frac{1}{2}\left(1- e^{-n(1-\cos(2\Phi_0))}\cos(\varphi-\Phi_0-n\sin(2\Phi_0))\right)
\end{align}
So the presence of the cat leads to a phase shift and a decrease in fringe contrast. This was observed in the experiments \selectlanguage{ngerman}\section{Making Schrödingers kitten}

In the previous section the cavity was tuned exactly on resonance with the incoming Rydberg atoms, such that coherent oscillations where possible. On the other hand it is possible to work in the regime, where $\delta_L$ is much larger than the Rabi coupling. In this 'dispersive' regime the atom does not change its internal state, but it only picks up a phase $\Phi_0 =  \frac{\delta_L^2}{4\Omega} T_R$. The inverse of the phase is then imprinted onto the electric field in the cavity. To create a kitten state the experiment goes as follows:
\begin{enumerate}
\item The cavity is filled by a coherent state $\alpha$.
\item The Rydberg atom is prepare in a superposition state $\frac{\ket{e}+\ket{g}}{\sqrt{2}}$
\item The atom now interacts with the cavity for the time $T_R$. At the end, the entangled state is created:
\end{enumerate}
\begin{align}
\ket{\Psi}_1= \frac{e^{-i\Phi_0}\ket{e, \alpha e^{-i\Phi_0}}+\ket{g,\alpha e^{i\Phi_0}}}{\sqrt{2}}
\end{align}
This is the typical situation of Scrödingers cat. Pushing to the extreme case $\Phi_0 = \frac{\pi}{2}$ we entangled the atom with the state $\ket{\pm i\alpha}$. As $\alpha$ is a complex number we entangled a single atom with a large 'cat' state. As the atom is detected it projects the full Schrödinger cat onto the dead or alive state. This projection can be avoided by adding a second Ramsey pulse, which mixes once again the states $\ket{e}$ and $\ket{g}$:
\begin{align}
\ket{e}\rightarrow\frac{\ket{e}+e^{i\varphi}\ket{g}}{\sqrt{2}}\\
\ket{g}\rightarrow\frac{\ket{g}-e^{-i\varphi}\ket{e}}{\sqrt{2}}
\end{align}
The field now becomes:
\begin{align}
\ket{\Psi_2} &= \frac{1}{2}\ket{e}\otimes\left[e^{-i\Phi_0}\ket{\alpha e^{-i\Phi_0}}-e^{-i\varphi}\ket{\alpha e^{i\Phi_0}}\right]+\frac{1}{2}\ket{g}\otimes\left[e^{i(\varphi-\Phi_0)}\ket{\alpha e^{-i\Phi_0}}+\ket{\alpha e^{i\Phi_0}}\right]
\end{align}
The final read-out is then given by:
\begin{align}
P_e &= \frac{1}{2}\left(1- e^{-n(1-\cos(2\Phi_0))}\cos(\varphi-\Phi_0-n\sin(2\Phi_0))\right)
\end{align}
So the presence of the cat leads to a phase shift and a decrease in fringe contrast. This was observed in the experiments \cite{Brune_1996a}, which are partially visualized in Fig. \ref{384136}.\selectlanguage{english}
\begin{figure}[h!]
\begin{center}
\includegraphics[width=0.70\columnwidth]{figures/CavityCat/CavityCat}
\caption{\selectlanguage{ngerman}{Observation of a Schrödinger cat state in cavity electrodynamics.
{\label{384136}}%
}}
\end{center}
\end{figure}\selectlanguage{ngerman}

Nowadays the entangled states have become an interesting platform  to create increasingly large Schrödinger cats. A common example is here the creation of a GHZ state:
\begin{align}
\ket{\psi} = \frac{\ket{0 \cdots 0}+ \ket{1\cdots 1}}{\sqrt{2}}
\end{align}

A interesting demonstration for up to 14 ions was performed in \cite{Monz_2011}. Importantly it also highlights the extremely fast decoherence of larger cat states. We will  go into more detail on how to create increasingly larger cats in the next lecture. However, I would like to finish the lecture with the discussion of quantum non-demolition measurements.

As of the time of writing cold atom systems systems
\cite{Omran_2019} cold the record of the largest cat with 20 atoms.

\section{Seeing a photon without destroying it}

The tool of the Rabi oscillations has been extend to observe photons without destroying them as detailed in great detail in the book by Raymond and Haroche \cite{quantum}. The underlying principle is the following:
\begin{itemize}
\item The atom is supposed to be in the ground state and the cavity is filled with one photon.
\item The interaction time is tuned such that the atom undergoes exactly one Rabi oscillation.
\item The initial and final state are therefore exactly the same, but the atom has picked up a phase $\pi$.
\end{itemize}
If the cavity was empty at the atom does not acquire a phase shift.

Finally, the phase is read out through a Ramsey sequence between the state $g$ and some unaffected independent state $i$ as visualized in Fig.\selectlanguage{english}
\begin{figure}[h!]
\begin{center}
\includegraphics[width=0.70\columnwidth]{figures/Bildschirmfoto-2018-12-18-um-15-19-24/Bildschirmfoto-2018-12-18-um-15-19-24}
\caption{{Projective measurment of a photon through with an atom.
{\label{338314}}%
}}
\end{center}
\end{figure}

This was implemented in Ref. \cite{Nogues_1999}. Based on this technique, the team later observed quantum jumps \cite{Gleyzes_2007} and even the stabilization of a Fock state through quantum feedback \cite{Sayrin_2011}.

\selectlanguage{english}
\FloatBarrier
\bibliographystyle{plainnat}
\bibliography{bibliography/converted_to_latex.bib%
}

\end{document}

