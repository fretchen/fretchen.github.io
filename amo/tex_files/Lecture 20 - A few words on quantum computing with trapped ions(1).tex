\documentclass[10pt]{article}

\usepackage{fullpage}
\usepackage{setspace}
\usepackage{parskip}
\usepackage{titlesec}
\usepackage[section]{placeins}
\usepackage{xcolor}
\usepackage{breakcites}
\usepackage{lineno}
\usepackage{hyphenat}





\PassOptionsToPackage{hyphens}{url}
\usepackage[colorlinks = true,
            linkcolor = blue,
            urlcolor  = blue,
            citecolor = blue,
            anchorcolor = blue]{hyperref}
\usepackage{etoolbox}
\makeatletter
\patchcmd\@combinedblfloats{\box\@outputbox}{\unvbox\@outputbox}{}{%
  \errmessage{\noexpand\@combinedblfloats could not be patched}%
}%
\makeatother


\usepackage[round]{natbib}
\let\cite\citep




\renewenvironment{abstract}
  {{\bfseries\noindent{\abstractname}\par\nobreak}\footnotesize}
  {\bigskip}

\titlespacing{\section}{0pt}{*3}{*1}
\titlespacing{\subsection}{0pt}{*2}{*0.5}
\titlespacing{\subsubsection}{0pt}{*1.5}{0pt}


\usepackage{authblk}


\usepackage{graphicx}
\usepackage[space]{grffile}
\usepackage{latexsym}
\usepackage{textcomp}
\usepackage{longtable}
\usepackage{tabulary}
\usepackage{booktabs,array,multirow}
\usepackage{amsfonts,amsmath,amssymb}
\providecommand\citet{\cite}
\providecommand\citep{\cite}
\providecommand\citealt{\cite}
% You can conditionalize code for latexml or normal latex using this.
\newif\iflatexml\latexmlfalse
\AtBeginDocument{\DeclareGraphicsExtensions{.pdf,.PDF,.eps,.EPS,.png,.PNG,.tif,.TIF,.jpg,.JPG,.jpeg,.JPEG}}

\usepackage[utf8]{inputenc}
\usepackage[ngerman,english]{babel}








\usepackage{siunitx}
\usepackage{amsmath}
\newcommand{\bra}[1]{\ensuremath{\left\langle#1\right|}}
\newcommand{\ket}[1]{\ensuremath{\left|#1\right\rangle}}
\newcommand{\braket}[1]{\ensuremath{\left\langle#1\right\rangle}}
\newcommand{\rhohat}{\hat{\rho}}
\newcommand{\tr}[1]{\mathrm{tr}(#1)}
\newcommand{\trarb}[2]{\mathrm{tr}_{#1}(#2)}
\newcommand{\vv}[1]{\mathbf{#1}}
\newcommand*\dif{\mathop{}\!\mathrm{d}}
\newcommand{\eexp}[1]{\mathrm{e}^{#1}}
\newcommand*\ch[1]{\ensuremath{\mathrm{#1}}}

\begin{document}

\title{Lecture 20 - A few words on quantum computing with trapped ions}



\author[1]{Fred Jendrzejewski}%
\affil[1]{Kirchhoff-Institut für Physik}%


\vspace{-1em}



  
  \date{January 07, 2025}


\begingroup
\let\center\flushleft
\let\endcenter\endflushleft
\maketitle
\endgroup





\selectlanguage{english}
\begin{abstract}
In this lecture we are going to discuss the fundamental ingredients for quantum computing with trapped ions. In a first step, we discuss trapping and cooling, then single qubit operations and finally two-qubit operations.%
\end{abstract}%



\sloppy


Quantum computation has become a branch of research at the interaction of physics, engineering, mathematices and computer science by now. The standard book on the topic is most likely the book by Nielsen and Chang \cite{Nielsen_2009}. However, an enormous amount of additional literature exists, I will only reference here to a nice introduction \cite{beginners} a more complete list is left for future discussions.


In this lecture we will discuss shortly the idea behind quantum computing and the discuss its implementation on trapped ions. While a large number of them exist, we decided to start with trapped ions for several very subjective reasons \footnote{Philipp Hauke worked a lot with them. Fred is an AMO person and Ferdinand Schmidt-Kaler was kind enough to provide a lot of background information on the experiments}.

And before we can start the discussion we would highly recommend the readers to take some time to go through the Nobel prize lecture of Dave Wineland as it gives a detailled discussion of the field from his point of view \cite{Wineland_2013}.

\section{What do we want from a QC ?}
In a QC we would like to implement algorithms, which are based on well defined operations. Influential examples of such algorithms are the quantum Fourier transform and the Grover algorithm \cite{beginners}. 

Given that computations are typically implemented through logical truth tables, we typically base a quantum computer on qubits. We then call one state $\ket{0}$ and on $\ket{1}$. Given that we would like to have reproducable computations, we always assume that we start them out with all qubits in the $\ket{0}$ state.

A computation consists then in applying a number of gates. The key is here that any algorithm might be built up from an extremely limited number of gates. Typically four are sufficient:
\begin{itemize}
\item The three gates that rotate each individual qubit on the Bloch sphere.
\item A gate that entangles them properly. The standard example is  here the CNOT gate, which we will come back too.
\end{itemize}

Such computations are then typically nicely visualized through circuit diagrams as used them already for the study of Bell inequalities and visualized in Fig. \ref{354313}\selectlanguage{english}
\begin{figure}[h!]
\begin{center}
\includegraphics[width=0.70\columnwidth]{figures/TwoQubitsCircuit/TwoQubitsCircuit}
\caption{{A simple circuit diagram. It show the initial state, an entanglement
gate, a number of single qubit gates and the final readout.
{\label{354313}}%
}}
\end{center}
\end{figure}

As atomic physics is only a minor part of the QC field, we typically have to learn the new notations of the field again. As such single qubit gates are typically not explained through the Pauli matrices but by different symbols like $H$ or $Z$. We come back to this later.

\subsection{Some of the hopes for QCs}
The major point to about a properly chosen set of gates is that it allows us to implement ANY algorithm. So they allow us to implement a \textit{universal} quantum computer. The main question is then how powerful such a QC would be. Could it solve problems as fast a a classical computer or maybe even faster ? This question is at the hard of the field of complexity classes, which studies which kind of problem can be solved how efficiently \cite{zoo}. 

The most fundamental question is then if a problem can be solved in a polynomial time (P hard) or not (NP-hard). Linear problems are P-hard and the travelling salesman problem is NP-hard. For some problems a quantum computer might then provide an answer in polynomial time, where a classical computer would not... The factorization of prime numbers is one of these problems as discussed in Shor algorithm. 

And the google paper that was published in 2019 actually indicated for the first time that a quantum computer achieved such a task \cite{Arute_2019}. 



\subsection{Requirements for a QC}
Given our excitement for a quantum computer, we might want a checklist of what we want from a quantum computer hardware.  DiVincenzo proposed the following ingredients \cite{HAFFNER_2008}:
\begin{enumerate}
\item Qubits that can store information in a scalable system.
\item The ability to initialize the system in the right state.
\item A universal set of gates.
\item Long coherence times, which are much longer than gate operation times.
\item Good measurement capabilities
\end{enumerate}

Trapped ions allow us to fulfill all these requirements as we will see in this lecture and we will go through them step-by-step. 

\section{Trapping and cooling}
For computing experiments one typically works with singe-charged ions like $^{40}Ca^+$. Given their charge, they can be trapped in very clean traps under vacuum. As such they are extremely well isolated from the environment and high precision experiments can be performed. Finally, they have only one remain electron in the outer shell. Therefore they have a hydrogenlike atomic structure.

However, the trap construction is not trivial given Maxwells equation $\text{div} \vec{E} = 0$. So, the experimentalists have to play some tricks with oscillating fields. We will not derive in detail how a resulting \textbf{Paul trap} works, but the  \href{https://youtu.be/Xb-zpM0UOzk}{linked video} gives a very nice impression of the idea behind it. 

This work on trapping ions dates back to the middle of the last century (!!!) and was recognized by the\href{https://www.nobelprize.org/prizes/physics/1989/summary/}{ Nobel prize in 1989} for Wolfgang Paul \cite{Paul_1990} and Hans Dehmelt \cite{Dehmelt_1990}. They shared the prize with Norman Ramsey, who developped extremely precise spectroscopic methods, now known as Ramsey spectroscopy \cite{Ramsey_1990}.\selectlanguage{english}
\begin{figure}[h!]
\begin{center}
\includegraphics[width=0.70\columnwidth]{figures/354px-Paul-Trap/354px-Paul-Trap}
\caption{{The two phases of the oscillating electric field of a Paul trap. Taken
from~\href{https://en.wikipedia.org/wiki/Quadrupole_ion_trap}{wikipedia}.
{\label{692754}}%
}}
\end{center}
\end{figure}\selectlanguage{english}
\begin{figure}[h!]
\begin{center}
\includegraphics[width=0.70\columnwidth]{figures/trap3/trap3}
\caption{{A linear ion (Paul) trap containing six calcium 40 ions. Taken
from~\href{https://quantumoptics.at/en/research/lintrap.html}{here} .
{\label{570611}}%
}}
\end{center}
\end{figure}

 A Paul trap provides a harmonic oscillator confinement with trapping frequencies in the order of hundreds of kHz. An ion trapped in such a trap can the be described by the Hamiltonian:
\begin{align}\label{Eq:HamHO}
\hat{H}_{t} &= \frac{\hat{p}^2}{2m}+ \frac{m\omega_t^2}{2}\hat{x}^2
\end{align}

The two variables $p$ and $x$ are non-commuting $[x, p] = i\hbar$, so they cannot be measured at the same time. It can be nicely diagonalized in terms of the ladder operators (see for more details see Section 1 of \cite{Jendrzejewski}):
\begin{align}
\hat{x} &= \sqrt{\frac{\hbar}{2m\omega_t}}\left(\hat{a}+\hat{a}^\dag\right)\\
\hat{p} &= i\sqrt{\frac{\hbar}{2m\omega_t}}\left(\hat{a}^\dag-\hat{a}\right)\\
\end{align}
So the Hamiltonian can now be written as:
\begin{align}
\hat{H} &= \hbar \omega_t \left(\hat{N} + \frac{1}{2}\right)\text{ with } \hat{N} = a^\dag a
\end{align}
Having loaded the ions into the Paul trap we also need to cool them down.

\section{Atom-light interaction}
Given that the ions keep only on atom on the outer shell, they have a hydrogenlike structure \cite{Jendrzejewskia, Jendrzejewskib}, which makes them optically well controllable. To control the ions further we use light of amplitude $E_0$ and frequency $\omega_L$:
\begin{align}
\vec{E}(t) &= \vec{E}_0 \cos(kx - \omega_L t+\varphi)\\
&= \frac{\vec{E}_0}{2} \left(e^{i[kx - \omega_lt+\varphi]}+e^{-i[kx-\omega_lt+\varphi]}\right) 
\end{align}
We will describe the interal states of the ion for the moment with the simple two state system of ground state $\ket{g}$ and excited state $\ket{e}$ at an energy $\hbar \omega_0$, which is typically in the order of thousands of THz. It has the Hamiltonian:
\begin{align}
H_{ion} = \hbar \omega_0 \ket{e}\bra{e}
\end{align}
Putting this ion into propagating light will induce a coupling between these two internal states. As previously , we will describe the coupling in the semi-classical approximation through $H_\textrm{int} = -\hat{\vec{D}} \cdot \vec{E} $. However, in this context we will not ignore the propagating nature of the light field and keep its position dependence. This is necessary as we would like to understand how the light influences the movement of the atoms and not only the internal states. Putting them together we obtain:
\begin{align}
H_\textrm{int} &= \frac{\Omega}{2}\left([\ket{g}\bra{e}+\ket{e}\bra{g}]e^{i(k \hat{x} - \omega_L t+\varphi)} + h.c.\right)
\end{align}
The laser frequency is tuned closely to the frequency of the internal state transition and we will be only interested in the detuning $\Delta = \omega_0 - \omega_L$. Importantly, it couples  the position of the atom and the internal states.

To simplify the problem, we can work in the rotating frame to describe the external and internal degrees of freedom for the ion \cite{Jendrzejewskic}:
\begin{align}\label{Eq:DressedAtomLightInteraction}
\hat{H}= \hbar \omega_t \hat{a}^\dag \hat{a} + \hbar\Delta \ket{e}\bra{e} + \frac{\Omega}{2}\left(\ket{e}\bra{g}e^{i\left(k \hat{x}+\varphi\right)} + h.c.\right)
\end{align}

We will now see how this system is used to cool the ions to the motional groundstate, perform single qubit operations and then two-qubit operations.


\section{Doppler cooling}
This interaction of the atom with a photon is at the origin of the all-important Laser cooling, which was pioneered for ions in the 1970s (!!) by the Wineland group. For cooling transition we couple the ground state to an excited state of finitie lifetime $\tau= \frac{1}{\Gamma}$.

The basic idea is visualized in Fig. \ref{960763} and more details can be found in Sec. IV.A of \cite{Leibfried_2003}.\selectlanguage{english}
\begin{figure}[h!]
\begin{center}
\includegraphics[width=0.70\columnwidth]{figures/LaserCooling/LaserCooling-01}
\caption{{The basic idea of laser cooling. The incoming light gives the ion a
momentum kick~\(\vec{k}_{in}\). The photon is reemitted in a random
direction such that~\(<\vec{k}_{out}>=0\).
{\label{960763}}%
}}
\end{center}
\end{figure}

This laser cooling had a tremendous impact on the field of atomic physics in general. Notably it gave rise to the field of cold atoms to which we will get back in the next lecture. This importance was recognized in the Nobel prizes of 1997 for Steve Chu \cite{Chu_1998}, Claude Cohen-Tannoudji \cite{Cohen_Tannoudji_1998} and Bill Phillips \cite{Phillips_1998}. 





\subsection{Working in the Lamb-Dicke regime}
After this initial cooling stage the atoms have to be cooled to the ground state in the trap. To treat the trapped particles we will express the position operator in terms of the ladder operator, such that:
\begin{align}
k\hat{x} &= \eta (\hat{a}^\dag+ \hat{a})\\
\eta &= \sqrt{\frac{\hbar^2 k^2/2m}{\hbar \omega_t}} =\sqrt{\frac{E_R}{\hbar \omega_t}}
\end{align}
$\eta$ is called the \textit{Lamb-Dicke} parameter. It compares the change in motional energy due to the absorption of the photon $E_r = \frac{(\hbar k)^2}{2m}$ compared to the energy spacing $\hbar \omega_t$ in the trap. When it is small it suppresses the change of the motional state of the atom due to the absorption of a photon. 

For simplicity we will set in this section $\varphi=0$ and develop the exponent to obtain:

\begin{align}
H_\textrm{int} &= \frac{\Omega}{2}\left(\ket{e}\bra{g}\left(1 + i\eta[\hat{a}^\dag+ \hat{a}]\right) + h.c.\right)
\end{align}
So it contains three couplings for different trap levels and internal states:
\begin{itemize}
\item The \textit{carrier} transition $\ket{g,n}\rightarrow \ket{e,n}$ with strength $\Omega$.
\item The \textit{red} sideband $\ket{g,n}\rightarrow \ket{e,n-1}$ with strength $\eta \Omega(n+1)$. It leads to a reduction of the trap level and it is resonant for $\Delta = -\omega_t$.
\item The \textit{blue} sideband $\ket{g,n}\rightarrow \ket{e,n+1}$ with strength $\eta \Omega n$. It leads to an increase of the trap level and it is resonant for $\Delta = \omega_t$.
\end{itemize}
The full energy diagram is summarized in Fig. \ref{226851}.\selectlanguage{english}
\begin{figure}[h!]
\begin{center}
\includegraphics[width=0.70\columnwidth]{figures/LevelStructure/LevelStructure-01}
\caption{{Level structure of an two-level system coupled to a laser field as
discussed in the text.
{\label{226851}}%
}}
\end{center}
\end{figure}

This scheme is used to perform \textbf{Raman side-band cooling}. The laser is tuned on the transition $\ket{n,g}\rightarrow \ket{n-1,e}$ such that each absorption involves a reduction in the trap level. This set-up for cooling was first demonstrated in 1995 by the Wineland group \cite{Monroe_1995}.

It is at this stage that the ions are in the motional ground state and we can focus our attention to the high control of the internal qubit states of the ion for quantum computing.

\section{Single-qubit operations}

The single qubit operations can now be identified with the transition $\ket{e,n}\leftrightarrow \ket{g,n}$. We can then simplify Eq. \eqref{Eq:DressedAtomLightInteraction} too:
\begin{align}\label{Eq:SingleQubitOperation}
\hat{H}= \hbar\Delta \ket{e}\bra{e} + \frac{\hbar\Omega}{2}\left(\ket{e}\bra{g}e^{i\varphi} +\ket{g}\bra{e}e^{-i\varphi}\right)
\end{align}
We can translate this into the language of qubit operations through the definitions:
\begin{align}
\sigma_z &= \frac{\ket{e}\bra{e}-\ket{g}\bra{g}}{2}\\
\sigma_x &= \frac{\ket{e}\bra{g}+\ket{g}\bra{e}}{2}\\
\sigma_y &= \frac{i\ket{e}\bra{g}-i\ket{g}\bra{e}}{2}
\end{align}
So we can now simply write the Hamiltonian as \cite{Jendrzejewskid}:
\begin{align}\label{Eq:SingleQubitOperation}
\hat{H}= \hbar\Delta \sigma_z +\Omega_x \sigma_x +\Omega_y \sigma_\\
\Omega_x = \Omega \cos(\varphi)\\
\Omega_y = \Omega \sin(\varphi)
\end{align}

In the QC community people rarely talk about the Pauli matrices, but much rather about a few specific gates. The most cited here is the \textit{Hadamard} gate, which transforms $\ket{0/1}\rightarrow \frac{\ket{0}\pm\ket{1}}{\sqrt{2}}$. So it has no good classical analog. Further a double application brings us back to the origin. 

The other gate we named about was a Z gate, which is simply a $\pi$ rotation around the z axis.

\section{Two-qubit operations}

To implement a quantum computer the system has to be completed by a two-qubit operation. For ions a number of two-qubit gates exist as discussed nicely in Sec. 2.6 of  \cite{HAFFNER_2008}:
\begin{itemize}
\item The \textbf{Cirac-Zoller} gate was the first proposed two-qubit gate \cite{Cirac_1995} and it was also the first one realized within the same year \cite{Monroe_1995a}.
\item The \textbf{Soerensen-Moelmer} gate was proposed later \cite{S_rensen_1999}, but it is extremely important from a practical point of view as it leads to very high entanglement fidelities. 
\item Another realization, which we mention for completeness is the geometric phase-gate, which is used in the NIST group \cite{Leibfried_2003a}.
\end{itemize}

We will now discuss a bit the Soerensen-Moelmer gate, which is nicely described in Ref. \cite{S_rensen_2000}. In this set-up two ions sit in a common trap. The cost of energy for exciting one of the ions will be labelled $\omega_t$ as in the first section. So we assume that the scheme starts in the state $\ket{ggn}$, where both atoms are in the internal ground-state $g$ and in some excited trap level $n$.

In the next step, these two ions experience two lasers, which are coupling excited and the ground state of the ions:
\begin{itemize}
\item One laser has frequency $\omega_1=\omega_0-\omega_t+\delta$ and Rabi coupling strength $\Omega$. It is therefore only slightly detuned from the transitions $|ggn\rangle\rightarrow|eg,n-1\rangle |ge,n-1\rangle$.
\item The second laser has frequency $\omega_2=\omega_0+\omega_t-\delta$ and Rabi coupling strength $\Omega$. It is therefore only slightly detuned from the transitions $|ggn\rangle\rightarrow|eg,n+1\rangle |ge,n+1\rangle$.
\end{itemize}
The resulting level diagram is depicted in Fig. \ref{658942}.\selectlanguage{english}
\begin{figure}[h!]
\begin{center}
\includegraphics[width=0.70\columnwidth]{figures/SM-LevelDiagramv1/SM-LevelDiagramv1}
\caption{{Level scheme of the Sorensen Moelmer gate as described in the text.
{\label{658942}}%
}}
\end{center}
\end{figure}

The gate is then operated in the regime of small coupling strength $\eta \Omega n \ll \delta$. In this case coupling to the excited motional states $n\pm 1$ is suppressed by a factor of $\frac{\eta \Omega n}{\delta}$. On the other hand we are exactly on resonance for the two-photon transitions $|ggn\rangle\rightarrow|eg,n+1\rangle\rightarrow|ee,n\rangle$ etc. So we can do second-order pertubation theory (Sec 1 of \cite{Jendrzejewskie})or adiabatic elimination (see Sec. 2.1 of \cite{Jendrzejewskic}) to obtain the effective Hamiltonian:
\begin{align}
H_\mathrm{SM} &= \frac{\Omega_\mathrm{SL}}{2}\left(\ket{ggn}\bra{een} + (\ket{een}\bra{ggn}\right)\text{ with }\Omega_{SL} = -\frac{(\Omega \eta)^2}{2(\eta - \delta)}
\end{align}
So starting out with the state $\ket{gg}$ and applying the laser for $t\Omega =\frac{\pi}{2}$, we obtain the entangled state that we are looking for. 

The operation of the gate was first demonstrated in 2000 by the Wineland group and allowed at the time for generating a Bell state with a fidelity of 83\% \cite{Sackett_2000}. This limit has been increasingly pushed of the years and now reaches the 99.9\% region \cite{Benhelm_2008,Gaebler_2016,Ballance_2016}. 

Such a fidelity sounds very impressive on first sight and it is by now the result of several decades of work. However, in a quantum computer we would like to chain a large number of these gates behind each other.
\begin{itemize}
\item After 10 iterations the fidelity dropped to 99\%.
\item After 100 iterations the fidelity dropped to 90\%.
\item After 1000 iterations the fidelity dropped to 30\%.
\end{itemize}
So even with such an excellent fidelity it will barely be possible to chain much more than 100 gates before the some extremely iffy things start to happen.\selectlanguage{english}
\begin{figure}[h!]
\begin{center}
\includegraphics[width=0.70\columnwidth]{figures/Bildschirmfoto-2019-05-13-um-06-03-32/Bildschirmfoto-2019-05-13-um-06-03-32}
\caption{{From the iSWAP to the CNOT gate.
{\label{483914}}%
}}
\end{center}
\end{figure}

So we have experimentally the choice of entanglement tool in the way that is most adapted to our work.



\section{Practical considerations}

A commonly used ion is $\ch{Ca+} \, (Z=20)$. Per shell we get:

\begin{tabular}{c|cccc}
n & 1 & 2 & 3 & 4\\
\hline \\
$N_{e}$ & 2 & 8 & 8 & 1
\end{tabular}

The level scheme of the calcium atom is shown in \ref{652265}. The different transitions are used for different purposes:
\begin{itemize}
\item The broad transition at 397nm is used for cooling.
\item Coupling between the qubit states is performed through the 729nm transition.
\item The 866nm and the 854nm are used for pumping the atoms into appropiate substates.
\end{itemize}\selectlanguage{english}
\begin{figure}[h!]
\begin{center}
\includegraphics[width=0.70\columnwidth]{figures/exploringthequantum/Bildschirmfoto-2019-04-23-um-16.33.03}
\caption{{The level scheme of the calcium atom. The arrows indicate transitions by
absorption and emission of photons. A qubit can be realized by choosing
the ground state and an excited state. Taken from~\protect\cite{Schindler_2013}.
{\label{652265}}%
}}
\end{center}
\end{figure}

Several solutions for scaling up the quantum computing architecture are
under way .~~\emph{Long ion chains} in linear Paul traps as shown
in~{\ref{570611}} with up to 40 ions are the current
`work-horse'. This is the `simplest' existing architecture. In such a
geometry Shors algorithm was shown for the number 15~\cite{Monz_2016}
with five qubits and entanglement between up to 14 qubits was
studied~\cite{Monz_2011}. However, it reaches its natural limits for
entangling distant ions due to cross-talk with other ions during the
operation. Therefore, different approaches are currently tested to scale
the architecture to larger fault-tolerant geometries \cite{Bermudez_2017}.

By no means we will be able to give a full picture of the booming field.
However, a few main players are:

\begin{itemize}
\tightlist
\item
  NIST, JQI and IonQ, which are all strongly connected through their
  shared past with Dave Wineland.
\item
  Innsbruck,~ Mainz and AQT which are connected through their shared
  past and present with Rainer Blatt.
\item
  ETH, Oxford, \ldots{}
\item
  The AQTION and the MicroQC network, which are part of the European
  flagship initiative.
\end{itemize}

\selectlanguage{english}
\FloatBarrier
\bibliographystyle{plainnat}
\bibliography{bibliography/converted_to_latex.bib%
}

\end{document}

