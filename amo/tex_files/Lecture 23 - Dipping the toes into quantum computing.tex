\documentclass[10pt]{article}

\usepackage{fullpage}
\usepackage{setspace}
\usepackage{parskip}
\usepackage{titlesec}
\usepackage[section]{placeins}
\usepackage{xcolor}
\usepackage{breakcites}
\usepackage{lineno}
\usepackage{hyphenat}





\PassOptionsToPackage{hyphens}{url}
\usepackage[colorlinks = true,
            linkcolor = blue,
            urlcolor  = blue,
            citecolor = blue,
            anchorcolor = blue]{hyperref}
\usepackage{etoolbox}
\makeatletter
\patchcmd\@combinedblfloats{\box\@outputbox}{\unvbox\@outputbox}{}{%
  \errmessage{\noexpand\@combinedblfloats could not be patched}%
}%
\makeatother


\usepackage[round]{natbib}
\let\cite\citep




\renewenvironment{abstract}
  {{\bfseries\noindent{\abstractname}\par\nobreak}\footnotesize}
  {\bigskip}

\titlespacing{\section}{0pt}{*3}{*1}
\titlespacing{\subsection}{0pt}{*2}{*0.5}
\titlespacing{\subsubsection}{0pt}{*1.5}{0pt}


\usepackage{authblk}


\usepackage{graphicx}
\usepackage[space]{grffile}
\usepackage{latexsym}
\usepackage{textcomp}
\usepackage{longtable}
\usepackage{tabulary}
\usepackage{booktabs,array,multirow}
\usepackage{amsfonts,amsmath,amssymb}
\providecommand\citet{\cite}
\providecommand\citep{\cite}
\providecommand\citealt{\cite}
% You can conditionalize code for latexml or normal latex using this.
\newif\iflatexml\latexmlfalse
\AtBeginDocument{\DeclareGraphicsExtensions{.pdf,.PDF,.eps,.EPS,.png,.PNG,.tif,.TIF,.jpg,.JPG,.jpeg,.JPEG}}

\usepackage[utf8]{inputenc}
\usepackage[ngerman,english]{babel}








\usepackage{amsmath}
\newcommand{\bra}[1]{\ensuremath{\left\langle#1\right|}}
\newcommand{\ket}[1]{\ensuremath{\left|#1\right\rangle}}
\newcommand{\braket}[1]{\ensuremath{\left\langle#1\right\rangle}}
\newcommand{\rhohat}{\hat{\rho}}
\newcommand{\tr}[1]{\mathrm{tr}(#1)}
\newcommand{\trarb}[2]{\mathrm{tr}_{#1}(#2)}
\newcommand{\vv}[1]{\mathbf{#1}}
\newcommand*\dif{\mathop{}\!\mathrm{d}}
\newcommand{\eexp}[1]{\mathrm{e}^{#1}}

\begin{document}

\title{Lecture 23  - Dipping the toes into quantum computing}



\author[1]{Fred Jendrzejewski}%
\author[2]{Selim Jochim}%
\author[2]{Matthias Weidemüller}%
\affil[1]{Kirchhoff-Institut für Physik}%
\affil[2]{Physikalisches Institut der Universität Heidelberg}%


\vspace{-1em}



  
  \date{January 07, 2025}


\begingroup
\let\center\flushleft
\let\endcenter\endflushleft
\maketitle
\endgroup





\selectlanguage{english}
\begin{abstract}
We will finish our discussion by the some words on quantum simulation and quantum computation.%
\end{abstract}%



\sloppy


In the last lecture  we spoke about the properties of quantum degenerate gases, what makes the difference between Bosons and Fermions and how they are employed for the precise study of simple condensed-matter problems. We will see how this approach generalizes step-by-step towards the idea of quantum computation.

\section{Lattice problems}
Amongst other things we discussed how optical lattices enable the study of the Hubbard model:
\begin{align}
\hat{H}_{H}&= -J \sum_{<i,j>} (\hat{a}^\dag_i \hat{a}_j +\hat{a}^\dag_j \hat{a}_i )+U\sum_i\hat{n}_i(\hat{n}_i-1)-\mu \sum_i \hat{n}_i
\end{align}
The three terms are associated with the following processes:
\begin{itemize}
\item The first term describes the hopping of the particles from one site to a neighboring site.
\item The second term describes the on-site interactions between the particles.
\item The third term fixes the average number of atoms per site in the grand-canonical ensemble.
\end{itemize}

Within this model it is possible to study with ultracold atoms problems, which are beyond the reach of theoretical methods. One of the most precise measurements of the different quantum states that emerges in these models can be performed in experiments with quantum gas microscopes.\selectlanguage{english}
\begin{figure}[h!]
\begin{center}
\includegraphics[width=0.98\columnwidth]{figures/QuantumGasMicroscope/QuantumGasMicroscope}
\caption{{Quantum gas microscopes and the observation of the Mott insulator shell
structure as presented in Ref. \protect\cite{Sherson_2010}
{\label{783515}}%
}}
\end{center}
\end{figure}

In these experiments the experimentalist hard-wires the model of interest and then studies its properties as a function of time or temperture. This approach, is nowadays called quantum simulation or analog quantum computation\footnote{The difference in language is mostly a matter of community and journals the papers are submitted too}.

\section{Spin systems}
These moving (itinerant) particles are by no means the only interesting problems from condensed matter. Another, widely studied problem are spin systems. The standard example being maybe the Ising model. For these kinds of spin systems, only two states are allowed, \textbf{up} or \textbf{down}, i.e. $\pm 1$. 

\subsection{Classical spins}
The classical Hamiltonian of the Ising model reads then :
\begin{align}\label{Eq:ClassicalIsing}
H_I = J \sum_{<i,j>} s_i s_{i+1}+h \sum_i s_i \text{ with }s_i = \pm 1
\end{align}
The two terms describe the interaction between spins and an external magnetic field:
\begin{itemize}
\item The first factor describes the interaction between neighboring spins and it can only take two values $+1$, if the spins are aligned and $-1$ if the spins are anti-aligned.
\item The second factor describes the external magnetic field, aligning the spins in the direction of the magnetic field.
\end{itemize}
From the two ingredients it is possible to write down the states of minimal energies. If no external magnetic field is applied, only the first factor counts.
\begin{itemize}
 \item If $J$ is positive, the spins should be anti-aligned for minimal energy. The ground-state is an \textbf{anti-ferromagnet}.
 \item If $J$ is negative, the spins should be aligned for minimal energy. The ground-state is an \textbf{ferromagnet}.
\end{itemize}
The external magnetic field is always attempting to align the spins, so if the $J$ is positive, there will be a point at which the external magnetic field breaks the state and the ferromagnetic phase becomes the new ground state.


\subsection{Quantum spins}

The same model for quantum spins differs fundamentally from its classical counter-part. The most direct connection to the classical Ising model is done by assuming that each spin has length 1/2. As we detect it, we know the we can only observe \textbf{up} $\uparrow$ and \textbf{down} $\downarrow$, exactly it should be. However, we now at least since the third lecture \cite{Jendrzejewski}, that the quantum spin does not only have the these two states, but it also evolves on any superposition on the Bloch sphere. The quantum version of \eqref{Eq:ClassicalIsing} reads therefore:
\begin{align}\label{Eq:ClassicalIsing}
H_{I} = J \sum_{<i,j>} \hat{s}_{i,z} \hat{s}_{i+1, z}+h \sum_i \hat{s}_{i,z} \text{ with }s_i = \pm 1
\end{align}
The index could also be $x,y,z$, the important thing is that both terms have the same index to make the analogy to the classical system.

To make clear, where the biggest differences between the classical model and the quantum version arise, it is actually possible to change the magnetic field term slightly and the make it point into the transverse direction:

\begin{align}\label{Eq:ClassicalIsing}
H_{TI} = J \sum_{<i,j>} \hat{s}_{i,z} \hat{s}_{i+1, z}+h \sum_i \hat{s}_{i,x} \text{ with }s_i = \pm 1
\end{align}

Such a mgnetic field does not make any sense for classical Ising spin as transverse magnetization is not allowed in the system. In the quantum version however, the small change in index can lead to very interesting physics. If there is no interaction term $J=0$, the system is described by plenty of two-state systems with eigenstates pointing into the $\ket{\pm} = \frac{\ket{\uparrow}\pm\ket{\downarrow}}{\sqrt{2}}$. If the transverse magnetic field is on the other hand zero, the system can be fully diagonalized in the $z$ basis and the basis states are $\ket{\uparrow}$ or $\ket{\downarrow}$. Most recently, artifical spin chains with up to 50 spins have been realized with ultracold ions and ultracold atoms as shown in Fig. \ref{920878}.\selectlanguage{english}
\begin{figure}[h!]
\begin{center}
\includegraphics[width=0.70\columnwidth]{figures/SpinChains/SpinChains}
\caption{{Artifical Quantum Ising models with Ions and neutral atoms. The system
is initialized in a well understood groundstate and then left to evolve
into some complicated superposition states. Pics are taken from
\protect\cite{Islam_2013}\protect\cite{Zhang_2017}\protect\cite{Bernien_2017}
{\label{920878}}%
}}
\end{center}
\end{figure}



This approach of analog quantum simulation allows us to simulate the dynamics of some specific model in regions, which are beyond the access for theory.  The complexity of the problem can be seen from the time evolution of the initial state is connected to the initial state via:
\begin{align}
\ket{\psi(t)} &= \hat{U}\ket{0}\\
\hat{U} &= e^{-i\hat{H}t}
\end{align}
Finding the unitary operator from the Hamiltonian is not simple as:
\begin{align}
e^{-i(\hat{H}_2 +\hat{H}_1)t}\neq e^{-i\hat{H}_2 t}e^{-i\hat{H}_1 t}
\end{align}
We would actually need all the commutators, which can become a impossible task for general models.



\section{Digital quantum simulation}

The Ising model was not just implemented in the analog approach presented above. Another  powerful approach is the digital approach to quantum simulation \cite{Lanyon_2011}.   The idea behind this problem is the decomposition of the complex Hamiltonian into a few fundamental ingredients, i.e. two-qubit vs one qubit gates. We then decompose the evolution in small time-steps, which allows us to write:
\begin{align}
e^{-i(\hat{H}_2 +\hat{H}_1)t}&=\lim_{n\rightarrow \infty}\left(1-\frac{i(\hat{H}_2 +\hat{H}_1)t}{n}\right)^n\\
&=\lim_{n\rightarrow \infty}\left(\left[1-\frac{i\hat{H}_2 t}{n}\right]\left[1-\frac{i\hat{H}_1t}{n}\right]\right)^n\\
&=\lim_{n\rightarrow \infty}\left(e^{-\frac{i\hat{H}_2 t}{n}}e^{-\frac{i\hat{H}_1t}{n}}\right)^n\\
\end{align}
So we can simulate the time evolution of the full Hamiltonian through a series of elementary gates, which are applied for a short amount of time. This approach is called trotterization and its results are visualized in Fig. \ref{866769}.\selectlanguage{english}
\begin{figure}[h!]
\begin{center}
\includegraphics[width=0.70\columnwidth]{figures/TrotterizationOfIsing/TrotterizationOfIsing}
\caption{{Trotterization of the Ising model as shown in Ref. \protect\cite{Lanyon_2011}
{\label{866769}}%
}}
\end{center}
\end{figure}

The idea of decomposing the Hamiltonian into a few general gates is actually enormously powerful as the first different ingredients provide all the necessary control to decompose basically any Hamiltonian into its elementary gates. The idea is the following:
\begin{itemize}
\item The Hamiltonian $H_1$ might be any single particle Hamiltonian, which rotates the single spin on its particular Bloch sphere.
\item The Hamiltonian  $H_2$ is describing the interaction between  two spins, so it creates entanglement, which can be created between the spins. Applying for example $\hat{\sigma}_x\hat{\sigma}_x$ on the state $\ket{\downarrow\downarrow}$ allows for the engeration of a Bell state.
\end{itemize}
The power of this idea was examplified in works with ultracold ions on high-energy problems \cite{Martinez_2016}. Here, the original Hamiltonian from high-energy physics was translated into the language of digital quantum simulation, such that it could be studied with ions.

\section{Quantum computation}
The power of the few gates is not just helpful for quantum simulation, but also quantum computation.  Only physicists would identify the previous systems with spins 1/2 systems. Most mathematician or computer scientists would most likely see some series of bits in the previous formulation. Let us move towards this language. In any modern digital computer the system consists of bits, which might be 0 or 1. We can then initialize the system and tell him in the next step to flip or not in some complicated fashion. We would say that we apply a gate onto the bit chain. 

We call the computer universal if it is able to transfer any bit sequence into any other bit sequence with the some  combination of gates.  For the simplest case we might just consider a system of two bits, which can then have the states ${00, 10, 01, 11}$. Any computation means that we transfer some initial bit configuration into another one, by some series of gates. Actually, we only need four gates for a universal quantum computer:
\begin{itemize}
\item Three gates to move the particles completely around on the Bloch sphere.
\item And a two-qubit gate, which entangles the different qubits.
\end{itemize}

The language of these kinds of algorithms can quite rapidly diverge substantially from the one that physicists use . Actually, the field of quantum computation is part of quantum information science. An overview over the most commonly used gates in quantum information science is given in Fig.\selectlanguage{english}
\begin{figure}[h!]
\begin{center}
\includegraphics[width=0.70\columnwidth]{figures/Bildschirmfoto-2019-01-21-um-10-22-40/Bildschirmfoto-2019-01-21-um-10-22-40}
\caption{{Some common gate of digital quantum computing as given in
\protect\cite{beginners}.
{\label{307961}}%
}}
\end{center}
\end{figure}

 The most common reference on the topic is \cite{Nielsen_2009} and they visualized the divergence quite nicely as shown in Fig. \ref{679777}.\selectlanguage{english}
\begin{figure}[h!]
\begin{center}
\includegraphics[width=0.70\columnwidth]{figures/Bildschirmfoto-2019-01-21-um-09-54-07/Bildschirmfoto-2019-01-21-um-09-54-07}
\caption{{The field of quantum computation as shown by \protect\cite{Nielsen_2009}
{\label{679777}}%
}}
\end{center}
\end{figure}




\subsection{Quantum algorithms}
The idea that quantum computation might have fundamental advantages over classical computation has driven the field since the 1990s. The different approaches can be read up in the numerous books and literature on the topic \cite{beginners}.


\section{A few words about commercial solutions}
In the last years the field of quantum computation has seen dramatic changes by the arrival of commercial players in the field. Since a few years, IBM. Google, Microsoft or IonQ have started to develop commerical quantum computers. This implies not only that they invested substantial amounts of money in an improved hardware. Even more curiously, it is now possible to test simple algorithms on the cloud \cite{ai}. They are for the moment not necessarily of great computational power, but enormous fun to play with.

\selectlanguage{english}
\FloatBarrier
\bibliographystyle{plainnat}
\bibliography{bibliography/converted_to_latex.bib%
}

\end{document}

