\documentclass[10pt]{article}

\usepackage{fullpage}
\usepackage{setspace}
\usepackage{parskip}
\usepackage{titlesec}
\usepackage[section]{placeins}
\usepackage{xcolor}
\usepackage{breakcites}
\usepackage{lineno}
\usepackage{hyphenat}





\PassOptionsToPackage{hyphens}{url}
\usepackage[colorlinks = true,
            linkcolor = blue,
            urlcolor  = blue,
            citecolor = blue,
            anchorcolor = blue]{hyperref}
\usepackage{etoolbox}
\makeatletter
\patchcmd\@combinedblfloats{\box\@outputbox}{\unvbox\@outputbox}{}{%
  \errmessage{\noexpand\@combinedblfloats could not be patched}%
}%
\makeatother


\usepackage[round]{natbib}
\let\cite\citep




\renewenvironment{abstract}
  {{\bfseries\noindent{\abstractname}\par\nobreak}\footnotesize}
  {\bigskip}

\titlespacing{\section}{0pt}{*3}{*1}
\titlespacing{\subsection}{0pt}{*2}{*0.5}
\titlespacing{\subsubsection}{0pt}{*1.5}{0pt}


\usepackage{authblk}


\usepackage{graphicx}
\usepackage[space]{grffile}
\usepackage{latexsym}
\usepackage{textcomp}
\usepackage{longtable}
\usepackage{tabulary}
\usepackage{booktabs,array,multirow}
\usepackage{amsfonts,amsmath,amssymb}
\providecommand\citet{\cite}
\providecommand\citep{\cite}
\providecommand\citealt{\cite}
% You can conditionalize code for latexml or normal latex using this.
\newif\iflatexml\latexmlfalse
\AtBeginDocument{\DeclareGraphicsExtensions{.pdf,.PDF,.eps,.EPS,.png,.PNG,.tif,.TIF,.jpg,.JPG,.jpeg,.JPEG}}

\usepackage[utf8]{inputenc}
\usepackage[ngerman,english]{babel}








\usepackage{amsmath}
\newcommand{\bra}[1]{\ensuremath{\left\langle#1\right|}}
\newcommand{\ket}[1]{\ensuremath{\left|#1\right\rangle}}
\newcommand{\braket}[1]{\ensuremath{\left\langle#1\right\rangle}}
\newcommand{\rhohat}{\hat{\rho}}
\newcommand{\tr}[1]{\mathrm{tr}(#1)}
\newcommand{\trarb}[2]{\mathrm{tr}_{#1}(#2)}
\newcommand{\vv}[1]{\mathbf{#1}}
\newcommand*\dif{\mathop{}\!\mathrm{d}}
\newcommand{\eexp}[1]{\mathrm{e}^{#1}}

\begin{document}

\title{Lecture 3 - The two-level system}



\author[1]{Fred Jendrzejewski}%
\author[2]{Selim Jochim}%
\affil[1]{Kirchhoff-Institut für Physik}%
\affil[2]{Physikalisches Institut der Universität Heidelberg}%


\vspace{-1em}



  
  \date{January 04, 2025}


\begingroup
\let\center\flushleft
\let\endcenter\endflushleft
\maketitle
\endgroup





\selectlanguage{english}
\begin{abstract}
We are going to discuss the two-level system, it's static properties like level splitting at avoided crossings and dynamical properties like Rabi oscillations.%
\end{abstract}%



\sloppy


After the previous discussions of some basic cooking recipes to quantum mechanics in last weeks lectures \cite{Jendrzejewskia} and \cite{Jendrzejewski}, we will use them to understand the two-level system. A very detailled discussion can be found in chapter 4 of Ref. \cite{1}. The importance of the two-level system is at least three-fold:
\begin{enumerate}
\item It is the simplest system of quantum mechanics as it spans a Hilbert space of only two states.
\item It is quite ubiquitous in nature and very widely used in atomic physics.
\item The two-level system is another word for the qubit, which is the fundamental building block of the exploding field of quantum computing and quantum information science.
\end{enumerate}\selectlanguage{english}
\begin{figure}[h!]
\begin{center}
\includegraphics[width=0.70\columnwidth]{figures/Bildschirmfoto-2018-09-28-um-14-39-281/Bildschirmfoto-2018-09-28-um-14-39-281}
\caption{{Examples for two-state systems. a) Benzene: In the ground state, the
electrons are delocalized. b) Ammonia: The nitrogen atom is either found
above or below the hydrogen triangle. The state changes when the
nitrogen atom tunnels. c) Molecular ion : The electron is either
localized near proton 1 or 2.
{\label{217687}}%
}}
\end{center}
\end{figure}

Some of the many examples for two-level systems that can be found in nature:
\begin{itemize}
\item Spin of the electron: Up vs. down state
\item Two-level atom with one electron (simplified): Excited vs. ground state
\item Structures of molecules, e.g., \hyperref[fig:twostate]{NH\textsubscript{3}}
\item Occupation of mesoscopic capacitors in nanodevices.
\item Current states in superconducting loops.
\item Nitrogen-vacancy centers in diamond.
\end{itemize}

\section{Hamiltonian, Eigenstates and Matrix Notation}

To start out, we will consider two eigenstates $\ket{0}$, $\ket{1}$ of the Hamiltonian $\hat{H}_0$ with
\begin{align}
 \hat{H}_0\ket{0}=E_0\ket{0}, \qquad \hat{H}_0\ket{1}=E_1\ket{1}.
\end{align}
Quite typically we might think of it as a two-level atom with states 1 and 2. The eigenstates can be expressed in matrix notation:
\begin{align}
 \ket{0}=\left( \begin{array}{c} 1 \\ 0 \end{array} \right), \qquad \ket{1}=\left( \begin{array}{c} 0 \\ 1 \end{array} \right),
\end{align}
so that $\hat{H}_0$ be written as a diagonal matrix
\begin{align}
    \hat{H}_0 = \left(\begin{array}{cc} E_0 & 0 \\ 0 & E_1 \end{array}\right).
\end{align}
If we would only prepare eigenstates the system would be rather boring. However, we typically have the ability to change the Hamiltonian by switching on and off laser or microwave fields \footnote{See the discussions of the next lecture}. We can then write the Hamiltonian in its most general form as:
\begin{align}\label{Eq:TwoLevelGeneral}
\hat{H} = \frac{\hbar}{2}\left( \begin{array}{cc} \Delta  & \Omega_x - i\Omega_y\\ \Omega_x +i\Omega_y & -\Delta \end{array} \right)
\end{align}
Sometimes we will also chose the definition:
\begin{align}
\Omega = |\Omega| e^{i\varphi}=\Omega_x + i\Omega_y
\end{align}
It is particularly useful for the case in which the coupling is created by a laser. Another useful way of thinking about the two-level system is as a spin in a magnetic field. Let us remind us of the definitions of the of the spin-1/2 matrices:
\begin{align}
s_x = \frac{\hbar}{2}\left(\begin{array}{cc}
0 & 1\\
1 &  0
\end{array}
\right)~
s_y = \frac{\hbar}{2}\left(\begin{array}{cc}
0 & -i\\
i &  0
\end{array}
\right)~s_z =\frac{\hbar}{2} \left(\begin{array}{cc}
1 & 0\\
0 &  -1
\end{array}
\right)
\end{align}
We then obtain:
\begin{align}\label{Eq:HamSpin}
\hat{H} = \mathbf{B}\cdot\hat{\mathbf{s}}\text{ with }\mathbf{B} = (\Omega_x, \Omega_y, \Delta)
\end{align}
You will go through this calculation in the excercise of this week.

\subsection{Case of no perturbation $\Omega = 0$}

This is exactly the case of no applied laser fields that we discussed previously. We simply removed the energy offset $E_m = \frac{E_0+E_1}{2}$ and pulled out the factor $\hbar$, such that $\Delta$ measures a frequency. So we have:
\begin{align}
E_0 = E_m+ \frac{\hbar}{2}\Delta\\
E_1 = E_m- \frac{\hbar}{2}\Delta
\end{align}
We typically call $\Delta$ the energy difference between the levels or the \textbf{detuning}.

\subsection{Case of no detuning $\Delta = 0$}

Let us suppose that the diagonal elements are exactly zero. And for simplicity we will also keep $\Omega_y =0$ as it simply complicates the calculations without adding much to the discussion at this stage. The Hamiltonian reads then:
\begin{align}
\hat{H} = \frac{\hbar}{2}\left( \begin{array}{cc} 0  & \Omega\\ \Omega &0 \end{array} \right)
\end{align}

Quite clearly the states $\varphi_{1,2}$ are not the eigenstates of the system anymore. How should the system be described now ? We can once again diagonalize the system and write
\begin{align}
\hat{H}\ket{\varphi_{\pm}} = E_{\pm}\ket{\varphi_\pm}\\
E_{\pm} = \pm\frac{\hbar}{2}\Omega\\
\ket{\varphi_\pm} = \frac{\ket{0}\pm\ket{1}}{\sqrt{2}}
\end{align}
Two important consequences can be understood from this result:
\begin{enumerate}
\item The coupling of the two states shifts their energy by $\Omega$. This is the idea of level repulsion.
\item The coupled states are a superposition of the initial states.
\end{enumerate}
This is also a motivation the formulation of the 'bare' system for $\Omega = 0$ and the 'dressed' states for the coupled system.

\subsection{General case}

Quite importantly we can solve the system completely even in the general case. By diagonalizing  Eq. \eqref{Eq:TwoLevelGeneral} we obtain:
\begin{align}\label{eq:Epm}
 E_\pm = \pm \frac{\hbar}{2} \sqrt{\Delta^2+|\Omega|^2}
\end{align}
The energies can be nicely summarized as in Fig. \ref{326199}\selectlanguage{english}
\begin{figure}[h!]
\begin{center}
\includegraphics[width=0.70\columnwidth]{figures/Bildschirmfoto-2018-09-28-um-14-35-34/AvoidedCrossing}
\caption{{Anticrossing of energy levels.
{\label{326199}}%
}}
\end{center}
\end{figure}

~

The Eigenstates then read:
\begin{align}
\ket{\psi_+}&=\cos\left(\frac{\theta}{2}\right) \eexp{-i{\varphi}/{2}}\ket{0}+\sin\left(\frac{\theta}{2}\right) \eexp{i{\varphi}/{2}}\ket{1}, \label{eq:staticpsiplus}
\end{align}
\begin{align}
\ket{\psi_-}&=-\sin\left(\frac{\theta}{2}\right) \eexp{-i{\varphi}/{2}}\ket{0}+\cos\left(\frac{\theta}{2}\right) \eexp{i{\varphi}/{2}}\ket{1}, \label{eq:staticpsiminus}
\end{align}
where 
\begin{align} \label{eq:parameters}
\tan(\theta) = \frac{|\Omega|}{\Delta} 
\end{align}

\section{The Bloch sphere}

While we could just discuss the details of the above state in the abstract, it is extremely helpful to visualize the problem on the Bloch sphere. The idea of the Bloch sphere is that the we have a complex wave function of well defined norm and two free parameters. So it seems quite natural to look for a good representation of it. And this is the Bloch sphere as drawn in Fig. \ref{613576}.\selectlanguage{english}
\begin{figure}[h!]
\begin{center}
\includegraphics[width=0.70\columnwidth]{figures/BlochSphereWithVectorForLecture/BlochSphereWithVectorForLecture}
\caption{{The presentation of the eigenstate on the Bloch sphere.
{\label{613576}}%
}}
\end{center}
\end{figure}

We will see especially its usefulness especially as we discuss the dynamics of the two-state system.

\section{Dynamical Aspects}
\subsection{Time Evolution of $\ket{\psi(t)}$}
 After the static case we now want to investigate the dynamical properties of the two-state system. We calculate the time evolution of $\ket{\psi(t)} = c_0(t)\ket{0} + c_1(t)\ket{1}$ with the Schr\selectlanguage{ngerman}ödinger equation and the perturbed Hamiltonian \eqref{Eq:TwoLevelGeneral}:
\begin{align}
i\hbar \frac{d}{dt}\ket{\psi(t)}&=\hat{H}\ket{\psi(t)},\\
i \frac{d}{dt}\left(\begin{array}{c} c_0(t) \\ c_1(t) \end{array}\right) &= \frac{1}{2}\left( \begin{array}{cc} \Delta & \Omega \\ \Omega^* & -\Delta \end{array} \right) \left(\begin{array}{c} c_0(t) \\ c_1(t) \end{array} \right).
\end{align}

We have two coupled differential equations and we luckily already know how to solve them as we have calculated the two eigenenergies in the previous section. For the state $\ket{\psi(t)}$ we get
\begin{align}
 \ket{\psi(t)}=\lambda \eexp{-i{E_+}t/{\hbar}} \ket{\psi_+} + \mu \eexp{-i{E_-}t/{\hbar}} \ket{\psi_-} \label{eq:psitimeevolution}
\end{align}
with the factors $\lambda$ and $\mu$, which are defined by the initial state. The most common question is then what happens to the system if we start out in the bare state $\ket{0}$ and then let it evolve under coupling with a laser ? So what is the probability to find it in the other state $\ket{1}$:
\begin{align}
P_1(t)=\left|\braket{1|\psi(t)}\right|^2.
\end{align}
 As a first step, we have to apply the initial condition to \eqref{eq:psitimeevolution} and express $\ket{\varphi}$ in terms of \eqref{eq:staticpsiplus} and \eqref{eq:staticpsiminus}:
\begin{align}
\ket{\psi(0)} \overset{!}{=}& \ket{0}\\
  = & \eexp{i{\varphi}/{2}} \left[ \cos\left( \frac{\theta}{2}\right) \ket{\psi_+}-\sin\left(\frac{\theta}{2}\right)\ket{\psi_-}\right]
\end{align}
By equating the coefficients we get for $\lambda$ and $\mu$:
\begin{align}
\lambda = \eexp{i{\varphi}/{2}}\cos\left(\frac{\theta}{2}\right), \qquad  \mu = -\eexp{i{\varphi}/{2}}\sin\left(\frac{\theta}{2}\right).
\end{align}
One thus gets:
\begin{align}
\hspace{-2mm} P_1(t)=&\left|\braket{1|\psi(t)}\right|^2 \\
=& \left|\eexp{i\varphi} \sin\left(\frac{\theta}{2}\right)\cos\left(\frac{\theta}{2}\right)\left[\eexp{-i{E_+}t/{\hbar}} - \eexp{-i{E_-}t/{\hbar}}\right]\right|^2\\
=& \sin^2(\theta)\sin^2\left(\frac{E_+-E_-}{2\hbar}t\right)
\end{align}
$P_1(t)$ can be expressed with $\Delta$ and $\Omega$ alone. The obtained relation is called Rabi's formula:
\begin{align}
 P_1(t)=\frac{1}{1+\left(\frac{\Delta}{|\Omega|}\right)^2}\sin^2\left(\sqrt{|\Omega|^2+\Delta^2}\frac{t}{2}\right)
\end{align}\selectlanguage{english}
\begin{figure}[h!]
\begin{center}
\includegraphics[width=0.70\columnwidth]{figures/Bildschirmfoto-2018-09-28-um-14-43-51/RabiOscillation}
\caption{{Rabi oscillations
{\label{833990}}%
}}
\end{center}
\end{figure}

\subsection{Visualization of the dynamics in the spin picture}

While the previous derivation might be the standard one, which certainly leads to the right results it might not be the most intuitive way of thinking about the dynamics. They become actually quite transparent in the spin language and on the Bloch sphere. So let us go back to the formulation of the Hamiltonian in terms of spins as in Eq. \eqref{Eq:HamSpin}.

How would the question of the time evolution from $0$ to $1$ and back go now ? Basically, we would assume that the spin has been initialize into one of the eigenstates of the $z$-basis and now starts to rotate in some magnetic field. How ? This can be nicely studied in the Heisenberg picture, where operators have a time evolution. In the Heisenberg picture we have:
\begin{align}
\frac{d}{dt} \hat{s}_i &= \frac{i}{\hbar}\left[\hat{H},\hat{s}_i\right]\\
\frac{d}{dt} \hat{s}_i &= \frac{i}{\hbar}\sum_j B_j \left[\hat{s}_j,\hat{s}_i\right]\\
 \end{align}
So to understand we time evolution, we only need to employ the commutator relationships between the spins:
\begin{align}
[ s_x, s_y] = \hbar is_z~~[ s_y, s_z] = \hbar is_x~~[ s_z, s_x] = \hbar is_y
\end{align}
For the specific case of $B_x=\Omega$, $B_y = B_z = 0$, we have then:
\begin{align}
\frac{d}{dt} \hat{s}_x &= 0\\
\frac{d}{dt} \hat{s}_y &= -\Omega \hat{s}_z\\
\frac{d}{dt} \hat{s}_z &= \Omega \hat{s}_y
 \end{align}
 
 So applying a field in x-direction leads to a rotation of the spin around the $x$ axis with velocity $\Omega$. We can now use this general picture to understand the dynamics as rotations around an axis, which is defined by the different components of the magnetic field.

\section{A few words on the quantum information notation}

The qubit is THE basic ingredient of quantum computers. A nice way to play around with them is actually the \href{https://quantum-computing.ibm.com/}{IBM Quantum experience}. However, you will typically not find Pauli matrices etc within these systems. The typical notation there is:
\begin{itemize}
\item $R_x(\phi)$ is a rotation around the x-axis for an angle $\phi$.
\item Same holds for $R_y$ and $R_z$.
\item $X$ denotes the rotation around the x axis for an angle $\pi$. So it transforms $\ket{1}$ into  $\ket{0}$ and vise versa.
\item $Z$ denotes the rotation around the x axis for an angle $\pi$. So it transforms $\ket{+}$ into  $\ket{-}$ and vise versa.
\end{itemize}
The most commonly used gate is actually one that we did not talk about at all, it is the \textit{Hadamard} gate, which transforms $\ket{1}$ into  $\ket{-}$ and $\ket{0}$ into  $\ket{+}$:
\begin{align}
\hat{H}\ket{1} &= \ket{-}  ~   \hat{H}\ket{0} &= \ket{+}\\
\hat{H}\ket{-} &= \ket{1}  ~   \hat{H}\ket{+} &= \ket{0}
\end{align}

In the \href{https://www.authorea.com/326506/emMDRkXxtm44IKqpCtDi6g}{forth lecture} we will see how it is that a time-dependent field can actually couple two atomic states, which are normally of very different energies.

\selectlanguage{english}
\FloatBarrier
\bibliographystyle{plainnat}
\bibliography{bibliography/converted_to_latex.bib%
}

\end{document}

