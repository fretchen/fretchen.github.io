\documentclass[10pt]{article}

\usepackage{fullpage}
\usepackage{setspace}
\usepackage{parskip}
\usepackage{titlesec}
\usepackage[section]{placeins}
\usepackage{xcolor}
\usepackage{breakcites}
\usepackage{lineno}
\usepackage{hyphenat}





\PassOptionsToPackage{hyphens}{url}
\usepackage[colorlinks = true,
            linkcolor = blue,
            urlcolor  = blue,
            citecolor = blue,
            anchorcolor = blue]{hyperref}
\usepackage{etoolbox}
\makeatletter
\patchcmd\@combinedblfloats{\box\@outputbox}{\unvbox\@outputbox}{}{%
  \errmessage{\noexpand\@combinedblfloats could not be patched}%
}%
\makeatother


\usepackage[round]{natbib}
\let\cite\citep




\renewenvironment{abstract}
  {{\bfseries\noindent{\abstractname}\par\nobreak}\footnotesize}
  {\bigskip}

\titlespacing{\section}{0pt}{*3}{*1}
\titlespacing{\subsection}{0pt}{*2}{*0.5}
\titlespacing{\subsubsection}{0pt}{*1.5}{0pt}


\usepackage{authblk}


\usepackage{graphicx}
\usepackage[space]{grffile}
\usepackage{latexsym}
\usepackage{textcomp}
\usepackage{longtable}
\usepackage{tabulary}
\usepackage{booktabs,array,multirow}
\usepackage{amsfonts,amsmath,amssymb}
\providecommand\citet{\cite}
\providecommand\citep{\cite}
\providecommand\citealt{\cite}
% You can conditionalize code for latexml or normal latex using this.
\newif\iflatexml\latexmlfalse
\AtBeginDocument{\DeclareGraphicsExtensions{.pdf,.PDF,.eps,.EPS,.png,.PNG,.tif,.TIF,.jpg,.JPG,.jpeg,.JPEG}}

\usepackage[utf8]{inputenc}
\usepackage[ngerman,english]{babel}








\usepackage{amsmath}
\newcommand{\bra}[1]{\ensuremath{\left\langle#1\right|}}
\newcommand{\ket}[1]{\ensuremath{\left|#1\right\rangle}}
\newcommand{\braket}[1]{\ensuremath{\left\langle#1\right\rangle}}
\newcommand{\rhohat}{\hat{\rho}}
\newcommand{\tr}[1]{\mathrm{tr}(#1)}
\newcommand{\trarb}[2]{\mathrm{tr}_{#1}(#2)}
\newcommand{\vv}[1]{\mathbf{#1}}
\newcommand*\dif{\mathop{}\!\mathrm{d}}
\newcommand{\eexp}[1]{\mathrm{e}^{#1}}

\begin{document}

\title{Lecture 2 - A few more cooking recipes for quantum mechanics}



\author[1]{Fred Jendrzejewski}%
\author[2]{Selim Jochim}%
\affil[1]{Kirchhoff-Institut für Physik}%
\affil[2]{Physikalisches Institut der Universität Heidelberg}%


\vspace{-1em}



  
  \date{January 03, 2025}


\begingroup
\let\center\flushleft
\let\endcenter\endflushleft
\maketitle
\endgroup





\selectlanguage{english}
\begin{abstract}
In this second lecture we will finish the discussion of the basic cooking recipes and discuss a few of the consequences like the uncertainty relation, the existance of wave packages and the Ehrenfest theorem.%
\end{abstract}%



\sloppy


In the first lecture \cite{Jendrzejewski} we discussed briefly the basic principles of quantum mechanics like operators, state vectors and the Schr\selectlanguage{ngerman}ödinger equation. We will finish this discussion today and then introduce the most important consequences. We will continue to closely follow the discussion of the introductory chapter of Ref. \cite{2006}

\section{Composite systems}
It is actually quite rare that we can label the system with a single quantum number. Any atom will involve spin, position, angular momentum. Other examples might just involve two spin which we observe. So the question is then on how we label those systems. We then have two questions to answer:
\begin{enumerate}
\item How many labels do we need for a system to fully determine its quantum state ?
\item Once I know all the labels, how do I construct the full state out of them ?
\end{enumerate}
We will actually discuss the second question first as it sets the notation for the first question.



\subsection{Entangled States}
 
In AMO we typically would like to characterize is the state of an electron in a hydrogen atom. We need to define its angular momentum label $L$, which might be 0, 1, 2 and also its electron spin $S$, which might be $\{\uparrow, \downarrow\}$. It state is then typically labelled as something like $\ket{L=0, S=\uparrow} = \ket{0,\uparrow}$ etc.

Another, simple example is that of two spins, each one having two possible states $\{\uparrow, \downarrow\}$. This is a standard problem in optical communication, where you send correlated photons with a certain polarization to different people. We will typically call them Alice and Bob \footnote{And if someone wants to listen the person is called Eve}.

We now  would like to understand than if we can disentangle the information about the different labels. Naively, we can now  associate with Alice one set of outcomes and describe it by some state $\ket{\psi_{A}}$ and the Bob has another set $\ket{\psi_{B}}$:
\begin{align}\label{Eq:Local}
\ket{\psi_A}&= a_{\uparrow} \ket{\uparrow_A}+ a_{\downarrow} \ket{\downarrow_A}\\
\ket{\psi_B}&= b_{\uparrow} \ket{\uparrow_B}+ b_{\downarrow} \ket{\downarrow_B}
\end{align}

The full state will then be described by the possible outcomes $\{\uparrow_A\uparrow_B,\downarrow_A\uparrow_B,\uparrow_A\downarrow_B, \downarrow_A\downarrow_B\}$. We can then write:
\begin{align}
\ket{\psi} &= \alpha_{\uparrow\uparrow}(\ket{\uparrow_A}\otimes\ket{\uparrow_B})+\alpha_{\uparrow\downarrow}(\ket{\uparrow_A}\otimes\ket{\downarrow_B})+\alpha_{\downarrow\uparrow}(\ket{\downarrow_A}\otimes\ket{\uparrow_B})+\alpha_{\downarrow\downarrow}(\ket{\downarrow_A}\otimes\ket{\downarrow_B})\\
&= \alpha_{\uparrow\uparrow}\ket{\uparrow\uparrow}+\alpha_{\uparrow\downarrow}\ket{\uparrow\downarrow}+\alpha_{\downarrow\uparrow}\ket{\downarrow \uparrow}+\alpha_{\downarrow\downarrow}\ket{\downarrow\downarrow}
\end{align}
So we will typically just plug the labels into a single ket and drop indices, to avoid rewriting the tensor symbol each time. We say that a state is \textit{separable}, if we can write it as a product of the two individual states \eqref{Eq:Local}:
\begin{align}
\ket{\psi} &= \ket{\psi_A}\otimes\ket{\psi_B}\\
&=a_{\uparrow} b_\uparrow \ket{\uparrow\uparrow}+a_{\downarrow} b_\uparrow \ket{\downarrow\uparrow}+a_{\uparrow} b_\downarrow \ket{\uparrow\downarrow}+a_{\downarrow} b_\downarrow \ket{\downarrow\downarrow}
\end{align}

All other states are called \textit{entangled}. The most famous entangled states are the Bell states:
\begin{align}
\ket{\psi_\textrm{Bell}}=\frac{\ket{\uparrow\uparrow}+\ket{\downarrow\downarrow}}{\sqrt{2}}
\end{align}

In general we will say that the quantum system is formed by two subsystems $S_1$ and $S_2$. If they are independent we can write each of them as:

\begin{align}
\ket{\psi_1}&=\sum_m^M a_m \ket{\alpha_m},\\
\ket{\psi_2}&=\sum_n^N b_n \ket{\beta_n}.

\end{align}
In general we will then write:
\begin{align}
\ket{\psi}=\sum_m^M \sum_n^N c_{mn}(\ket{\alpha_m}\otimes \ket{\beta_n}).
\end{align}
So we can determine such a state by $M \times N$ numbers $c_{mn}$ here.  If the states are \textit{separable}, we can write $\ket{\psi}$ as a product of the individual states:
\begin{align}
 \label{eq:psientangled} 
\ket{\psi}&=\ket{\psi_1}\otimes\ket{\psi_2}=\left(\sum_m^M a_m \ket{\alpha_m}\right) \otimes \left(\sum_n^N b_n \ket{\beta_n}\right)
\end{align}
\begin{align}
\ket{\psi}&=\sum_m^M \sum_n^N a_m b_n \ket{\alpha_m} \otimes \ket{\beta_n}. \label{eq:psientangled3} 
\end{align}

Separable states thus only describes a small subset of all possible states. 

\section{Statistical Mixtures and Density Operator}

Having set up the formalism for writing down the full quantum state with plenty of labels, we have to solve the next problem. As an experimentalist, you will rarely measure all of them. This means that you only perform a partial measurement and you have only partial information of the system. The extreme case is the thermodynamic ensemble, where we measure only temperature to describe $10^{23}$ particles.

A similiar problem arises for Alice and Bob. They typically measure the state of the qubit in their lab without knowing what the other did. So they need some way to describe the system locally. This is done through the density operator approach.

In the density operator approach the state of the system is described by a Hermitian density operator

\begin{align}
 \hat{\rho} = \sum_{n=1}^N p_n \ket{\phi_n}\bra{\phi_n}.
\end{align}
Here, $\bra{\phi_n}$ are the eigenstates of $\hat{\rho}$, and $p_n$ are the probabilities to find the system in the respective states $\ket{\phi_n}$. The trace of the density operator is the sum of all probabilities $p_n$:
\begin{align}
  \tr{\rhohat} = \sum p_n = 1.
\end{align}

For a pure state $\ket{\psi}$, we get $p_n=1$ for only one value of $n$. For every other $n$, the probabilities vanish. We thus obtain a ``pure'' density operator $\rhohat_{\text{pure}}$ which has the properties of a projection operator:

\begin{align}
\rhohat_{\text{pure}} = \ket{\psi}\bra{\psi} \qquad \Longleftrightarrow \qquad \rhohat^2 = \rhohat.
\end{align}
For the simple qubit we then have:
\begin{align}
  \rhohat &= \left(\alpha_\uparrow\ket{\uparrow}+\alpha_\downarrow\ket{\downarrow}\right)\left(\alpha_\uparrow^*\bra{\uparrow}+\alpha_\downarrow^*\bra{\downarrow}\right)\\
  &= |\alpha_\uparrow|^2\ket{\uparrow}\bra{\uparrow}+|\alpha_\downarrow|^2\ket{\downarrow}\bra{\downarrow}+\alpha_\downarrow\alpha_\uparrow^*\ket{\downarrow}\bra{\uparrow}+\alpha_\uparrow\alpha_\downarrow^*\ket{\uparrow}\bra{\downarrow}
\end{align}
Then it is even simpler to write in matrix form:
\begin{align}
  \rhohat &= \left(\begin{array}{cc}
  |\alpha_\uparrow|^2&\alpha_\uparrow\alpha_\downarrow^*\\
  \alpha_\downarrow\alpha_\uparrow^*&|\alpha_\downarrow|^2
  \end{array}\right)
\end{align}
For a thermal state on the other hand we have:
\begin{align}
\rhohat_{\text{thermal}} = \sum_{n=1}^N \frac{e^{-\frac{E_n}{k_BT}}}{Z} \ket{\phi_n}\bra{\phi_n}\text{ with }Z = \sum_{n=1}^N e^{-\frac{E_n}{k_BT}}
\end{align}
With this knowledge we can now determine the result of a measurement of an observable $A$ belonging to an operator $\hat{A}$. For the pure state $\ket{\psi}$ we get:
%
\begin{align}
\langle \hat{A}\rangle = \bra{\psi} \hat{A} \ket{\psi}.
\end{align}
%
For a mixed state we get:
%
\begin{align}
\langle \hat{A}\rangle = \tr{\rhohat \cdot \hat{A}} = \sum_n {p_n} \bra{\phi_n} \hat{A} \ket{\phi_n}.
\end{align}
The time evolution of the density operator can be expressed with the von Neumann equation:
\begin{align}
i\hbar \partial_{t}\rhohat(t) = [\hat{H}(t),\rhohat(t)].
\end{align}

\subsection{Back to partial measurements}

We can now come back to the correlated photons sent to Alice and Bob, sharing a Bell pair. They full density matrix is then especially simple:
\begin{align}
  \rhohat &= \left(\begin{array}{cccc}
  \frac{1}{2}& 0& 0 &\frac{1}{2}\\
  0 & 0 & 0& 0\\
  0&0&0&0\\
    \frac{1}{2}&0&0&\frac{1}{2}
  \end{array}\right)
\end{align}

Let us write the system as $S = S_A \otimes S_B$.  If we are looking for the density operator $\rhohat_i$ of each individual, we can simply write:

\begin{align}
\rhohat_A=&\trarb{B}{\rhohat},\\
\rhohat_B=&\trarb{A}{\rhohat},
\end{align}
where $\hat{\rho}=\ket{\psi}\bra{\psi}$ and $\trarb{j}{\rhohat}$ is the trace over the Hilbert space of subsystem $j$. 

To reduce the density matrix of the Bell state it is actually helpful to write out the definitions:
\begin{align}
\trarb{B}{\rhohat} &= \bra{\uparrow_B}\rhohat\ket{\uparrow_B}+\bra{\downarrow_B}\rhohat\ket{\downarrow_B}\\
&=\frac{1}{2}\left(\ket{\uparrow_A}\bra{\uparrow_A}+\ket{\downarrow_A}\bra{\downarrow_A}\right)
\end{align}
So we end up with the fully mixed state:
\begin{align}
  \rhohat_{A,B} &= \left(\begin{array}{cc}
  \frac{1}{2}&0\\
  0&\frac{1}{2}
  \end{array}\right)
\end{align}
Alice and Bob are simply cossing a coin if they ignore the outcome of the other member. But once they start comparing results we will see that the quantum case can dramatically differ from the classical case. This will be the content of lecture 12  \cite{entanglement}.

\section{Important Consequences of the Principles}
\subsection{Uncertainty Relation}

The product of the variances of two noncommuting operators has a lower limit:
\begin{align}
    \Delta \hat{A} \cdot \Delta \hat{B} \geq \frac{1}{2} \left| \braket{\left[\hat{A},\hat{B}\right]} \right|,
\end{align}
where the variance is defined as $\Delta \hat{A} = \sqrt{\braket{\hat{A}^2}-\braket{\hat{A}}^2}$.

\textbf{Examples.}
\begin{align}
\left[ \hat{x}, \hat{p} \right] &= i \hbar \\
\left[ \hat{J}_i , \hat{J}_j \right] &= i \hbar \epsilon_{ijk} \hat{J}_k
\end{align}

\textbf{Note.} This is a statement about the \emph{state} itself, and not the measurement!

\subsection{Ehrenfest Theorem}
With the Ehrenfest theorem, one can determine the time evolution of the expectation value of an operator $\hat{A}$:
\begin{align}
 \frac{d}{dt}\braket{\hat{A}}=\frac{1}{i\hbar}\braket{\left[\hat{A},\hat{H}\right]}+\braket{\partial_t{\hat{A}}{t}}. \label{eq:ehrenfest}
\end{align}
If $\hat{A}$ is time-independent and $\left[\hat{A},\hat{H}\right]=0$, the expectation value $\braket{\hat{A}}$ is a constant of the motion.

\subsection{Complete Set of Commuting Observables}

A set of commuting operators $\{\hat{A},\hat{B},\hat{C},\cdots,\hat{X}\}$ is considered a complete set if their common eigenbasis is unique. Thus, the measurement of all quantities $\{A,B,\cdots,X\}$ will determine the system uniquely. The clean identification of such a Hilbert space can be quite challenging and a nice way of its measurment even more. Coming back to our previous examples:

\begin{enumerate}
\item Performing the full spectroscopy of the atom. Even for the hydrogen atom we will see that the full answer can be rather involved...
\item The occupation number is rather straight forward. However, we have to be careful that we really collect a substantial amount of the photons etc.
\item Are we able to measure the full position information ? What is the resolution of the detector and the point-spread function ?
\item Here it is again rather clean to put a very efficient detector at the output of the two arms ...
\item What are the components of the spin that we can access ? The $z$ component does not commute with the other components, so what should we measure ?
\end{enumerate}

In the \href{https://www.authorea.com/326444/GsbfEypTdf4dvncV23L8_Q}{third lecture} of this course will start to apply these discussions to the two-level system, which is one of the simplest yet most powerful models of quantum mechanics.

\selectlanguage{english}
\FloatBarrier
\bibliographystyle{plainnat}
\bibliography{bibliography/converted_to_latex.bib%
}

\end{document}

