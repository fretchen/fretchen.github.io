\documentclass[10pt]{article}

\usepackage{fullpage}
\usepackage{setspace}
\usepackage{parskip}
\usepackage{titlesec}
\usepackage[section]{placeins}
\usepackage{xcolor}
\usepackage{breakcites}
\usepackage{lineno}
\usepackage{hyphenat}





\PassOptionsToPackage{hyphens}{url}
\usepackage[colorlinks = true,
            linkcolor = blue,
            urlcolor  = blue,
            citecolor = blue,
            anchorcolor = blue]{hyperref}
\usepackage{etoolbox}
\makeatletter
\patchcmd\@combinedblfloats{\box\@outputbox}{\unvbox\@outputbox}{}{%
  \errmessage{\noexpand\@combinedblfloats could not be patched}%
}%
\makeatother


\usepackage[round]{natbib}
\let\cite\citep




\renewenvironment{abstract}
  {{\bfseries\noindent{\abstractname}\par\nobreak}\footnotesize}
  {\bigskip}

\titlespacing{\section}{0pt}{*3}{*1}
\titlespacing{\subsection}{0pt}{*2}{*0.5}
\titlespacing{\subsubsection}{0pt}{*1.5}{0pt}


\usepackage{authblk}


\usepackage{graphicx}
\usepackage[space]{grffile}
\usepackage{latexsym}
\usepackage{textcomp}
\usepackage{longtable}
\usepackage{tabulary}
\usepackage{booktabs,array,multirow}
\usepackage{amsfonts,amsmath,amssymb}
\providecommand\citet{\cite}
\providecommand\citep{\cite}
\providecommand\citealt{\cite}
% You can conditionalize code for latexml or normal latex using this.
\newif\iflatexml\latexmlfalse
\AtBeginDocument{\DeclareGraphicsExtensions{.pdf,.PDF,.eps,.EPS,.png,.PNG,.tif,.TIF,.jpg,.JPG,.jpeg,.JPEG}}

\usepackage[utf8]{inputenc}
\usepackage[ngerman,english]{babel}








\usepackage{siunitx}
\usepackage{amsmath}
\newcommand{\bra}[1]{\ensuremath{\left\langle#1\right|}}
\newcommand{\ket}[1]{\ensuremath{\left|#1\right\rangle}}
\newcommand{\braket}[1]{\ensuremath{\left\langle#1\right\rangle}}
\newcommand{\rhohat}{\hat{\rho}}
\newcommand{\tr}[1]{\mathrm{tr}(#1)}
\newcommand{\trarb}[2]{\mathrm{tr}_{#1}(#2)}
\newcommand{\vv}[1]{\mathbf{#1}}
\newcommand*\dif{\mathop{}\!\mathrm{d}}
\newcommand{\eexp}[1]{\mathrm{e}^{#1}}
\newcommand*\ch[1]{\ensuremath{\mathrm{#1}}}

\begin{document}

\title{Lecture 22 - Quantum degenerate gases}



\author[1]{Fred Jendrzejewski}%
\author[2]{Selim Jochim}%
\author[2]{Matthias Weidemüller}%
\affil[1]{Kirchhoff-Institut für Physik}%
\affil[2]{Physikalisches Institut der Universität Heidelberg}%


\vspace{-1em}



  
  \date{January 07, 2025}


\begingroup
\let\center\flushleft
\let\endcenter\endflushleft
\maketitle
\endgroup





\selectlanguage{english}
\begin{abstract}
We will discuss today how quantum effects emerge in degenerate gases and how they are connected to problems that come traditionally from condensed-matter physics.%
\end{abstract}%



\sloppy


We have seen now, how we can cool down the gases to very low temperatures. At some point we expect their quantum character to play a role, this is the regime degenerate quantum gases. We will first discuss qualitatively when we expect this regime to be reached and then discuss the case of ideal Bose gas vs fermi Gas.

\section{Quantum degenerate gases}
Massive particles only seem point-like for high temperatures as they have very large momentum. The associated wavelength is called the the \textit{de Broglie} wavelength
%
\begin{align}
\lambda_\text{dB} = \frac{h}{p} = \frac{h}{\sqrt{2mE_{kin}}} = \frac{h}{\sqrt{2mk_\textrm{b}T}}
\end{align}

When do we expect this quantum character to be relevant ?
\begin{enumerate}
\item If the cloud consists of many-particles, we can describe it best by its density $n$. The quantum correlations will start to become relevant as the particles 'see' each other. The characteristic relation is here the regime of:
\begin{align}
n\lambda_{dB}^3\sim 1
\end{align}
\item If there is only a single particle trapped in the potential, the quantum character will become important as its energy is in the order of the level spacing of the trapped states. For a harmonic trap whose characteristic length scale is $a_{HO}=\sqrt{\hbar/m\omega}$, this means that:
\begin{align}
\lambda_{dB}\sim a_{HO}
\end{align}
\end{enumerate}
Both regimes are nowadays frequently reached in experiments:
\begin{enumerate}
\item Degenerate quantum gases have become a major branch of atomic physics, as they allow for the clean emulation of complex many-body problems known from condensed-matter physics \cite{Bloch_2008}.
\item Single ions can be cooled within tight traps into the ground state. They have now become a major platform for quantum computing \cite{Leibfried_2003}.
\end{enumerate}

We will now discuss first the state of the degenerate Bose gas for instructiveness.

\section{Ideal Quantum Gases}
Let us start with non-interacting particles, which might be bosons or fermions. We will call $\{\ket{\lambda}\}$ the base of eigenvectors with energies:
\begin{align}
\hat{h}\ket{\lambda}=\epsilon_\lambda \ket{\lambda}
\end{align}
Each of these states is allowed to contain a multitude of particles. This can be nicely described through occupation numbers, and counted through the raising and lowering operators:
\begin{align}
\hat{n}_\lambda &= \hat{a}^\dag_\lambda \hat{a}_\lambda\\
\hat{n}_\lambda \ket{\cdots, n_\lambda, \cdots} &= n_\lambda \ket{\cdots, n_\lambda, \cdots}
\end{align}
We can then write down the Hamiltonian as:
\begin{align}
\hat{H} &= \sum_\lambda \epsilon_\lambda a_\lambda^\dag a_\lambda
\end{align}
The total number of atoms is then gives as:
\begin{align}
\hat{N} &= \sum_\lambda  a_\lambda^\dag a_\lambda
\end{align}

The connection between this microscopic Hamiltonian and thermodynamics is then done through the grand-canonical ensemble, which is describe by the chemical potential $\mu$ and the temperature $T$. The density operator reads now:
\begin{align}
\hat{\rho} &= \frac{e^{\frac{\mu}{k_BT} \hat{N}-\frac{1}{k_BT}\hat{H}}}{Z_G}
\end{align}
The partition function reads here:
\begin{align}
Z_G = \mathrm{Tr}\left(e^{\frac{\mu}{k_BT} \hat{N}-\frac{1}{k_BT}\hat{H}}\right)
\end{align}
To calculate the partition function, we can work with the following notation for the states:
\begin{align}
\ket{l}&\equiv\ket{n_1, n_2,\cdots}\\
\hat{N}\ket{l} &= N_l\ket{l}\text{ with }N_l=\sum_\lambda n_\lambda\\
\hat{H}\ket{l} &= E_l\ket{l}\text{ with }E_l=\sum_\lambda \epsilon_\lambda n_\lambda
\end{align}
The partition function decomposes then in:
\begin{align}
Z_G &= \sum_l\left(e^{\frac{\mu}{k_BT} N_l-\frac{1}{k_BT}E_l}\right)\\
&= \sum_{n_1, n_2, ...} e^{\frac{\mu-\epsilon_1}{k_BT} n_1}\cdot e^{\frac{\mu-\epsilon_2}{k_BT} n_2}\cdots\\
&=\prod_\lambda \xi_\lambda
\end{align}
In the last line we have defined:
\begin{align}\label{Eq:XiGeneral}
\xi_\lambda &= \sum_{n_\lambda} e^{\frac{\mu-\epsilon_\lambda}{k_BT} n_\lambda}
\end{align}


The total number of atoms in the clouds the derivative of the grand canoncial partition function with respect to $z = e^{\frac{\mu}{k_B T}}$:
\begin{align}
N &= \langle \hat{N}\rangle\\
&= z\partial_z \ln Z_G
\end{align}
We can relate to the partition function 
The new parameters are
How does indistinguishability matter in a gas? We can count the possible states and infer the possible occupation numbers!





\subsection{Fermions}

For fermions we can only have occupation numbers of 0 or 1. We can therefore solve easily \eqref{Eq:XiGeneral} as:
\begin{align}
\xi_\lambda^F = 1+e^{\frac{\mu-\epsilon_\lambda}{k_BT}}
\end{align}

The resulting atom number is:
\begin{align}
N = \sum_\lambda n_\lambda\\
n_\lambda^F  = \frac{1}{e^{\frac{\epsilon_\lambda-\mu}{k_BT}}+1}
\end{align}
This is the \textit{Fermi-Dirac} distribution. We reach the classical limit for large negative chemical potential, where we recover the \textit{Boltzmann} distribution:
\begin{align}
n_\lambda  \simeq z e^{-\frac{\epsilon_\lambda}{k_BT}}
\end{align}

The limit of very low temperatures is reached for large positive chemical potential $\mu\gg k_BT$. This is the typical regime of condensed matter physics. We then have:
\begin{align}
n_\lambda \approx 1 \text{ for }\epsilon_\lambda < \mu\\
n_\lambda \approx 0 \text{ for }\epsilon_\lambda > \mu
\end{align}
In this regime we typically call $\mu$ the Fermi energy and rename it $E_F$. The very high fidelity of occupation in low-lying states is successfully employed in the Jochim group for the deterministic preparation of fermionic clouds with very few atoms \cite{Serwane_2011}. 

\subsection{Bosons}
For Bosons we have sum over all positive integer numbers. However, the sum in \eqref{Eq:XiGeneral} becomes now a geometric series and we obtain:
\begin{align}
\xi_\lambda^B = \frac{1}{1-e^{\frac{\mu-\epsilon_\lambda}{k_BT}}}
\end{align}
The total atom is then once again decomposed like for the fermions, but we obtain:
\begin{align}\label{Eq:nB}
n_\lambda^B  = \frac{1}{e^{\frac{\epsilon_\lambda-\mu}{k_BT}}-1}
\end{align}
This function diverages if the first part becomes to close to 1, i.e. the chemical potential is limited to values between $-\infty$ and $\epsilon_\text{min}$. For large negative values we obtain again the Boltzmann distribution. Something interesting is happening on the other hand as the chemical potential is getting closer to $\epsilon_{min}$. The difference was strikingly observed in the experiments represented in Fig. \ref{249436}\selectlanguage{english}
\begin{figure}[h!]
\begin{center}
\includegraphics[width=0.70\columnwidth]{figures/Bildschirmfoto-2018-11-28-um-11-32-40/Bildschirmfoto-2018-11-28-um-11-32-40}
\caption{{Cooling a cloud of fermions (right hand side) below the Fermi
temperature shows a sharpened profile. However, the cloud size does not
shrink once it is below~\(T_F\) as the all low lying energy
states are already occupied. For bosonic particles (left hand side) the
cloud size keeps shrinking.
{\label{249436}}%
}}
\end{center}
\end{figure}

\section{Bose-Einstein condensation}

As the the chemical becomes close to the ground state energy we can develop \eqref{Eq:nB} and find:
\begin{align}
N_0 \simeq \frac{k_B T}{\epsilon_{min}-\mu}
\end{align}
So it diverges. Typically this state is then simply ignored and the number of atoms in the excited states is calculated:
\begin{align}
N' = N - N_0 = \sum_{\lambda>0} \frac{1}{e^{\frac{\epsilon_\lambda - \mu}{k_BT}}-1}
\end{align}
Quite remarkably this number has an upper limit for a given temperature as the chemical potential cannot increase to values above $\epsilon_{min}$.
\begin{align}
N_{max} = N - N_0 = \sum_{\lambda>0} \frac{1}{e^{\frac{\epsilon_\lambda - \epsilon_{min}}{k_BT}}-1}
\end{align}
. If the number of atoms is larger than $N_{max}$ they have to go into the ground state. This transition is known as Bose-Einstein condensation. The critical number of atoms for a given temperature can be calculated in a number of cases. In a harmonic trap we find:
\begin{align}
N_{ho} &= 1.202 \left(\frac{k_B T}{\hbar \omega}\right)^3 
\end{align}
In a three dimensional box it is better to calculate the critical density and we obtain the condition:
\begin{align}
n_c\lambda_{dB}^3\simeq 2.612
\end{align}
A more detailled discussion can be found in \cite{Dalfovo_1999, Leggett_2001}. We will now discuss on the blackboard how the observation of Bose-Einstein condensation was achieved by the group of Eric Cornell and Carl Wieman as well as the team of Wolfgang Ketterle, gaining all three of them the Nobel prize in 2001 \cite{Cornell_2002,Ketterle_2002}. 

A sketch of the observed signature can be found in Fig. \ref{460581}\selectlanguage{english}
\begin{figure}[h!]
\begin{center}
\includegraphics[width=0.70\columnwidth]{figures/Bildschirmfoto-2019-01-16-um-09-05-27/Bildschirmfoto-2019-01-16-um-09-05-27}
\caption{{Observation of Bose-Einstein condensation as reported
in~\protect\cite{Anderson_1995}.
{\label{460581}}%
}}
\end{center}
\end{figure}



\section{Synthetic quantum systems}

The observation of Bose-Einstein condensation was the first step in  a major shift in atomic physics as it brought it closer to condensed-matter physics. Actually, Bose-Einstein condensate were long predicted to be at the origin of Helium superfluidity. Further, superconductivity is also based on the idea of condensing 'cooper' pairs into a marcoscopic wavefunction \cite{superconductivity}. Numerous experiments studied therefore the superfluid properties of Bose-Einstein condensated as already discussed in the previous citations. One of the most striking feature of superfluidity is the resisitivity without flow as rather strikingly observed in the experiments on Josephson oscillations by the Oberthaler group shown in Fig.\selectlanguage{english}
\begin{figure}[h!]
\begin{center}
\includegraphics[width=0.70\columnwidth]{figures/Bildschirmfoto-2019-01-16-um-09-27-55/Bildschirmfoto-2019-01-16-um-09-27-55}
\caption{{Observation of the Josephson effect in Bose-Einstein condensates as
presented in \protect\cite{Albiez_2005}
{\label{448277}}%
}}
\end{center}
\end{figure}



\section{Optical lattices}

This idea of making close connections between condensed-matter physics and atomic physics was pushed even further by the introduction of optical lattices \cite{Bloch_2008}.

\selectlanguage{english}
\FloatBarrier
\bibliographystyle{plainnat}
\bibliography{bibliography/converted_to_latex.bib%
}

\end{document}

