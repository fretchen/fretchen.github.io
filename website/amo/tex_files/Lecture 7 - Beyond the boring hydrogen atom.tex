\documentclass[10pt]{article}

\usepackage{fullpage}
\usepackage{setspace}
\usepackage{parskip}
\usepackage{titlesec}
\usepackage[section]{placeins}
\usepackage{xcolor}
\usepackage{breakcites}
\usepackage{lineno}
\usepackage{hyphenat}





\PassOptionsToPackage{hyphens}{url}
\usepackage[colorlinks = true,
            linkcolor = blue,
            urlcolor  = blue,
            citecolor = blue,
            anchorcolor = blue]{hyperref}
\usepackage{etoolbox}
\makeatletter
\patchcmd\@combinedblfloats{\box\@outputbox}{\unvbox\@outputbox}{}{%
  \errmessage{\noexpand\@combinedblfloats could not be patched}%
}%
\makeatother


\usepackage[round]{natbib}
\let\cite\citep




\renewenvironment{abstract}
  {{\bfseries\noindent{\abstractname}\par\nobreak}\footnotesize}
  {\bigskip}

\titlespacing{\section}{0pt}{*3}{*1}
\titlespacing{\subsection}{0pt}{*2}{*0.5}
\titlespacing{\subsubsection}{0pt}{*1.5}{0pt}


\usepackage{authblk}


\usepackage{graphicx}
\usepackage[space]{grffile}
\usepackage{latexsym}
\usepackage{textcomp}
\usepackage{longtable}
\usepackage{tabulary}
\usepackage{booktabs,array,multirow}
\usepackage{amsfonts,amsmath,amssymb}
\providecommand\citet{\cite}
\providecommand\citep{\cite}
\providecommand\citealt{\cite}
% You can conditionalize code for latexml or normal latex using this.
\newif\iflatexml\latexmlfalse
\AtBeginDocument{\DeclareGraphicsExtensions{.pdf,.PDF,.eps,.EPS,.png,.PNG,.tif,.TIF,.jpg,.JPG,.jpeg,.JPEG}}

\usepackage[utf8]{inputenc}
\usepackage[ngerman,english]{babel}








\usepackage{siunitx}
\usepackage{amsmath}
\newcommand{\bra}[1]{\ensuremath{\left\langle#1\right|}}
\newcommand{\ket}[1]{\ensuremath{\left|#1\right\rangle}}
\newcommand{\braket}[1]{\ensuremath{\left\langle#1\right\rangle}}
\newcommand{\rhohat}{\hat{\rho}}
\newcommand{\tr}[1]{\mathrm{tr}(#1)}
\newcommand{\trarb}[2]{\mathrm{tr}_{#1}(#2)}
\newcommand{\vv}[1]{\mathbf{#1}}
\newcommand*\dif{\mathop{}\!\mathrm{d}}
\newcommand{\eexp}[1]{\mathrm{e}^{#1}}
\newcommand{\Hzero}{\hat{H}_0} 
\newcommand{\Wop}{\hat{W}} 

\begin{document}

\title{Lecture 7 - Beyond the 'boring' hydrogen atom}



\author[1]{Fred Jendrzejewski}%
\author[2]{Selim Jochim}%
\affil[1]{Kirchhoff-Institut für Physik}%
\affil[2]{Physikalisches Institut der Universität Heidelberg}%


\vspace{-1em}



  
  \date{January 04, 2025}


\begingroup
\let\center\flushleft
\let\endcenter\endflushleft
\maketitle
\endgroup





\selectlanguage{english}
\begin{abstract}
In this lecture we will use the hydrogen atom to study static perturbations in form of external magnetic fields  and relativistic effects, leading to the fine structure splitting.%
\end{abstract}%



\sloppy


We spend quite some time on the properties of the hydrogen atom in the previous lectures \cite{Jendrzejewski,atom}. However, we completely neglected any effects of quantum-electrodynamics and relativistic physics. In this lecture we will study, why this is a good approximation for the hydrogen atom and then investigate in a perturbative fashion the terms. Most importantly, we will introduce that coupling between the orbital angular momentum and the spin of the electron, which leads to the fine splitting.

\section{Perturbation theory}
Up to now have studied the hydrogen atom to find its eigensystem and then studied how it evolves under the presence of oscillating electric fields. This allowed us to understand in more detail the idea of eigenstates and then of time-dependent perturbation theory. However, one of the most important concepts that can be introduced very nicely on the hydrogen atom is stationnary perturbation theory in form of external magnetic fields or relativistic corrections. We will remind you of perturbation theory here and then apply it to some simple cases.

We can now simply write down the problem as:
\begin{eqnarray}
\left(\Hzero +\lambda \Wop\right)\ket{\psi_m} = E_m\ket{\psi_m}
\end{eqnarray}
$\lambda$ is a very small parameter and $\Hzero$ is describing the hydrogen atom  system. We will note the eigenvalues and eigenstates of this system as:
\begin{align}\label{Eq:EigsUnperturb}
\Hzero \ket{\varphi_n} = \epsilon_n \ket{\varphi_n}
\end{align}
While, we do not know the exact solution of $\ket{\psi_m}$ and the energy $E_m$, we decide to decompose them in the following expansion of the small parameter $\lambda$:
\begin{align}
\ket{\psi_m} &= \ket{\psi_m^{(0)}} + \lambda\ket{\psi_m^{(1)}}+\lambda^2\ket{\psi_m^{(2)}}+O(\lambda^3)\\
E_m &= E_m^{(0)} +\lambda E_m^{(1)} + \lambda^2 E_m^{(2)}+O(\lambda^3)\,
\end{align}
To zeroth order in $\lambda$ we obtain:
\begin{eqnarray}
\Hzero \ket{\psi_m^{(0)}} = E_m^{(0)}\ket{\psi_m^{(0)}}
\end{eqnarray}
So it is just the unperturbed system and we can identify:
\begin{eqnarray}
\ket{\psi_m^{(0)}} = \ket{\varphi_m}~~E_m^{(0)} = \epsilon_m
\end{eqnarray}
For the first order we have to solve
\begin{eqnarray}\label{Eq:FirstOrder}
(\Hzero-E_m^{(0)}) \ket{\psi_m^{(1)}} + (\Wop-E_m^{(1)})\ket{\psi_m^{(0)}}= 0\\
(\Hzero-\epsilon_m) \ket{\psi_m^{(1)}} + (\Wop-E_m^{(1)})\ket{\varphi_m}= 0
\end{eqnarray}
We can multiply the whole equation by $\bra{\varphi_m}$ from the right. As $\bra{\varphi_m}\Hzero = \epsilon_m\bra{\varphi_m}$, the first term cancels out. Hence, we obtain:
\begin{eqnarray}\label{Eq:PerturbFirstOrder}
\boxed{E_m^{(1)} = \bra{\varphi_m}\Wop\ket{\varphi_m}}
\end{eqnarray}
We now also need to obtain the correction to the eigenstate. For that, we put \eqref{Eq:PerturbFirstOrder} into \eqref{Eq:FirstOrder}:
\begin{eqnarray}
(\Hzero-\epsilon_m) \ket{\psi_m^{(1)}} + (\Wop\ket{\varphi_m}-\ket{\varphi_m}\bra{\varphi_m}\Wop\ket{\varphi_m})= 0
\end{eqnarray}
We can now multiply the whole equation by $\bra{\varphi_i}$ from the right and obtain:
\begin{eqnarray}
(\epsilon_i-\epsilon_m)\bra{\varphi_i}\ket{\psi_m^{(1)}}+\bra{\varphi_i}\Wop\ket{\varphi_m} &=& 0
\end{eqnarray}
By rewriting the above equation, this directly gives us the decompositon of the $\ket{\psi_m^{(1)}}$ onto the original eigenstates and have:
\begin{eqnarray}\label{Eq:FirstOrderState}
\boxed{\ket{\psi_m^{(1)}} = \sum_{i\neq m} \frac{\bra{\varphi_i}\Wop\ket{\varphi_m}}{(\epsilon_m-\epsilon_i)}\ket{\varphi_i}}
\end{eqnarray}
And we end the calculation with second order pertubation in $\lambda$
\begin{eqnarray}
(\Hzero-E_m^{(0)}) \ket{\psi_m^{(2)}} + (\Wop-E_m^{(1)})\ket{\psi_m^{(1)}}-E_m^{(2)} \ket{\psi_m^{(0)}}= 0\\
(\Hzero-\epsilon_m) \ket{\psi_m^{(2)}} + (\Wop-E_m^{(1)})\ket{\psi_m^{(1)}}-E_m^{(2)} \ket{\varphi_m}= 0\\
\end{eqnarray}
We can multiply once again whole equation by $\bra{\varphi_m}$ from the right, which directly drops the first term. The term $E_m^{(1)}\bra{\varphi_m}\ket{\psi_m^{(1)}}$ drops out as the first order perturbation does not contain a projection onto the initial state. So we can write:
\begin{eqnarray}
E_m^{(2)}= \bra{\varphi_m}\Wop\ket{\psi_m^{(1)}}
\end{eqnarray}
Plugging in our solution \eqref{Eq:FirstOrderState}, we obtain:
\begin{equation}\label{Eq:PerturbSecOrder}
\boxed{E_m^{(2)} = \sum_{i\neq m} \frac{|\bra{\varphi_i}\Wop\ket{\varphi_m}|^2}{(\epsilon_m-\epsilon_i)}}
\end{equation}

\section{Static external magnetic fields}

A first beautiful application of perturbation theory is the study of static magnetic fields (see Ch 1.9 and Ch. 2.7.1 of \cite{Hertel_2015} for more details). 
The motion of the electron around the nucleus implies a magnetic current
\begin{align}
I = \frac{e}{t} = \frac{ev}{2\pi r}
\end{align}
and this implies a magnetic moment $M = I A$, with the enclosed surface $A=\pi r^2$. It may be rewritten as:
\begin{align}
\vec{M}_L &= -\frac{e}{2m_e}\vec{L} &=-\frac{\mu_B}{\hbar} \vec{L} \\
\mu_B &= \frac{\hbar e}{2m_e}
\end{align}
where $\mu_B$ is the \textbf{Bohr magneton}. Its potential energy in a magnetic field $\vec{B} = B_0 \vec{e}_z$ is then:
\begin{align}
V_B &= -\vec{M}_L\cdot \vec{B}\\
&= \frac{\mu_B}{\hbar} L_z B_0
\end{align}
Its contribution is directly evaluated from Eq. \eqref{Eq:PerturbFirstOrder} to be:
\begin{align}
E_{Zeeman} = \mu_B m B_0
\end{align}
This is the Zeeman splitting of the different magnetic substates. It is visualized in Fig. \ref{982283}.\selectlanguage{english}
\begin{figure}[h!]
\begin{center}
\includegraphics[width=0.70\columnwidth]{figures/Bildschirmfoto-2018-10-31-um-08-07-29/Bildschirmfoto-2018-10-31-um-08-07-29}
\caption{{The Zeeman effect in the hydrogen atom.
{\label{982283}}%
}}
\end{center}
\end{figure}




\section{Trapping with electric or magnetic fields}
We have now investigated the structure of the hydrogen atom and seen how its energy gets shifted in external magnetic fields. We  can combine this understanding to study conservative traps for atoms and ions. Neutral atoms experience the external field:
\begin{align}
E_{mag}(x,y) = \mu_B m B_0(x,y)
\end{align}
For ions on the other hand we have fully charged particles. So they simply experience the external electric field directly:
\begin{align}
E_{el}(x,y) = -q E(x,y)
\end{align}

Trapping atoms and ions has to be done under very good vacuum such that they are well isolate from the enviromnent and high precision experiments can be performed. 

However, the trap construction is not trivial given Maxwells equation $\text{div} \vec{E} = 0$ and $\text{div} \vec{B} = 0$. So, the experimentalists have to play some tricks with oscillating fields. We will not derive in detail how a resulting \textbf{Paul trap} works, but the  \href{https://youtu.be/Xb-zpM0UOzk}{linked video} gives a very nice impression of the idea behind it. A sketch is presented in Fig. \ref{149591}.\selectlanguage{english}
\begin{figure}[h!]
\begin{center}
\includegraphics[width=0.70\columnwidth]{figures/IonTraps-01/IonTraps-01}
\caption{{The upper stage shows the phases of The two phases of the oscillating
electric field of a Paul trap. Taken
from~\href{https://en.wikipedia.org/wiki/Quadrupole_ion_trap}{wikipedia}.
Below we can see a linear ion (Paul) trap containing six calcium 40
ions. Taken
from~\href{https://quantumoptics.at/en/research/lintrap.html}{here} .
{\label{149591}}%
}}
\end{center}
\end{figure}


This work on trapping ions dates back to the middle of the last century (!!!) and was recognized by the\href{https://www.nobelprize.org/prizes/physics/1989/summary/}{ Nobel prize in 1989} for Wolfgang Paul \cite{Paul_1990} and Hans Dehmelt \cite{Dehmelt_1990}. They shared the prize with Norman Ramsey, who developped extremely precise spectroscopic methods, now known as Ramsey spectroscopy \cite{Ramsey_1990}. 

For atoms we can play similiar games with magnetic traps. Again we have to solve the problem of the zero magnetic fields. Widely used configurations are the Ioffe-Pritchard trap, where quadrupole fields are superposed with a bias field \cite{Pritchard_1983}, or TOP-traps \cite{Petrich_1995}.

Ion traps are now the basis of ionic quantum computers \cite{ions}  and magnetic traps paved the way for quantum simulators with cold atoms \cite{Jendrzejewskia}. 



\section{What we missed from the Dirac equation}
Until now we have completely neglected relativistic effects, i.e. we should have really solved the Dirac equation instead of the Schr\selectlanguage{ngerman}ödinger equation. However, this is is major task, which we will not undertake here. But what were the main approximations ?
\begin{enumerate}
\item We neglected the existance of the electron spin.
\item We did not take into account the relativistic effects.
\end{enumerate}

So, how does relativity affect the hydrogen spectrum? In a first step, we should actually introduce the magnetic moment of the spin:
\begin{align}
\vec{M}_S = -g_e \mu_B \frac{\vec{S}}{\hbar}
\end{align}
The spin of the electron is $1/2$, making it a fermion and the \textit{g factor of the electron} reads 
\begin{align}
g_e \approx 2.0023
\end{align}
Further discussions of the g-factor might be found in Chapter 6.6 of \cite{Hertel_2015}.

\subsection{Amplitude of the relativistic effects}

We saw in lecture 5 \cite{Jendrzejewski} and 6 \cite{Jendrzejewskib}, that the energy levels of hydrogenlike atoms are given by:
\begin{align}\label{Eq:EnergyHydrogen}
E_n &= \frac{Z^2 R_{y,\infty}}{n^2}\\
R_{y,\infty} &= \frac{m_e e^4}{32 \pi^2 \epsilon_0^2 \hbar^2}
\end{align}
We can now use the fine-structure constant, which measures the coupling strength of the electric charges to the electromagnetic field:
\begin{align}\label{Eq:FS}
\alpha &= \frac{e^2}{4\pi\epsilon_0\hbar c}\\
&= \frac{1}{137.035999139(31)}
\end{align}
We can now rewrite Eq. \eqref{Eq:EnergyHydrogen} as:
\begin{align}
E_n = \frac{1}{2} \underbrace{m_e c^2}_{\text{rest mass energy}} Z^2 \alpha^2 \frac{1}{n^2}
\end{align}
Here, $m_e c^2\approx \SI{511}{\kilo eV}$ is the rest mass energy of the electron. $E_n \approx \SI{10}{eV}$ on the other hand is the energy of the bound state and therefore in the order of the kinetic energy of the electron. As long as it is much smaller than the rest-mass of the electron, we can neglect the relativistic effects. A few observations:

\begin{itemize}
\item Relativistic effects are most pronounced for deeply bound states of small quantum number $n$.
\item Relativistic effects effects will become important once $(Z\alpha)\approx 1$, so they will play a major role in heavy nuclei. 
\end{itemize}

For the hydrogen atom we can thus treat the relativistic effects in a perturbative approach.But the most important consequence of the relativistic terms is actually the existance of the electron spin.

\subsection{The relativistic mass and Darwin term}


\begin{enumerate}
\item ``Relativistic mass'':
The relativistic relation between energy and momentum reads:
\begin{align}
E_\text{rel} &= \sqrt{(mc^2)^2+(\vec{p}c)^2}\\
&\approx mc^2 + \frac{p^2}{2m}- \frac{\vec{p}^{\,4}}{8m^3c^2} + \cdots
\end{align}
The first two terms of the expansion are the nonrelativistic limit and the third term is the first correction. Therefore, the corresponding Hamiltonian is:
\begin{align}
\hat{H}_\text{rm} = - \frac{\hat{\vec{p}}^{\,4}}{8m^3c^2}.
\end{align}

\item Darwin term:
If $r=0$, the potential $V(r)$ diverges to $-\infty$. We get:
\begin{align}
\hat{H}_\text{Darwin} = \frac{\pi \hbar^2}{2m^2c^2}\left( \frac{Ze^2}{4\pi\epsilon_0}\right) \delta(\hat{\vec{r}})
\end{align}

\end{enumerate}

If we perform a first correction to the energy of the eigenstates $\braket{n,l,m}$ by calculating
%
\begin{align}
\braket{n,l,m|\hat{H}'|n,l,m},
\end{align}
%
we find that it works perfectly for case (1) and (2) which is due to degeneracy.
$\hat{H}_\text{rm}$ and $\hat{H}_\text{Darwin}$ commute with all observables forming the complete set of commuting observables (CSCO) for $\hat{H}_0$
\begin{align}
\hat{H}_0,\hat{\vec{L}}^2, \hat{L}_z,
\end{align}
%
with states described by $\ket{n,l,m}$.

\selectlanguage{english}
\FloatBarrier
\bibliographystyle{plainnat}
\bibliography{bibliography/converted_to_latex.bib%
}

\end{document}

