\documentclass[10pt]{article}

\usepackage{fullpage}
\usepackage{setspace}
\usepackage{parskip}
\usepackage{titlesec}
\usepackage[section]{placeins}
\usepackage{xcolor}
\usepackage{breakcites}
\usepackage{lineno}
\usepackage{hyphenat}





\PassOptionsToPackage{hyphens}{url}
\usepackage[colorlinks = true,
            linkcolor = blue,
            urlcolor  = blue,
            citecolor = blue,
            anchorcolor = blue]{hyperref}
\usepackage{etoolbox}
\makeatletter
\patchcmd\@combinedblfloats{\box\@outputbox}{\unvbox\@outputbox}{}{%
  \errmessage{\noexpand\@combinedblfloats could not be patched}%
}%
\makeatother


\usepackage[round]{natbib}
\let\cite\citep




\renewenvironment{abstract}
  {{\bfseries\noindent{\abstractname}\par\nobreak}\footnotesize}
  {\bigskip}

\titlespacing{\section}{0pt}{*3}{*1}
\titlespacing{\subsection}{0pt}{*2}{*0.5}
\titlespacing{\subsubsection}{0pt}{*1.5}{0pt}


\usepackage{authblk}


\usepackage{graphicx}
\usepackage[space]{grffile}
\usepackage{latexsym}
\usepackage{textcomp}
\usepackage{longtable}
\usepackage{tabulary}
\usepackage{booktabs,array,multirow}
\usepackage{amsfonts,amsmath,amssymb}
\providecommand\citet{\cite}
\providecommand\citep{\cite}
\providecommand\citealt{\cite}
% You can conditionalize code for latexml or normal latex using this.
\newif\iflatexml\latexmlfalse
\AtBeginDocument{\DeclareGraphicsExtensions{.pdf,.PDF,.eps,.EPS,.png,.PNG,.tif,.TIF,.jpg,.JPG,.jpeg,.JPEG}}

\usepackage[utf8]{inputenc}
\usepackage[ngerman,english]{babel}










\begin{document}

\title{Lecture 11 -  Laser fundamentals}



\author[1]{Fred Jendrzejewski}%
\affil[1]{Kirchhoff-Institut für Physik}%


\vspace{-1em}



  
  \date{January 07, 2025}


\begingroup
\let\center\flushleft
\let\endcenter\endflushleft
\maketitle
\endgroup





\selectlanguage{english}
\begin{abstract}
We will study some basic properties of the laser.%
\end{abstract}%



\sloppy


In the last lecture \cite{Jendrzejewski} we studied the interaction of atoms and light. Most importantly, we saw that electric field can be dephased and absorped through the interaction with atomic gases.  In this lecture, we will see how this interaction can be employed to induce lasing and then study some basic properties of the laser. In the laser would most importantly find a situation in which the light coming out of the dielectric medium is amplified as shown in Fig. \ref{370550}.\selectlanguage{english}
\begin{figure}[h!]
\begin{center}
\includegraphics[width=0.70\columnwidth]{figures/Bildschirmfoto-2018-11-13-um-15-43-50/Bildschirmfoto-2018-11-19-um-07.48.06}
\caption{{The Laser set-up.
{\label{370550}}%
}}
\end{center}
\end{figure}

The crucial idea of the system is that it is possible to find a configuration for the medium in which it has a certain gain for the propagation of the electric field. So if the light comes into the gain medium with amplitude $I_A$ it exits with amplitude $G^{(0)}I_A$. The output mirror and losses lower the intensity such that the intensity. We can describe losses by the aborption rate $A$. The transmission of the semi-transparent mirror is quantified by the transmission coefficient $T$. Putting it all together, the intensity just before reentering the medium reads then $G^{(0)}I_A(1-T)(1-A)$. Amplification will happen if this intensity is higher than the initial one:
\begin{align}
G^{(0)}>\frac{1}{(1-T)(1-A)}
\end{align}


We will see that such a configuration is not trivial at all on the two-level system. Then we will discuss the appropiate configuration for lasing and a few properties of the laser.

\section{The two level system}
In the last lecture, we saw that the two-level system might be described by the following rate equations:
\begin{align}
\dot{N}_g &= i\Omega(\sigma_{eg}-\sigma_{ge})+\Gamma N_e\\
\dot{\sigma}_{ge}&= -i\delta \sigma_{ge}-i (N_g-N_e)\Omega-\frac{\Gamma}{2}\sigma_{ge}
\end{align}
The definitions were:
\begin{itemize}
\item $\Omega$ is the strength of the Rabi coupling.
\item $\delta$ the detuning.
\item $\Gamma$ is the lifetime of the excited state.
\item $N_g$ the groundstate population.
\item And $\sigma_{g,e}$ is the coherence $\sigma_{g,e} = \gamma_g^* \gamma_e$.
\item $\Gamma_2$ is the lifetime of the coherence. For the moment we will work in the limit $\Gamma_2 = \frac{\Gamma}{2}$. Let we will relax this point a bit.
\end{itemize}
At this stage we can find the steady-state solutions by setting $\dot{N}_g = \dot{\sigma}_{ge} = 0$. This leads too:
\begin{align}\label{Eq:PopTwoLevel}
N_e &= \frac{1}{2}\frac{\Omega^2}{2\delta^2+\Gamma^2/2+\Omega^2}\end{align}
\begin{align}\label{Eq:CohTwoLevel}
\sigma_{ge} &= i\frac{\Omega}{2}\frac{\Gamma-i2\delta}{2\delta^2+\Gamma^2/2+\Omega^2}
\end{align}

We will now discuss these results in the two important regimes of very weak and very strong coupling. The first one is important for probe experiments, while the second one is typically the one, where we would like to operate a laser.

\subsection{Linear response}
For very small coupling strength we can neglect the $\Omega$ dependence in the coherence and we obtain:
\begin{align}
\sigma_{ge} &= \frac{\Omega}{2}\frac{\delta+i\Gamma/2}{\Gamma^2/4+\delta^2}
\end{align}

We can now plug this into the dipole element \footnote{The sign change appears as we are now working with $\sigma_{eg}$ instead of $\sigma_{ge}$ }:
\begin{align}
D &= d \sigma_{eg}\\
D &= \frac{d^2}{2}\frac{\delta-i\Gamma/2}{\Gamma^2/4+\delta^2}E
\end{align}
In the second line, we used the relationship $\Omega = dE$ from lecture 4 \cite{Jendrzejewskia}. We can then directly read of the polarizability and hence the susceptibility, which was defined through $D = \alpha E$:
\begin{align}
\alpha =\frac{d^2}{2}\frac{\delta-i\Gamma/2}{\Gamma^2/4+\delta^2}
\end{align}
On resonance $\delta=0$ we have:
\begin{align}
\alpha(0) =-i\frac{d^2}{\Gamma}\\
\chi(0) = -i\frac{n}{\epsilon_0}\frac{d^2}{\Gamma}
\end{align}
We obtain now rather directly that:
\begin{itemize}
\item Absorption is maximal on resonance. 
\item There is no dephasing on resonance. 
\item For large detunings the absorption can be increasingly neglected and the media becomes refractive as it only keeps an optical index.
\end{itemize}
We can then also look for maximum dephasing and find the it happens close to the resonance. A summary can be found in Fig. \ref{715970}.\selectlanguage{english}
\begin{figure}[h!]
\begin{center}
\includegraphics[width=0.70\columnwidth]{figures/Bildschirmfoto-2018-11-13-um-14-12-29/Bildschirmfoto-2018-11-13-um-14-12-29}
\caption{{Susceptibility of the two level system. Figure taken from
\protect\cite{grynberg}
{\label{715970}}%
}}
\end{center}
\end{figure}

\subsection{Saturation}
The population and coherence in the two-level system will saturate as a substantial amount of atoms is excited. To simplify the discussion let us rewrite the eq. \eqref{Eq:PopTwoLevel} and \eqref{Eq:CohTwoLevel} on resonance:
\begin{align}
N_e &= \frac{1}{2}\frac{\Omega^2}{\Gamma^2/2+\Omega^2}\\
\sigma_{ge} &= i\frac{\Omega}{2}\frac{\Gamma}{\Gamma^2/2+\Omega^2}
\end{align}
We can see that the excited fraction is limited to $1/2$ of the atoms. So at some point the system cannot react anymore to the additional coupling strength. This is also the reason for the decay of the coherence at very large $\Omega$, which is basically telling us that the medium becomes transparent. This kind of observations motivates for practical purposes to introduce the saturation intensity $I_{sat} \propto \frac{\Gamma^2}{2}$, which allows us to rewrite the previous to equations as:
\begin{align}
N_e &= \frac{1}{2}\frac{I/I_{sat}}{1+I/I_{sat}}
\end{align}

Finally, this kind of expressions allows us also nicely to see the direct connection between the dipole element and the number of excited atoms:
\begin{align}
\sigma_{ge} = i\frac{N_e}{\Omega}
\end{align}
So we never obtain a situation, where the two-level system amplifies light. Let us look into this situation once again from a technical point of view to see if we can obtain situations of amplificiation.

\section{Rate equations}
To understand laser it is best to formulate the interaction of atoms and light in terms of rate equations for the populations, assuming that the coherences follow adiabatically. While strict derivations can become very tedious, they can be written down in a phenomenological way rather easily. So we will convince us here in some limiting cases of the usefulness.

We now would like to use the previous discussions to set up the necessary formalism for laser amplification, which is based on the idea of rate equations. While these equations are of phenomenological nature, we can convince ourselves of their soundness in a first step.

\subsection{The two-level system}

In the last lecture we saw that we can write down the following Bloch equations for the two-level system on resonance:
\begin{align}
\dot{N}_g &= -i\Omega(\sigma_{eg}-\sigma_{ge})+\Gamma N_e\\
\dot{\sigma}_{ge}&= i (N_g-N_e)\Omega-\Gamma_2\sigma_{ge}
\end{align}
In a substantial amount of situations the coherences reach the steady state much faster than the population. This can be due to technical noise, collisions or other effects. In this case we can assume $\dot{\sigma}_{ge}=0$ and the solve for the populations:
\begin{align}
\sigma_{ge}&= i \frac{\Omega}{\Gamma_2} (N_g-N_e)\\
\sigma_{eg}-\sigma_{ge} &= -2i \frac{\Omega}{\Gamma_2} (N_g-N_e)
\end{align}
We can see in this limit that the inversion of the sign of the dipole element would come with $N_g < N_e$ \textbf{in the steady-state}. We saw previously that we cannot achieve this limit in the two-level system.


Having eliminated adiabatically the coherences, we end up with the following time evolution of the population:
\begin{align}
\dot{N}_g &= 2\frac{\Omega^2}{\Gamma_2}(N_e-N_g)+\Gamma N_e\\
\end{align}
We can now identify the three terms as:
\begin{itemize}
\item Stimulated emission.
\item Stimulated absorption.
\item Spontaneous emission.
\end{itemize}
We will then focus on the rate equation for the populations and attempt to find situations, where $N_g < N_e$.

\subsection{Optional: Rate equations for light}
To see, when light amplification will happen, we now need to connect the rate equations to evolution of light within the medium. One heuristic approach is guided by the Beer-Lambert law (Ch. 4.2.3 of \cite{Hertel_2015}).
\begin{align}
\frac{dI}{dz}&=- N_g\sigma I
\end{align}
We would like to translate this now more precisely into a change of photon numbers through the relation:
\begin{align}
I = c \hbar\omega N_{ph}
\end{align}
This allows us to rewrite for a propagation wave with $z = ct $:
\begin{align}
- \frac{N_g\sigma I}{\hbar \omega} = \frac{d}{dt}N_{ph} = \frac{dN_g}{dt} = -\frac{dN_e}{dt}
\end{align}
This allows us to define an absorption rate:
\begin{align}
R_{ab} &=  \frac{\sigma I}{\hbar \omega}\\
&= \frac{1}{N_g}\frac{dN_g}{dt}
\end{align}
This process is just the description of \textbf{stimulated absorption}. Nothing was special about the discussion of the absorption and we can actually also have exactly the inverse situation, where we assume that all the atoms start out in the excited state and then lead to an increased intensity:
\begin{align}
\frac{dI}{dz}&= N_e\sigma I
\end{align}
Both processes are now computing and we obtain in general:
\begin{align}
\frac{dI}{dz}&= (N_e-N_g)\sigma I
\end{align}

So if we have most atoms in the excited state and neglect the atoms in the ground state we can actually have light amplification. This is idea is underlying the laser. However, we have already seen for the two-level system that this situation is not easily achieved and we will now discuss it a bit further.




\section{Lasing condition in a four-level system}
To obtain in the medium, it is necessary to have an excited state population which is higher than the population of the ground state. This is not possible in the two-level system and in practice realized mostly in four-level systems.\selectlanguage{english}
\begin{figure}[h!]
\begin{center}
\includegraphics[width=0.70\columnwidth]{figures/Bildschirmfoto-2018-11-14-um-08-43-49/Bildschirmfoto-2018-11-19-um-07.55.23}
\caption{{The 4 level system.
{\label{804096}}%
}}
\end{center}
\end{figure}


The idea of such a system is the following:
\begin{itemize}
\item A strong pumping beam excites atoms into the state $e$.
\item From $e$ they decay rapidely into the state b. This is the upper state for the lasing transition $b\rightarrow a$.
\item We obtain lasing on the transition $b\rightarrow a$, which has a decay rate of $\tau_B$
\item The lower state $a$ is easily depopulated through the fast relaxation $a\rightarrow g$.
\end{itemize}
We can now write down the rate equations for a weak laser, such that we can only keep terms in first order:
\begin{align}
\dot{N}_e &= w(N_g-N_e)-N_e/\tau_e\\
\dot{N}_b &= \frac{N_e}{\tau_e}-N_b/\tau_b\\ 
\dot{N}_a &= \frac{N_b}{\tau_b}-N_a/\tau_a\\
\dot{N}_g &= \frac{N_a}{\tau_a}-w(N_g-N_e)
\end{align}
We can now find steady state solutions assuming that $\tau_e, \tau_a \ll \tau_b$. Further we assume that the pumping to $e$ is not too strong, i.e. $w \tau_e\ll1$. We then obtain:
\begin{align}
N_e \simeq w\tau_e N_g\\
N_b \simeq w\tau_b N_g\\
N_a \simeq w\tau_a N_g
\end{align}
We then obtain the state populations:
\begin{align}
\frac{N_b-N_a}{N_g + N_e + N_a + N_b}\simeq \frac{w\tau_b}{1+w\tau_b}
\end{align}

Lasing is then obtained above the pumping threshold at which the gain overcomes the losses.

\section{Steady-state operation of the laser}
Assuming that the lasing condition is fullfilled, we can now investigate its steady-state behavior. Quite importantly, we have to have an electric field, which remains exactly constant after each round trip.This implies to conditions:
\begin{itemize}
\item The gain has to cancel the losses:
\begin{align}
G = \frac{1}{(1-T)(1-A)}
\end{align}
\item The phase after the round trip has to be a multiple of $2\pi$
\end{itemize}
For a cavity of length $L$, the wavelength  $\lambda_p$ has to be an integer fraction:
\begin{align}
L_{cav} &= p\lambda_p\text{ with }p \in  \mathbb{N}\\
\omega_p/2\pi &=p \frac{c}{L} 
\end{align}
So the lasing will not happen at one single frequency, but actually for any wavelength fulfilling this condition. The laser has multiple \textbf{longitudinal modes}. Some tricks allow to suppress this multi-mode behavior, such that we obtain a very pure light source.\selectlanguage{english}
\begin{figure}[h!]
\begin{center}
\includegraphics[width=0.70\columnwidth]{figures/Bildschirmfoto-2018-11-19-um-08-08-56/Bildschirmfoto-2019-01-29-um-12.59.41}
\caption{{Frequency distribution of the steady-state laser.
{\label{659829}}%
}}
\end{center}
\end{figure}

\section{Mode-locked operation}

We have seen in the last section that a laser might act in the multimode regime. So let us write down for simplicity the total field, where we assume that the relative phase between modes in uncorrelated and that the amplitude is the same for all of them. We then have:
\begin{align}
E(t) = \sum_{k=0}^{N-1}E_0 \cos(\omega_k t +\phi_k)
\end{align}
The frequency of each mode:
\begin{align}
\omega_k = \omega_0 + k \Delta\text{ with }\Delta/2\pi = \frac{c}{L_{cav}}
\end{align}
Summing the electric fields leads to an intensity:
\begin{align}
I(t)= \frac{NE_0^2}{2}+E_0^2\sum_{j_k}\cos\left[(\omega_j-\omega_k)t+\phi_j-\phi_k\right]
\end{align}

For uncorrelated fields this intensity is on average $\overline{I} = \frac{NE_0^2}{2}$ with temporal fluctuations in the order of the amplitude itself \footnote{We obtain actually a speckle pattern in time here.}
For correlated fields the equations simplify a lot and we obtain:
\begin{align}
I &= \frac{1}{2}\left|\sum_{k=0}^{N-1}E_0e^{-i\omega_k t}\right|^2\\
 &= \frac{E_0^2}{2}\left|\sum_{k=0}^{N-1}e^{i k\Delta  t}\right|^2\\
 &= \frac{E_0^2}{2}\left|\frac{\sin(\frac{N\Delta t}{2})}{\sin(\frac{\Delta t}{2})}\right|^2
\end{align}
The maximum intensity in this coherent sum is now
\begin{align}
I_{max} = N\overline{I}
\end{align}
We can then reach petawatt (!!) peak powers. 

In the next lecture \cite{entanglement}, we will study how the laser is actually used for the study of entanglement

\selectlanguage{english}
\FloatBarrier
\bibliographystyle{plainnat}
\bibliography{bibliography/converted_to_latex.bib%
}

\end{document}

