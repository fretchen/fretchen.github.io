\documentclass[10pt]{article}

\usepackage{fullpage}
\usepackage{setspace}
\usepackage{parskip}
\usepackage{titlesec}
\usepackage[section]{placeins}
\usepackage{xcolor}
\usepackage{breakcites}
\usepackage{lineno}
\usepackage{hyphenat}





\PassOptionsToPackage{hyphens}{url}
\usepackage[colorlinks = true,
            linkcolor = blue,
            urlcolor  = blue,
            citecolor = blue,
            anchorcolor = blue]{hyperref}
\usepackage{etoolbox}
\makeatletter
\patchcmd\@combinedblfloats{\box\@outputbox}{\unvbox\@outputbox}{}{%
  \errmessage{\noexpand\@combinedblfloats could not be patched}%
}%
\makeatother


\usepackage[round]{natbib}
\let\cite\citep




\renewenvironment{abstract}
  {{\bfseries\noindent{\abstractname}\par\nobreak}\footnotesize}
  {\bigskip}

\titlespacing{\section}{0pt}{*3}{*1}
\titlespacing{\subsection}{0pt}{*2}{*0.5}
\titlespacing{\subsubsection}{0pt}{*1.5}{0pt}


\usepackage{authblk}


\usepackage{graphicx}
\usepackage[space]{grffile}
\usepackage{latexsym}
\usepackage{textcomp}
\usepackage{longtable}
\usepackage{tabulary}
\usepackage{booktabs,array,multirow}
\usepackage{amsfonts,amsmath,amssymb}
\providecommand\citet{\cite}
\providecommand\citep{\cite}
\providecommand\citealt{\cite}
% You can conditionalize code for latexml or normal latex using this.
\newif\iflatexml\latexmlfalse
\AtBeginDocument{\DeclareGraphicsExtensions{.pdf,.PDF,.eps,.EPS,.png,.PNG,.tif,.TIF,.jpg,.JPG,.jpeg,.JPEG}}

\usepackage[utf8]{inputenc}
\usepackage[ngerman,english]{babel}








\usepackage{siunitx}
\usepackage{amsmath}
\newcommand{\bra}[1]{\ensuremath{\left\langle#1\right|}}
\newcommand{\ket}[1]{\ensuremath{\left|#1\right\rangle}}
\newcommand{\braket}[1]{\ensuremath{\left\langle#1\right\rangle}}
\newcommand{\rhohat}{\hat{\rho}}
\newcommand{\tr}[1]{\mathrm{tr}(#1)}
\newcommand{\trarb}[2]{\mathrm{tr}_{#1}(#2)}
\newcommand{\vv}[1]{\mathbf{#1}}
\newcommand*\dif{\mathop{}\!\mathrm{d}}
\newcommand{\eexp}[1]{\mathrm{e}^{#1}}
\newcommand*\ch[1]{\ensuremath{\mathrm{#1}}}

\begin{document}

\title{Lecture 14 - The spin in atoms with many electrons}



\author[1]{Fred Jendrzejewski}%
\author[2]{Selim Jochim}%
\author[2]{Matthias Weidemüller}%
\affil[1]{Kirchhoff-Institut für Physik}%
\affil[2]{Physikalisches Institut der Universität Heidelberg}%


\vspace{-1em}



  
  \date{January 07, 2025}


\begingroup
\let\center\flushleft
\let\endcenter\endflushleft
\maketitle
\endgroup








\sloppy


In the last lecture \cite{electrons} we have seen that we can typically treat complex atomic systems within the central field approximation. 
\begin{align} \label{eq:lsvsjjhamiltonian}
\hat{H} &&=& \underbrace{\sum_i^N \left( \frac{1}{2} \vec{\nabla}^2_{\vec{r}_i} + V_\textrm{cf} (r_i) \right)}_{\hat{H}_0} + \underbrace{\sum^N_{j>i} \left( \frac{1}{r_{ij}} - S(r_i) \right)}_{\hat{H}_1}
\end{align}

So we we can can treat atoms through the shell structure known from the \ch{H} atom, but the screening lifts the $l$ degeneracy. For a single outer electron, we have even seen how this screening can be described by the quantum defect.

We would now like to go beyond this simple picture and discuss the following questions:
\begin{itemize}
\item How should the residual term $\hat{H}_1$ be taken into account?
\item How do we properly take into account the Pauli principle ?
\item How can we treat the fine-splitting ?
\end{itemize}

\section{On the residual coupling}
If we ignore the residual coupling, we obtain a spherically symmetric problem, which implies that the angular momentum $\vec{l}_i$ of each electron is conserved. This conservation will be broken by $\hat{H}_{1}$. However, these forces are internal, which implies that the total angular momentum $\vec{L} = \sum_i \vec{l}_i$ is conserved. So we should label the states in the complex Hamiltonian by $\vec{L}$.

The total angular momentum will then set the symmetry of the spatial wavefunction. As already discussed in some detail for the He atom, this has wide-reaching consquence on the spin degree of freedom through exchange interaction. 

\section{The Pauli principle and spin}
\begin{itemize}
\item According to the Pauli principle, each single-particle state can be occupied only by one electron. After distributing all electrons over different single-particle eigenstates (``orbitals''), the resulting state needs to be fully antisymmetrized (Slater determinant).
\item There is a simplification for atoms with many electrons: The angular momenta and spins of a complete subshell  with $n,\,l,\,\{m_{-l},\cdots,m_l\}$ add to zero and can be ignored in the further considerations (``shell structure''). Note that this is often broken in molecular binding!
\item Alkali atoms are the simplest atoms with shell structure: All but one \emph{valence} electron add to $L=0,S=0$. The ground state thus has $L=0,S=1/2$.
\item For more complex atoms, the valence electrons couple to a total orbital angular momentum $L$ with a given symmetry according to particle exchange.
\end{itemize}

Let us have a look at two examples for light atoms, starting with \ch{He}:
\begin{itemize}
\item $1s^2 \rightarrow L=0,S=0$. The corresponding term is $^1S$
\item $1s2s \rightarrow L=0, \{S=0,S=1\}$. The corresponding terms are $^1S$ and $^3S$. .
\end{itemize}

The electronic configuration of \ch{Si} is:
\begin{align}
\underbrace{1s^2 2s^2 2p^6 3s^2}_{L=0,\,S=0} 3p^2
\end{align}
 Per valence electron we have $l=1$ and $s=1/2$. So we get $L=0,1,2$ and $S=0,1$. Here $S=1$ means symmetry and $S=0$ antisymmetry with respect to particle exchange.
 In principle we can form the following terms:
\begin{align}
^1S,\,^3S,\,^1P,\,^3P,\,^1D,\,^3D
\end{align}
Which of these terms can be fully antisymmetrized? Here, only the terms $^1S$, $^3P$ and $^1D$ fulfill Pauli's principle. In general the exchange interaction (seen in the discussion of He), will then lower the energy of the states with high spins.

\subsection{Optional: Symmetry of the $L$ states}
We can construct the following $L$-states for them:
\begin{align}
\ket{L=2,M_L=2} &= \ket{\overbrace{1}^{l_1},\overbrace{1}^{m_{l_1}};\overbrace{1}^{l_2},\overbrace{1}^{m_{l_2}}} \label{eq:lstate1}
\end{align}
\begin{align}
\ket{L=1,M_L=0} &= \frac{1}{\sqrt{2}} ( \ket{\overbrace{1}^{m_{l_1}},\overbrace{-1}^{m_{l_2}}} - \ket{-1,1} ) \label{eq:lstate2}
\end{align}
\begin{align}
\ket{L=0,M_L=0} &= \frac{1}{\sqrt{3}}(\ket{\overbrace{1}^{m_{l_1}},\overbrace{-1}^{m_{l_2}}} - \ket{0,0} + \ket{-1,1} ) \label{eq:lstate3}
\end{align}

The states \eqref{eq:lstate1} and \eqref{eq:lstate3} are symmetric and the state \eqref{eq:lstate2} is antisymmetric with respect to particle exchange. 


\section{Fine splitting}

We have already seen in the discussion of the hydrogen atom that relativistic effects should be taken into account to fully understand the level spectrum of different atoms. 
To take into account the spin, we can decompose the Hamiltonian as follows:
%
\begin{align} \label{eq:lsvsjjhamiltonian}
\hat{H} &&=& \underbrace{\sum_i^N \left( \frac{1}{2} \vec{\nabla}^2_{\vec{r}_i} + V_\textrm{cf} (r_i) \right)}_{\hat{H}_0} + \underbrace{\sum^N_{j>i} \left( \frac{1}{r_{ij}} - S(r_i) \right)}_{\hat{H}_1} + \underbrace{\sum_i^N c_i(\vec{r}_i) \hat{\vec{L}}_i \cdot \hat{\vec{S}}_i}_{\hat{H}_2}
\end{align}
The term $\hat{H}_0$ is from the central field and the independent particle model. The Hamiltonian $\hat{H}_1$ results from the residual electrostatic interaction and the Hamiltonian $\hat{H}_2$ comes from the spin-orbit coupling of individual electrons.

The question is now which term dominates. Since $\hat{H}_2 \propto (Z\alpha)^2$, we can ignore it if $Z$ is small. This is the case for light atoms.

For our example this means that $^3P$ is the lowest energy state. However, the triplet state has a multiplicity.  Which are these states? There has to be an additional degree of freedom. The total electronic angular momentum $\hat{\vec{J}} = \hat{\vec{L}} + \hat{\vec{S}}$ of the atom is a conserved quantity. The basis
\begin{align}
\ket{L,M_\textrm{l},S,M_\textrm{s}}
\end{align}
is therefore not the right basis, since $M_\textrm{l}$ and $M_\textrm{s}$ are not conserved because of $LS$ coupling. The correct basis is
\begin{align}
\ket{L,S,J,M_\textrm{j}}.
\end{align}
Which $J$ corresponds to the ground state?
 If the outermost shell is more than half filled, the maximum $J$ value is the ground state, otherwise, the ground state has the minimum $J$ value. In our example, \ch{Si}: $^3P_0$ is the ground state!


So far, we have been concerned with the ``Russell-Saunders'' coupling, also known as $LS$ coupling. However, for atoms with large $Z$, the term $\hat{H}_2$ in the Hamiltonian \eqref{eq:lsvsjjhamiltonian}
%and therefore the $jj$ coupling
might become large, since it is proportional to $(Z \alpha)^2$. We can ignore $\hat{H}_1$ instead.

According to $\hat{H}_2$ the individual electron orbital angular momentum $l$ pairs with the spin $s$ of the same electron to form the total electron angular momentum $j$. This will lead us to the concept of $jj$ coupling. Let us consider carbon first (see \ref{693186}). The ground state reads:
\begin{align}
\ch{C}: 1s^22s^22p^2.
\end{align}
Let us consider one excited state:
\begin{align}
1s^22s^22p3s.
\end{align}\selectlanguage{english}
\begin{figure}[h!]
\begin{center}
\includegraphics[width=0.70\columnwidth]{figures/lsvsjj/lsvsjj}
\caption{{``The splitting of levels in the first excited~\textsuperscript{1}P
and~\textsuperscript{3}P terms of the carbon sequence.
As~\(Z\) increases, the two electrons change their
character from being in an~\(LS\) coupled state to a
\(jj\) coupled state.'' Taken from \protect\cite{pearson}.
{\label{693186}}%
}}
\end{center}
\end{figure}

According to $LS$ coupling the total angular momentum of the two valence electrons is $L=1$ and the total spin can be $S=0,1$. Thus, we should expect a singlet and a triplet. For heavy atoms we can observe two doublets instead of a singlet and a triplet. Because of the very strong $\hat{\vec{L}}_i \cdot \hat{\vec{S}}_i$ coupling, $\hat{\vec{L}}$ and $\hat{\vec{S}}$ are no longer conserved quantities. On the contrary, $\hat{\vec{J}}$ still is a conserved quantity. In very heavy atoms, $l_i$ and $s_i$ couple to $j_i$ for each electron $i$.
%
In our example we get
\begin{align*}
l_1=0, s_1=\frac{1}{2} \qquad \text{and} \qquad l_2=1, s_2=\frac{1}{2}
\end{align*}
for electron $1$ and $2$ and thus
\begin{align*}
j_1=\frac{1}{2} \qquad \text{and}\qquad j_2=\frac{1}{2},\frac{3}{2}.
\end{align*}
The individual $j_i$ couple to a total orbital angular momentum $J$ in the following manner:
\begin{align*}
J= \left\{ \begin{array}{ccc}0,1 & \text{for} & (j_1,j_2)=\left(\frac{1}{2},\frac{1}{2}\right)\\ && \\ 1,2 & \text{for} &(j_1,j_2) = \left(\frac{1}{2}, \frac{3}{2}\right) \end{array} \right.
\end{align*}

\section{Hyperfine splitting}

Until now we have investigated the structure of the atom depending on:
\begin{itemize}
\item The orbital angular momentum $\vec{L}$, which defines the shells.
\item The total electronic angular momentum $\vec{J} = \vec{L} + \vec{S}$, which defines the fine structure because of the spin-orbit coupling.
\end{itemize}

Further the nucleus has a spin angular momentum $\vec{I}$, which leads to a magnetic moment:
\begin{align}
\vec{\mu}_I&= g_I \mu_N \vec{I}
\end{align}
We have introduced the new constants:
\begin{itemize}
\item $g_I$, which is always in the order of one, but it changes due to the changing structure of the different nuclei.
\item The \textit{nuclear magneton} $\mu_N = \frac{e\hbar}{2m_p}= \frac{m_e}{m_p}\mu_B$.
\end{itemize}
From the values of the prefactors, we can immediatly deduce that the hyperfine structure will be roughly three orders of magnitude smaller than the fine structure. As with the spin-orbit coupling this nuclear spin will experience the magnetic field produced by the motion of the electrons $\vec{B}_{el}$ and we have:
\begin{align}
H_{hfs} &= -\vec{\mu}_I \cdot \vec{B}_{el}\\
&= A_{hfs}\vec{I}\vec{J}
\end{align}
This couples $\vec{I}$ and $\vec{J}$ and the full structure is given by total angular momentum:
\begin{align}
\vec{F} &= \vec{J} + \vec{I}
\end{align}
 
\subsection{Hydrogen}
In hydrogen the ground state has no angular momentum and the spin is simply $S=1/2$. The nucleus has the same, such that we have $I=1/2$ and the groundstate splits into the $F=0$ and the $F=1$ doublet. They are separated by an energy difference of $1.42$GHz. This corresponds to a transition wavelength of 21 cm, which is widely used in astronomy.
\subsection{Cesium clocks}
The definition of time is based the hyperfine levels of Cs. Cs has just one stable isotope Cs-133 with nuclear spin $I=7/2$. The ground state electron is in the the 6s state in the \textsuperscript{2}S\textsubscript{1/2} state. So the groundstate splits into the $F=3$ and $F=4$ manifold. They are separated by 9.4GHz and the clicking between these transitions is our definition of time. The second is actually defined as \cite{units}:
\begin{quote}
The second is the duration of 9 192 631 770 periods of the radiation corresponding to the transition between the two hyperfine levels of the ground state of the cesium 133 atom.
\end{quote}

All the other units have similiar definitions, except the embarrising kilogram. It is defined as:
\begin{quote}
The kilogram is the unit of mass; it is equal to the mass of the international prototype of the kilogram.
\end{quote}
This will change soon as the member states of \textit{Bureau International des Poids et des Mesures} decided to base the SI system on the measurement of fundamental constants instead of some prototypes. So, now the fundamental constants will have no errorbars left, as shown in Fig. \ref{134930}.\selectlanguage{english}
\begin{figure}[h!]
\begin{center}
\includegraphics[width=0.70\columnwidth]{figures/17pml015-si-constants-card/17pml015-si-constants-card}
\caption{{This wallet card displays the fundamental constants and other physical
values that will define a revised international system of units. Taken
from
\href{http://https://www.nist.gov/si-redefinition/turning-point-humanity-redefining-worlds-measurement-system}{the
Blog of NIST.}
{\label{134930}}%
}}
\end{center}
\end{figure}

\subsection{Bosonic and fermionic isotopes}
A particularly interesting problem is the influence of the nuclear spin on how the atoms talk to each other. For example we have two different stable isotope of Li, namely Li\textsuperscript{6} and Li\textsuperscript{7}, which only differ by one neutron in the nucleus. All the wavelengths for controlling the atoms are extremely similiar, however the hyperfine structure is different. For Li\textsuperscript{6} we have a nuclear spin of $I=1$ and for Li\textsuperscript{7} $I=3/2$. The resulting total angular momentum 
\begin{itemize}
\item $F=1/2$ in the ground state of Li\textsuperscript{6}
\item $F=1$ in the ground state of Li\textsuperscript{7}
\end{itemize}
So one is of them is an integer spin boson and the other one is a spin half fermion. One can nicely see these differences as one tries to put them onto each other for cold temperatures and tight traps as visualized in Fig.\selectlanguage{english}
\begin{figure}[h!]
\begin{center}
\includegraphics[width=0.70\columnwidth]{figures/Bildschirmfoto-2018-11-28-um-11-32-40/Bildschirmfoto-2018-11-28-um-11-32-40}
\caption{{Observation of Fermi pressure for Li\textsuperscript{6} as it getting
cooled down. Figure taken from \protect\cite{Truscott_2001}
{\label{280523}}%
}}
\end{center}
\end{figure}

\selectlanguage{english}
\FloatBarrier
\nocite{*}

\bibliographystyle{plainnat}
\bibliography{bibliography/converted_to_latex.bib%
}

\end{document}

