\documentclass[10pt]{article}

\usepackage{fullpage}
\usepackage{setspace}
\usepackage{parskip}
\usepackage{titlesec}
\usepackage[section]{placeins}
\usepackage{xcolor}
\usepackage{breakcites}
\usepackage{lineno}
\usepackage{hyphenat}





\PassOptionsToPackage{hyphens}{url}
\usepackage[colorlinks = true,
            linkcolor = blue,
            urlcolor  = blue,
            citecolor = blue,
            anchorcolor = blue]{hyperref}
\usepackage{etoolbox}
\makeatletter
\patchcmd\@combinedblfloats{\box\@outputbox}{\unvbox\@outputbox}{}{%
  \errmessage{\noexpand\@combinedblfloats could not be patched}%
}%
\makeatother


\usepackage[round]{natbib}
\let\cite\citep




\renewenvironment{abstract}
  {{\bfseries\noindent{\abstractname}\par\nobreak}\footnotesize}
  {\bigskip}

\titlespacing{\section}{0pt}{*3}{*1}
\titlespacing{\subsection}{0pt}{*2}{*0.5}
\titlespacing{\subsubsection}{0pt}{*1.5}{0pt}


\usepackage{authblk}


\usepackage{graphicx}
\usepackage[space]{grffile}
\usepackage{latexsym}
\usepackage{textcomp}
\usepackage{longtable}
\usepackage{tabulary}
\usepackage{booktabs,array,multirow}
\usepackage{amsfonts,amsmath,amssymb}
\providecommand\citet{\cite}
\providecommand\citep{\cite}
\providecommand\citealt{\cite}
% You can conditionalize code for latexml or normal latex using this.
\newif\iflatexml\latexmlfalse
\AtBeginDocument{\DeclareGraphicsExtensions{.pdf,.PDF,.eps,.EPS,.png,.PNG,.tif,.TIF,.jpg,.JPG,.jpeg,.JPEG}}

\usepackage[utf8]{inputenc}
\usepackage[ngerman,english]{babel}








\usepackage{siunitx}
\usepackage{amsmath}
\newcommand{\bra}[1]{\ensuremath{\left\langle#1\right|}}
\newcommand{\ket}[1]{\ensuremath{\left|#1\right\rangle}}
\newcommand{\braket}[1]{\ensuremath{\left\langle#1\right\rangle}}
\newcommand{\rhohat}{\hat{\rho}}
\newcommand{\tr}[1]{\mathrm{tr}(#1)}
\newcommand{\trarb}[2]{\mathrm{tr}_{#1}(#2)}
\newcommand{\vv}[1]{\mathbf{#1}}
\newcommand*\dif{\mathop{}\!\mathrm{d}}
\newcommand{\eexp}[1]{\mathrm{e}^{#1}}
\newcommand{\Hzero}{\hat{H}_0}
\newcommand{\Wop}{\hat{W}}

\begin{document}

\title{Lecture 18 - Quantization of the Electromagnetic field}



\author[1]{Fred Jendrzejewski}%
\author[2]{Selim Jochim}%
\author[2]{Matthias Weidemüller}%
\affil[1]{Kirchhoff-Institut für Physik}%
\affil[2]{Physikalisches Institut der Univiersität Heidelberg}%


\vspace{-1em}



  
  \date{January 07, 2025}


\begingroup
\let\center\flushleft
\let\endcenter\endflushleft
\maketitle
\endgroup





\selectlanguage{english}
\begin{abstract}
We are looking into the quantization of the electromagnetic field. How to find the conjugate momenta and how we can identify the photon as a quantized particle.%
\end{abstract}%



\sloppy


Until now we exclusively treated the atom, molecule etc in a quantum-mechanical way. The electromagnetic field was always treated classically. We will attempt to change that for various reasons:
\begin{itemize}
\item Spontaneous emission and the Lamb shift can only be understood within quantum electrodynamics.
\item Several experiments control the electromagnetic field at the single photon level, so we have to understand how this works.
\item It frankly very unsatifying to only quantize half the problem.
\end{itemize}

So before we start the endevour let us start out with the some reminders on the properties of radiation in classical electromagnetism.

\section{Maxwell's Equations and Vector Potential}

Electrodynamics is described by Maxwell's equations:
\begin{align}
\vec{\nabla} \cdot \vec{B} &= 0\\
\vec{\nabla} \times \vec{E} + \frac{\partial \vec{B}}{\partial t}&=0\\
\vec{\nabla} \cdot \vec{E} &= \frac{1}{\epsilon_0}\rho(\vec{r},t) \\
\vec{\nabla} \times \vec{B} &= \frac{1}{c^2} \frac{\partial \vec{E}}{\partial t}+\frac{1}{\epsilon_0 c^2}\vec{j}
\end{align}
These equations are the equations of motion for the electromagnetic field. The first two are the homogeneous Maxwell equations and we can directly solve them by setting:
\begin{align}
\vec{B} &= \vec{\nabla}\times\vec{A}\\
\vec{E}&= -\frac{\partial}{\partial t}A-\nabla \phi
\end{align}
Quite importantly, the choice of the potential $\vec{A}$  has an inherent gauge freedom as we can basically perform the \textit{local gauge transformation}:
\begin{align}\label{Eq:PotPolar}
\vec{A} &\rightarrow \vec{A} + \vec{\nabla} f (\vec{r},t)\\
\phi &\rightarrow \phi - \frac{\partial f}{\partial t}
\end{align}
So we can transform the electromagnetic field and the magnetic field, but the equations of motion remain the same. It is this local gauge freedom, which also allows us to have charge conservation. But it is also this gauge freedom, which will make the quantization rather technically involved. 

\subsection{Free radiation}
To simplify the problem, we will actually, only work on free electromagnetic radiation, which simplifies the remaining Maxwell equations too:
\begin{align}
\vec{\nabla} \cdot \vec{E} &= 0 \\
\vec{\nabla} \times \vec{B} &= \frac{1}{c^2} \frac{\partial\vec{E}}{\partial t}
\end{align}
They become very simple to within the \textbf{Coulomb gauge}, were we fix:
\begin{align}\label{Eq:CouGauge}
\nabla \cdot \vec{A} = 0
\end{align}
For the electric field we have:
\begin{align}
\vec{\nabla} \cdot \left(-\frac{\partial}{\partial t}A-\nabla \phi\right) &= 0 \\
-\Delta \phi &= 0
\end{align}
This is solved through the boring solution $\phi = 0$. For the magnetic field we obtain\footnote{Remember $\vec{\nabla} \times (\nabla \times \vec{A}) =\nabla(\nabla \vec{A})-\nabla^2 \vec{A}$}:
\begin{align}
\vec{\nabla} \times (\nabla \times \vec{A}) &= -\frac{1}{c^2} \frac{\partial^2\vec{A}}{\partial t^2}\\
\nabla^2 \vec{A}  -\frac{1}{c^2} \frac{\partial^2\vec{A}}{\partial t^2}&=0
\end{align}

\subsection{Solution in terms of plane waves}

We can directly solve the problem, by using the Fourier representation:
\begin{align}
\vec{A}(\vec{r}, t)&= \frac{1}{(2\pi)^{3/2}}\int d^3\vec{k}\vec{A}(\vec{k},t)e^{i\vec{k}\vec{r}}
\end{align}
We then find that the solutions fulfill the requirement:
\begin{align}
\left(\vec{k}^2 +\frac{1}{c^2} \frac{\partial^2}{\partial t^2}\right)\vec{A}(\vec{k},t)&=0
\end{align}

So the free radiation describes a travelling wave in the direction $\vec{k}$. The coulomb gauge \eqref{Eq:CouGauge} further tells us that the vector potential only has components, which are orthogonal to $\vec{k}$ as:
\begin{align}
\vec{k}\cdot\vec{A} = 0\\
\vec{A}(\vec{k}) = \vec{e}_1 A_1(\vec{k}) + \vec{e}_2 A_2(\vec{k})
\end{align}

Given that these equations of motion for the vector potential look a lot like the ones of a harmonic oscillator, let us go through the quantization of the harmonic oscillator first.

\section{Quantization procedure for the harmonic oscillator}\label{Eq:QuantHO}
To the the problem, we first have to go back the cooking recipe for the quantization of a classical problem.  In a first step, we have to obtain the relevant Lagrangian. Knowing what is it, is a rather interesting artform of theoretical physics, but for most cases you might just take it as a given thing. We have:
\begin{align}
L_{HO} = \frac{1}{2}m\dot{x}^2-\frac{m\omega^2}{2}x^2
\end{align}
We can now identify the conjugate momentum as:
\begin{align}
p &= \frac{\partial L}{\partial \dot{x}}\\
 &= m\dot{x}
\end{align}
At this stage we can calculate the Hamiltonian:
\begin{align}
H &= \dot{x}p - L\\
H_{HO} &= \frac{p^2}{2m}+\frac{m\omega^2}{2}x^2
\end{align}
At this stage we can identify the classical equations of motion through:
\begin{align}
\frac{dx}{dt}&= \frac{\partial H}{\partial p}\\
\frac{dp}{dt}&= -\frac{\partial H}{\partial x}
\end{align}
It results just in the usual Newtons law. To get now the quantum formulation, we can quantize the system by imposing the commutation relationship on position and its conjugate momentum:
\begin{align}
~[\hat{x}, \hat{p}] = i\hbar
\end{align}
We then had the final Hamiltonian:
\begin{align}
\hat{H} &= \frac{\hat{p}^2}{2m}+\frac{m\omega^2}{2}\hat{x}^2
\end{align}
However, we know from the previous lecture that it is much nicer to work in the basis of raising and lowering operators:
\begin{align}
\hat{a} &= \frac{1}{\sqrt{2\hbar}}\left(\sqrt{m\omega}x+i\frac{p}{\sqrt{m\omega}}\right)\\
\hat{a}^\dag &= \frac{1}{\sqrt{2\hbar}}\left(\sqrt{m\omega}x-i\frac{p}{\sqrt{m\omega}}\right)\\
~[\hat{a}, \hat{a}^\dag] &=1
\end{align}
for which we obtain:
\begin{align}
\hat{H} &=\hbar\omega\left(\hat{a}^\dag \hat{a}+\frac{1}{2}\right)
\end{align}

\section{The lagrangien for the electromagnetism}

At this stage we would like to roll out the formalism for electromagnetism, following the discussion of Cohen-Tannoudji Vol 3 (appendix of 18) \cite{laloe2017}. It adds sequentially the following new problems, which we will tackle as we get to them:
\begin{itemize}
\item The Lagrangien has plenty degrees of freedom.
\item The dynamical variables are complex.
\end{itemize}
We start out with the Lagrangien:
\begin{align}
L &= \int d^3r \mathcal{L}(\vec{r},t)\\
\mathcal{L}(\vec{r},t)&= \frac{\epsilon_0}{2}\left(\vec{E}^2-c^2\vec{B}^2\right)
\end{align}
As it depends explicitly on the electric and magnetic field it is manifestly gauge invariant. However, the current version does not allow us to to identify the conjugate variables. We will use the vector potential to introduce them:
\begin{align}
\mathcal{L}(\vec{r},t)&= \frac{\epsilon_0}{2}\left(\dot{\vec{A}}^2(\vec{r},t)-c^2\left(\nabla \times \vec{A}\right)^2\right)
\end{align}
Now we obtained the time derivative, which we can employ to identify the conjugate momentum, but it also gave us the unwanted rot term. We get rid of them transforming into Fourier space:
We can then write:
\begin{align}
\int d\vec{r} \vec{E}(\vec{r},t)\vec{E}(\vec{r},t)&= \int d\vec{k} \dot{\vec{A}}(\vec{k},t)\dot{\vec{A}}^*(\vec{k},t)\\
\int d\vec{r} \vec{B}(\vec{r},t)\vec{B}(\vec{r},t)&= \int d\vec{k}k^2 \vec{A}(\vec{k},t)\vec{A}^*(\vec{k},t)\\
\end{align}
We can then write the Lagrangien as:
\begin{align}
L &= \int d^3k \mathcal{L}(\vec{k},t)\\
\mathcal{L}(\vec{k},t) &= \epsilon_0\left( \dot{\vec{A}}(\vec{k},t)\dot{\vec{A}}^*(\vec{k},t)-c^2k^2 \vec{A}(\vec{k},t)\vec{A}^*(\vec{k},t)\right)
\end{align}
As for the integral over $\vec{k}$, we will only integrate over the positive contributions. This avoids summing over identical terms at $\vec{k}$ and $-\vec{k}$. Finally, we can also use the polarization \eqref{Eq:PotPolar} to obtain:
\begin{align}
\mathcal{L}(\vec{k},t)&= \epsilon_0\sum_i \left( \dot{A}_i(\vec{k},t)\dot{A}_i^*(\vec{k},t)-c^2k^2 A_i(\vec{k},t)A_i^*(\vec{k},t)\right)
\end{align}


\subsection{The conjugate moment and hamiltonian}
We have discussed in Sec. \ref{Eq:QuantHO}  how to find the conjugate momentum for classical variables. Here, we have complex variables. But, we can deduce the conjugate momentum through as decomposition $X = x_1 + i x_2$. We actually obtain:
\begin{align}
P &= \frac{\partial L}{\partial \dot{X}^*}
\end{align}

Here, it implies that the conjugate momentum $\Pi_i(\vec{k})$ is 
\begin{align}
\Pi_i(\vec{k}) &= \frac{\partial\mathcal{L}(\vec{k},t)}{\partial \dot{A}^*_i(\vec{k},t)}\\
&= \epsilon_0 \dot{A}_i(\vec{k},t)\\
&= -\epsilon_0 E_i(\vec{k},t)
\end{align}
\textbf{So the conjugate momentum to the vector potential is the electric field.}

We can now calculate the Hamiltonian before we quantize the system. We obtain:
\begin{align}
H &= \sum_i \int d\vec{k}\left(\dot{A}_i(\vec{k},t)\Pi_i(\vec{k},t)+\dot{A}^*_i(\vec{k},t)\Pi^*_i(\vec{k},t)\right)-L
\end{align}
We finally obtain the Hamiltonian of free radiation:
\begin{align}
H &= \int d\vec{k} \mathcal{H}(\vec{k})\\
\mathcal{H}(\vec{k}) &= \sum_i  \left(\frac{1}{\epsilon_0}\Pi_i^*(\vec{k},t)\Pi_i(\vec{k},t)+\epsilon_0c^2k^2 A^*_i(\vec{k},t)A_i(\vec{k},t)\right)
\end{align}

\subsection{The quantized Hamiltonian}
We are now ready to quantize the system, we simply have to be careful about the quantization of the complex operators. Going through the components, we obtain:
\begin{itemize}
\item $A_i^*\rightarrow \hat{A}_i^\dag$
\item $A_i\rightarrow \hat{A}_i$
\item $[\hat{A}_i(\vec{k}),\hat{\Pi}^\dag_j(\vec{k}')]= i\hbar \delta_{ij}\delta(\vec{k}-\vec{k}')$
\end{itemize}
Hence, the fully quantized Hamiltonian is:
\begin{align}
H &= \int d\vec{k} \mathcal{H}(\vec{k})\\
\hat{\mathcal{H}}(\vec{k}) &= \sum_i  \left(\frac{1}{\epsilon_0}\hat{\Pi}_i^\dag(\vec{k})\hat{\Pi}_i(\vec{k})+\epsilon_0c^2k^2 \hat{A}^\dag_i(\vec{k})\hat{A}_i(\vec{k})\right)\\
 &= \epsilon_0 \sum_i  \left(\hat{E}_i^\dag(\vec{k})\hat{E}_i(\vec{k})+c^2k^2 \hat{A}^\dag_i(\vec{k})\hat{A}_i(\vec{k})\right)
\end{align}



\section{The normal modes}
The hamiltonian above looks roughly like a harmonic oscillator, but not really yet as there are some funny conjugates trailing. This can get solved through the definition of the appropiate raising and lowering operators, named \textbf{normal modes}. They are defined through:
\begin{align}
\hat{a}_i(\vec{k}) = \sqrt{\frac{\epsilon_0 \omega}{2\hbar }}\left(\hat{A}_i(\vec{k})+\frac{i}{\epsilon_0\omega}\hat{\Pi}_i\right)\\
\hat{a}^\dag_i(\vec{k}) = \sqrt{\frac{\epsilon_0 \omega}{2\hbar }}\left(\hat{A}^\dag_i(\vec{k})-\frac{i}{\epsilon_0\omega}\hat{\Pi}_i^\dag \right)
\end{align}
With the underlying commutation relationships of the conjugate operators we obtain the usual raising and lowering operators:
\begin{align}
~[\hat{a}_i(\vec{k}), \hat{a}^\dag_j(\vec{k}')] &=\delta_{ij}\delta(\vec{k}-\vec{k}')
\end{align}
Multiplying it out brings the Hamiltonian in normal form:
\begin{align}
H &= \int d\vec{k} \mathcal{H}(\vec{k})\\
\hat{\mathcal{H}}(\vec{k}) &= \sum_i  \frac{\hbar\omega_k}{2}\left(\hat{a}_i(\vec{k})\hat{a}_i^\dag(\vec{k})+\hat{a}_i^\dag(\vec{k})\hat{a}_i(\vec{k})\right)\\
&= \sum_i \hbar\omega_k\left(\hat{a}^\dag_i(\vec{k})\hat{a}_i(\vec{k})+\frac{1}{2}\right)
\end{align}

\subsection{Quadratures}
It is common to push the terminology of the harmonic oscillator even further, by the definition of the quadratures:
\begin{align}
\hat{a}_i(\vec{k}) &= \frac{1}{\sqrt{\hbar2}}\left(\hat{Q}_i(\vec{k})+\hat{P}_i(\vec{k})\right)\\
\hat{a}^\dag_i(\vec{k}) &=\frac{1}{\sqrt{\hbar2}}\left(\hat{Q}_i(\vec{k})-\hat{P}_i(\vec{k})\right)
\end{align}

\subsection{The field operators}
We can also express the actual field operators in terms of the normal modes:
\begin{align}
\hat{E}(\vec{r})&= i \int \frac{d\vec{k}}{(2\pi)^{3/2}}\sum_i \left(\frac{\hbar\omega}{2\epsilon_0}\right)\vec{e}_i\left(\hat{a}_i(\vec{k})e^{i\vec{k}\vec{r}}-\hat{a}^\dag_i(\vec{k})e^{-i\vec{k}\vec{r}}\right)\\
\hat{B}(\vec{r})&= \frac{i}{c} \int \frac{d\vec{k}}{(2\pi)^{3/2}}\sum_i \left(\frac{\hbar\omega}{2\epsilon_0}\right)(\vec{k}\times\vec{e}_i)\left(\hat{a}_i(\vec{k})e^{i\vec{k}\vec{r}}-\hat{a}^\dag_i(\vec{k})e^{-i\vec{k}\vec{r}}\right)\\
\hat{A}(\vec{r})&=  \int \frac{d\vec{k}}{(2\pi)^{3/2}}\sum_i \left(\frac{\hbar}{2\epsilon_0\omega}\right)\vec{e}_i\left(\hat{a}_i(\vec{k})e^{i\vec{k}\vec{r}}+\hat{a}^\dag_i(\vec{k})e^{-i\vec{k}\vec{r}}\right)
\end{align}

In the following lectures we will typically focus on the electric field as it couples to the electron charge through the electric dipole moment:
\begin{align}
H_I = \vec{D}\cdot\vec{E}
\end{align}
This will be the content of the next lecture \cite{states}.

\section{The notion of the photon}
We can now get back to the interpretation of the eigenstates of the raising and lowering operators as a photon. We have seen previously in lecture 6\cite{Jendrzejewski} that the operator $\hat{n}_i(\vec{k}) = \hat{a}_i^\dag (\vec{k}) \hat{a}_i(\vec{k}) $ is counting the occupation number in the Fock basis:
\begin{align}
\hat{n}_i(\vec{k}) \ket{n_i}&= n_i\ket{n_i}
\end{align}

The $n_i$ are then non-negative integers. We can further create a well-defined photon number state through the raising operators from the vacuum:
\begin{align}
\ket{n} = \frac{(a^\dag)^n}{\sqrt{n!}}\ket{0}
\end{align}
Given all the numbers the single photon energy is corresponding to intensities in the order pico to femto Watts($\sim 10^{-12}$ -- $10^{-15}$W). 

A rather nice discussion concerning the details of a proper quantization of the interacting field theory of quantum electromagnetism is given in the book by Kleinert \cite{electrodynamics}

\selectlanguage{english}
\FloatBarrier
\bibliographystyle{plainnat}
\bibliography{bibliography/converted_to_latex.bib%
}

\end{document}

