\documentclass[10pt]{article}

\usepackage{fullpage}
\usepackage{setspace}
\usepackage{parskip}
\usepackage{titlesec}
\usepackage[section]{placeins}
\usepackage{xcolor}
\usepackage{breakcites}
\usepackage{lineno}
\usepackage{hyphenat}





\PassOptionsToPackage{hyphens}{url}
\usepackage[colorlinks = true,
            linkcolor = blue,
            urlcolor  = blue,
            citecolor = blue,
            anchorcolor = blue]{hyperref}
\usepackage{etoolbox}
\makeatletter
\patchcmd\@combinedblfloats{\box\@outputbox}{\unvbox\@outputbox}{}{%
  \errmessage{\noexpand\@combinedblfloats could not be patched}%
}%
\makeatother


\usepackage[round]{natbib}
\let\cite\citep




\renewenvironment{abstract}
  {{\bfseries\noindent{\abstractname}\par\nobreak}\footnotesize}
  {\bigskip}

\titlespacing{\section}{0pt}{*3}{*1}
\titlespacing{\subsection}{0pt}{*2}{*0.5}
\titlespacing{\subsubsection}{0pt}{*1.5}{0pt}


\usepackage{authblk}


\usepackage{graphicx}
\usepackage[space]{grffile}
\usepackage{latexsym}
\usepackage{textcomp}
\usepackage{longtable}
\usepackage{tabulary}
\usepackage{booktabs,array,multirow}
\usepackage{amsfonts,amsmath,amssymb}
\providecommand\citet{\cite}
\providecommand\citep{\cite}
\providecommand\citealt{\cite}
% You can conditionalize code for latexml or normal latex using this.
\newif\iflatexml\latexmlfalse
\AtBeginDocument{\DeclareGraphicsExtensions{.pdf,.PDF,.eps,.EPS,.png,.PNG,.tif,.TIF,.jpg,.JPG,.jpeg,.JPEG}}

\usepackage[utf8]{inputenc}
\usepackage[ngerman,english]{babel}








\usepackage{siunitx}
\usepackage{amsmath}
\newcommand{\bra}[1]{\ensuremath{\left\langle#1\right|}}
\newcommand{\ket}[1]{\ensuremath{\left|#1\right\rangle}}
\newcommand{\braket}[1]{\ensuremath{\left\langle#1\right\rangle}}
\newcommand{\rhohat}{\hat{\rho}}
\newcommand{\tr}[1]{\mathrm{tr}(#1)}
\newcommand{\trarb}[2]{\mathrm{tr}_{#1}(#2)}
\newcommand{\vv}[1]{\mathbf{#1}}
\newcommand*\dif{\mathop{}\!\mathrm{d}}
\newcommand{\eexp}[1]{\mathrm{e}^{#1}}
\newcommand*\ch[1]{\ensuremath{\mathrm{#1}}}

\begin{document}

\title{Lecture 20 - Controlling spontaneous emission and absorption}



\author[1]{Fred Jendrzejewski}%
\author[2]{Selim Jochim}%
\author[2]{Matthias Weidemüller}%
\affil[1]{Kirchhoff-Institut für Physik}%
\affil[2]{Physikalisches Institut der Universität Heidelberg}%


\vspace{-1em}



  
  \date{January 07, 2025}


\begingroup
\let\center\flushleft
\let\endcenter\endflushleft
\maketitle
\endgroup





\selectlanguage{english}
\begin{abstract}
In todays lecture we will study spontaneous emission its connection with absorption and how we can control it certain situations.%
\end{abstract}%



\sloppy


In the last lecture \cite{states} we treated the atom as a two-level system and coupled it only to a single mode of the electromagnetic vacuum. However, this situation is a highly engineered one and more typically the atom is coupled to a large (infinite) number of electromagnetic field mode. This coupling is at the origin of spontaneous emission as we will see now.

\section{Spontaneous Emission}

We assume that initially the atom is in the excited state $\ket{e}$ and that there is no light field present, i.e., the field state is $\ket{0}$. We would now like to understand how it can get into the state $\ket{g}$. The answer is after our previous discussions that it will emit a photon into one of the vacuum modes, which was already presented in \cite{field}. 

This coupling is associated with the matrix element
%
\begin{align}
w = \braket{g|\hat{\vec{D}} \cdot \vec{\epsilon}^*|e} \braket{n|a^\dag|n'},
\end{align}
%
where $n\overset{!}{=} 1$ and $n'=0$, where the occupation number might describe any polarization or direction. What kind of dynamics do we expect from this? Since, the emission is not restricted to a single mode. So we can \emph{not} write
\begin{align}
\ket{e,0} \rightarrow \ket{g,1}.
\end{align}
%
In fact, the photon is emitted into many modes with almost equal probability:
\begin{align}
\ket{e,\{ 0 \}} \rightarrow \sum\limits_{\vec{k},\;\text{pol.}} c_{\vec{k},\text{pol.}} (t) \ket{g,1_{\vec{k},\text{pol.}}}.
\end{align}
Given that the photon might travel into any direction we are basically studying now the problem of coupling a well-defined initial mode to a continuum.  This problem was as already adressed for the first time in Lecture 4 \cite{Jendrzejewski} and it thoroughly discussed in Sec. 1.3 of \cite{grynberg}.

\subsection{Exponential decay}
So we will describe the initial situation by the initial state $\ket{i} = \ket{e,0}$, which is coupled to a large number of states, which will will note $\ket{k}$. The interaction might be noted $\hat{W}$ and for simplicity we will assume:
\begin{align}
\bra{i}\hat{W}\ket{k} &= w\\
\bra{k'}\hat{W}\ket{k} &= 0\\
\bra{i}\hat{W}\ket{i} &= 0
\end{align}
$w$ is then the constant, real coupling of the initial state to the  states $\ket{k}$. The different energy levels $k$ are then assumed to be of energy:
\begin{align}
E_k = E_i +k\epsilon
\end{align}
$\epsilon$ is therefore the energy spacing for the different states. It is actually quite common to use the density of states.
It is generally defined as:
\begin{align}
\rho(E) &= \frac{dN(E)}{dE}\\
&= \frac{1}{\epsilon}
\end{align}
We can now decompose the wavefunction as:
\begin{align}
\ket{\gamma(t)}=\gamma_i(t)\ket{i} +\sum_k \gamma_k(t) e^{-ik\epsilon t/\hbar}\ket{k} 
\end{align}
The equations of motion are then:
\begin{align}
i\hbar \dot{\gamma}_i &= w\sum_k \gamma_k e^{-ik\epsilon t/\hbar}\\
i\hbar \dot{\gamma}_k &= w \gamma_i  e^{ik\epsilon t/\hbar}
\end{align}
We already discussed its evolution of short times (see \cite{Jendrzejewski}). Here, we will follow the more general Wigner-Weisskopf approach, which also holds for long times. 

Given that the system is initially within the state $\ket{i}$, we can integrate the second equation and obtain:
\begin{align}
\gamma_k &= \frac{w}{i\hbar} \int_0^t dt'~\gamma_i(t') e^{ik\epsilon t'/\hbar}
\end{align}
We can now use this solution to rewrite the first equation as:
\begin{align}\label{Eq:GammaIRepl}
\dot{\gamma}_i &= -\frac{\Gamma}{2\pi\hbar}\int_0^t dt'~\gamma_i(t') \sum_k \epsilon e^{-ik\epsilon (t-t')/\hbar}
\end{align}
We have defined here the transition rate (known from Fermis golden rule):
\begin{align}
\Gamma &= \frac{2\pi}{\hbar}w^2\rho
\end{align}
We can solve this by rewriting the sum over k as an integral over the energy:
\begin{align}
\sum_k \epsilon e^{-ik\epsilon (t-t')/\hbar}\rightarrow \int dE e^{-iE (t-t')/\hbar} = 2\pi \hbar \delta(t-t')
\end{align}
Plugging this into \eqref{Eq:GammaIRepl} we actually end up with\footnote{The factor 1/2 appears as we are not integrating over all times}:
\begin{align}\label{Eq:GammaIRepl}
\dot{\gamma}_i &= -\frac{\Gamma}{2}\gamma_i(t)
\end{align}
So the coupling to the continuum leads to the well-known exponential decay.



\subsection{The Mollow triplet}

We have now seen that spontaneous emission leads to an exponential decay of the excited state $\ket{e}$. How is it modified if we take into account a strong light beam that pumps the atoms in the excited state ? For that it is best to work in the dressed atom picture. For absorption we get for the bare states
\begin{align}
\ket{g,n+1} \rightarrow \ket{e,n},
\end{align}
and emission can be described by
\begin{align} \label{eq:barestatesemission}
\ket{e,n} \rightarrow \ket{g,n+1}.
\end{align}
The laser is dressing the bare states with strength $\hbar \Omega$, which leads to the dressed atom picture shown in (see \ref{786304}): One photon gets emitted into a mode different from the one considered and is lost.\selectlanguage{english}
\begin{figure}[h!]
\begin{center}
\includegraphics[width=0.70\columnwidth]{figures/mollowtriplet/mollowtriplet}
\caption{{Spontaneous emission under dressing, which leads to the Mollow triplet.
Figure taken from \protect\cite{Cohen_Tannoudji_1998}
{\label{786304}}%
}}
\end{center}
\end{figure}



From \ref{786304} we see that in spontaneous emission, three different spectral lines can be observed at:
\begin{align}
&\omega_\textrm{l}\\
&\omega=\omega_\textrm{l} \pm \Omega
\end{align}
This is called the \textit{Mollow triplet}. To actually observe it with a spectrometer, $\Omega$ needs to be much larger than the natural linewidth $\Gamma_\text{sp}$. It was observed in Ref. \cite{Grove_1977}.



\section{Absorption under dressing}

We can move the idea of dressing a step further and ask how do the dressed states absorb light? A possible experiment to investigate this could be to illuminate atoms with a dressing light field and a probing light field, i.e. a beam of a tunable laser (see \ref{118683}). One would then measure how much of the probe  field is transmitted through the bunch of atoms.\selectlanguage{english}
\begin{figure}[h!]
\begin{center}
\includegraphics[width=0.70\columnwidth]{figures/Bildschirmfoto-2018-10-02-um-08-37-21/ThreeLevelAbsorption}
\caption{{A possible experiment to investigate how dressed atoms absorb light.
{\label{118683}}%
}}
\end{center}
\end{figure}

A typical configuration of the experiments that is presented in Fig. \ref{118683} b) is that it addresses a total of three levels. The dressing laser will then couple the state $\ket{i}$ and the excited state $\ket{e}$ with strength $\Omega_1$. The probe laser couples the state $\ket{g}$ to the same excited state $\ket{e}$ with coupling strength $\Omega_2$.  Only the excited state has some finite lifetime, the other two states are very long lived or stable. 
We can then describe the Hamiltonian through the following Hamiltonian:
\begin{align}\label{Eq:ThreeLevelHam}
\hat{H}&= \Omega_1\left(\ket{i}\bra{e}+\ket{e}\bra{i}\right)+\Omega_2\left(\ket{g}\bra{e}+\ket{e}\bra{g}\right)
\end{align}
To understand the absorption qualitatively, we can once again work in the dressed atom picture. The dressing laser is creating a superpositions of $\ket{e}$ and $\ket{i}$, which are separated by $2\Omega_1$. The probe laser field should then experience two peaks with a dip on resonance. This is indeed the observed shape, however the on resonance we do not just have a dip, but the absorption  disappears as shown in Fig. \ref{893938}.\selectlanguage{english}
\begin{figure}[h!]
\begin{center}
\includegraphics[width=0.70\columnwidth]{figures/Bildschirmfoto-2019-01-07-um-08-04-42/Bildschirmfoto-2019-01-07-um-08-04-42}
\caption{{Observation of EIT. Figure taken from \protect\cite{Boller_1991}
{\label{893938}}%
}}
\end{center}
\end{figure}

\subsection{EIT}
This observation is known as Electromagnetically-Induced Transmission (EIT). To understand it, we can rewrite \eqref{Eq:ThreeLevelHam} as:
\begin{align}
\hat{H}&= (\Omega_1\ket{i}+\Omega_2\ket{g})\bra{e}+\ket{e}(\Omega_1\bra{i}+\Omega_2\bra{g})\\
&\propto\ket{B}\bra{e}+\ket{e}\bra{B}\\
\ket{B}&= \frac{\Omega_1\ket{i}+\Omega_2\ket{g}}{\sqrt{\Omega_1^2+\Omega_2^2}}
\end{align}
So in the three level scheme the excited state is always could to the so-called bright state, which is a coherent superposition of $\ket{g}$ and $\ket{i}$. The orthogonal state is the dark state:
\begin{align}
\ket{D}&= \frac{\Omega_2\ket{i}-\Omega_1\ket{g}}{\sqrt{\Omega_1^2+\Omega_2^2}}\\
\langle B| D\rangle &= 0
\end{align}
Only the bright state is coupled to excited state, while the dark state is uneffected by the two lasers.

\subsection{Slow light}
The complex absorption spectrum tests the imaginary part of the susceptibility $\chi$. The real part of the susceptibility on the other hand indicates the group velocity of the light propagating through the medium. Actually, the group velocity of the light pulse is given by (see \cite{Jendrzejewskia}):
\begin{align}
v_g = \frac{c}{1+\frac{\omega_L}{2}\frac{d\chi}{d\omega}}
\end{align}
The predictions for the three level system are shown on left hand side of Fig. \ref{112751}.\selectlanguage{english}
\begin{figure}[h!]
\begin{center}
\includegraphics[width=0.70\columnwidth]{figures/SlowLight/SlowLight}
\caption{{Observation of slow light as shown in \protect\cite{Hau_1999}
{\label{112751}}%
}}
\end{center}
\end{figure}

So while the bright state might be able to propagate with strongly reduced absorption this comes to the price of a strongly reduced group velocity.

\subsection{Light storage}

The idea of bright and dark states has been pushed to the extreme nowadays in the idea of storing light. Imagine the following situation:
\begin{itemize}
\item An atomic vapor is dressed through a dressing beam.
\item A weak probe pulse is then coupled into the dark state. In other words it sees a transparent medium and simply travels through the atomic vapor.
\item As the pulse is within the atomic vapor, we switch off the dressing field. What happens ?
\end{itemize}
The results of such an experiment are shown in Fig. \ref{561560}.\selectlanguage{english}
\begin{figure}[h!]
\begin{center}
\includegraphics[width=0.70\columnwidth]{figures/Bildschirmfoto-2019-01-07-um-08-58-14/Bildschirmfoto-2019-01-07-um-08-58-14}
\caption{{Storage of light as observed in Ref. \protect\cite{Phillips_2001}
{\label{561560}}%
}}
\end{center}
\end{figure}

So basically, the light pulse was stored in the atomic vapor. The precise theoretical description goes substantially beyond the scope of this lecture, but the idea is rather roughly the following:
\begin{itemize}
\item In the presence of the controlling beam the propagating 'dressed' states is a mixture of the atoms and the light itself. It  in system.
\item When the control beam is switched off the dressed dark state becomes a purely atomic excitation, which hence does not propagate anymore.
\item Once the control beam is switched on again the 'dark' state becomes once again a mixture of light and atoms and the light can escape the atomic vapor once again.
\end{itemize}

This idea of mixed atom+light states is widely used by now for the storage of light, but also to create 'interacting' photons \cite{Firstenberg_2013}, polaritons etc. 
\subsection{STIRAP}

This idea was also used to get molecules into their groundstate, through a transfer sequence known as STIRAP (stimulated Raman adiabatic passage). 

Given all the complexities of molecules it seems non-trivial to find a scheme that gets them into the ground state. For atoms laser cooling has proven very efficient as we will discuss later. However, it mainly adresses the cooling of external degrees of freedom. In molecules a significant amount of energy its in the rotational and vibrational levels. In this connection, a beautiful solution has been demonstrated in \cite{Ni_2008}. The scheme is visualized in Fig. \ref{304719}.\selectlanguage{english}
\begin{figure}[h!]
\begin{center}
\includegraphics[width=0.70\columnwidth]{figures/Bildschirmfoto-2018-12-12-um-10-32-59/Bildschirmfoto-2018-12-12-um-10-32-59}
\caption{{Production of groundstate molecules of K + Rb. Figure is taken from
\protect\cite{Ni_2008}
{\label{304719}}%
}}
\end{center}
\end{figure}



In a first step the atoms are cooled and then associated to a highly excited molecule in the a\textsuperscript{3}$\Sigma$ state. From there the atom has to be transferred down in to the ground state $\ket{g}$. A direct thransfer is not possible as the Franck-Condon factors do not allow for it. Another path is to go through an intermediate level (here the 2$^3 \Sigma$ level), which has overlap with both of them. However, this level has typically overlap with plenty of other levels and a finite lifetime. How can we then optimize the transfer ? The idea is to use the concept of dark states in the triplet of $\{i, e, g\}$.


STIRAP transfers the loosely bound molecules coherently into the groundstate without ever passing through the lossy excited level. It has the following steps:
\begin{enumerate}
\item The dressing laser $\Omega_2$ is ramped on. The initial $\ket{i}$ is now the dark state.
\item The coupling laser $\Omega_1$ is ramped on, while the laser $\Omega_2$ is ramped down. This transfers the $\ket{i}$ adiabatically into the state $\ket{g}$, which is the dark state for fully switched of $\Omega_2$. 
\end{enumerate}
The molecules are now in the groundstate with a transfer efficiency of roughly $50\%$.

\selectlanguage{english}
\FloatBarrier
\bibliographystyle{plainnat}
\bibliography{bibliography/converted_to_latex.bib%
}

\end{document}

