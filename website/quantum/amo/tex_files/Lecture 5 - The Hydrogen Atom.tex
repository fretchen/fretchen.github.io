\documentclass[10pt]{article}

\usepackage{fullpage}
\usepackage{setspace}
\usepackage{parskip}
\usepackage{titlesec}
\usepackage[section]{placeins}
\usepackage{xcolor}
\usepackage{breakcites}
\usepackage{lineno}
\usepackage{hyphenat}





\PassOptionsToPackage{hyphens}{url}
\usepackage[colorlinks = true,
            linkcolor = blue,
            urlcolor  = blue,
            citecolor = blue,
            anchorcolor = blue]{hyperref}
\usepackage{etoolbox}
\makeatletter
\patchcmd\@combinedblfloats{\box\@outputbox}{\unvbox\@outputbox}{}{%
  \errmessage{\noexpand\@combinedblfloats could not be patched}%
}%
\makeatother


\usepackage[round]{natbib}
\let\cite\citep




\renewenvironment{abstract}
  {{\bfseries\noindent{\abstractname}\par\nobreak}\footnotesize}
  {\bigskip}

\titlespacing{\section}{0pt}{*3}{*1}
\titlespacing{\subsection}{0pt}{*2}{*0.5}
\titlespacing{\subsubsection}{0pt}{*1.5}{0pt}


\usepackage{authblk}


\usepackage{graphicx}
\usepackage[space]{grffile}
\usepackage{latexsym}
\usepackage{textcomp}
\usepackage{longtable}
\usepackage{tabulary}
\usepackage{booktabs,array,multirow}
\usepackage{amsfonts,amsmath,amssymb}
\providecommand\citet{\cite}
\providecommand\citep{\cite}
\providecommand\citealt{\cite}
% You can conditionalize code for latexml or normal latex using this.
\newif\iflatexml\latexmlfalse
\AtBeginDocument{\DeclareGraphicsExtensions{.pdf,.PDF,.eps,.EPS,.png,.PNG,.tif,.TIF,.jpg,.JPG,.jpeg,.JPEG}}

\usepackage[utf8]{inputenc}
\usepackage[ngerman,english]{babel}








\usepackage{siunitx}
\usepackage{amsmath}
\newcommand{\bra}[1]{\ensuremath{\left\langle#1\right|}}
\newcommand{\ket}[1]{\ensuremath{\left|#1\right\rangle}}
\newcommand{\braket}[1]{\ensuremath{\left\langle#1\right\rangle}}
\newcommand{\rhohat}{\hat{\rho}}
\newcommand{\tr}[1]{\mathrm{tr}(#1)}
\newcommand{\trarb}[2]{\mathrm{tr}_{#1}(#2)}
\newcommand{\vv}[1]{\mathbf{#1}}
\newcommand*\dif{\mathop{}\!\mathrm{d}}
\newcommand{\eexp}[1]{\mathrm{e}^{#1}}

\begin{document}

\title{Lecture 5 - The Hydrogen Atom}



\author[1]{Fred Jendrzejewski}%
\author[2]{Selim Jochim}%
\affil[1]{Kirchhoff-Institut für Physik}%
\affil[2]{Physikalisches Institut der Universität Heidelberg}%


\vspace{-1em}



  
  \date{January 04, 2025}


\begingroup
\let\center\flushleft
\let\endcenter\endflushleft
\maketitle
\endgroup





\selectlanguage{english}
\begin{abstract}
In this lecture we will first discuss the diagonalization of the harmonic oscillator and then discuss the main properties of the hydrogen atom.%
\end{abstract}%



\sloppy


In the previous lectures we have seen how to treat eigenstates of the two-level system and then how we can derive its effective emergence from some complex level structure if we \href{https://www.authorea.com/users/143341/articles/326506-lecture-4-atoms-in-oscillating-fields}{apply oscillating fields}.

Today, we will increase the complexity towards the harmonic oscillator and the hydrogen atom.


\section{The harmonic oscillator}

The harmonic oscillator is another great toy model to understand certain properties of quantum mechanical systems. Most importantly, it is a great introduction into the properties of bound systems and ladder operators. The basic Hamiltonian comes along in a rather innocent fashion, namely:

\begin{align}\label{Eq:HamHO}
\hat{H} &= \frac{\hat{p}^2}{2m}+ \frac{m\omega^2}{2}\hat{x}^2
\end{align}
The two variables $\hat{p}$ and $\hat{x}$ are non-commuting $[\hat{x}, \hat{p}] = i\hbar$, so they cannot be measured at the same time. We would now like to put the operator into a diagonal form such that it reads something like:
\begin{align}\label{Eq:HamHO}
\hat{H} &= \sum_n \epsilon_n \ket{n}\bra{n}
\end{align}

We will follow he quite closely the discussion of Ref. \cite{interactions}.

\subsection{The ladder operators}
We would like to get the spectrum first. So make the equation look a bit nicer we will define $\hat{p} = \hat{P} \sqrt{m\omega}$ and $\hat{x} = \frac{\hat{X}}{\sqrt{m\omega}}$ such that we have:
\begin{align}\label{Eq:HamHO}
\hat{H} &= \frac{\omega}{2}\left(\hat{P}^2 + \hat{X}^2\right)
\end{align}
\footnote{The commutator between $\hat{X}$ and $\hat{P}$ is still as for $x$ and $p$.} The next step is then to define the ladder operators:
\begin{align}
\hat{a} = \frac{1}{\sqrt{2\hbar}}\left(\hat{X}+i\hat{P}\right)\\
\hat{a}^\dag = \frac{1}{\sqrt{2\hbar}}\left(\hat{X}-i\hat{P}\right)\\
\end{align}
At this stage we can just try to rewrite the Hamiltonian in terms of the operators, such that:
\begin{align}
\hat{a}^\dag \hat{a} &= \frac{1}{2\hbar}(\hat{X}-i\hat{P})(\hat{X}+i\hat{P})\\
&= \frac{1}{2\hbar}(\hat{X}^2 +\hat{P}^2 -\hbar)\\
 \frac{1}{2}(X^2 +\hat{P}^2 ) &= \hbar \left(\hat{a}^\dag \hat{a}-\frac{1}{2}\right)
\end{align}
So the Hamiltonian can now be written as:
\begin{align}
\hat{H} &= \hbar \omega \left(\hat{N} + \frac{1}{2}\right)\text{ with } \hat{N} = a^\dag a
\end{align}
At this stage we have diagonalized the Hamiltonian, what remains to be understood is the the values that $\hat{a}^\dag a$ can take.

\subsection{Action of the ladder operators in the Fock basis}

We would like to understand the basis, which is defined by:
\begin{align}
\hat{N} \ket{n} = n \ket{n}
\end{align}
The non-commutation between $\hat{X}$ and $\hat{P}$ is translated to the ladder operators as:
\begin{align}
[\hat{a}, \hat{a}^\dag] &= \frac{1}{2\hbar}[\hat{X}+iP,\hat{X}-i\hat{P}] = 1\\
~[\hat{N}, a] &= -\hat{a}\\
~[\hat{N}, a^\dag] &= a^\dag
\end{align}
From these relationship we can show then that:
\begin{align}
\hat{a}\ket{n} = \sqrt{n}\ket{n-1}\\
\hat{a}^\dag \ket{n} = \sqrt{n+1}\ket{n+1}\\
\end{align}
These relations are the motivation for the name ladder operators as they connect the different eigenstates. And they are raising/lowering the quantum number by one. Finally we have to find the lower limit. And this is quite naturally 0 as $n = \bra{n}\hat{N}\ket{n} = \bra{\psi_1}\ket{\psi_1}\geq 0$. So we can construct the full basis by just defining the action of the lowering operator on the zero element $a\ket{0} = 0$ and the other operators are then constructed as:
\begin{align}
\ket{n} = \frac{(a^\dag)^n}{\sqrt{n!}}\ket{0}
\end{align}

\subsection{Spatial representation of the eigenstates}

While we now have the spectrum it would be really nice to obtain the spatial properties of the different states. For that we have to project them onto the x basis. Let us start out with the ground state for which we have $\hat{a}\ket{0}= 0$:
\begin{align}
\bra{x}\frac{1}{\sqrt{2\hbar}}\left(\sqrt{m\omega}\hat{x} +i \frac{1}{\sqrt{m\omega}}\hat{p}\right)\ket{0}= 0\\
\left(\sqrt{\frac{m\omega}{\hbar}}x + \sqrt{\frac{\hbar}{m\omega}}\partial_x\right)\psi_0(x)= 0\\
\Rightarrow \psi_0(x) \propto e^{-\frac{x^2}{2a_{HO}^2}}
\end{align}
This also introduces the typical distance in the quantum harmonic oscillator which is given by $a_{HO} =\sqrt{\hbar/m\omega}$. The other states are solutions to the defining equations:
\begin{align}
\psi_n(x) = \frac{1}{\sqrt{n!}2^n}\left(\sqrt{m\omega}x - \frac{1}{\sqrt{m\omega}}\frac{d}{dx}\right)^n \psi_0(x)\\
\psi_n(x) = \frac{1}{\sqrt{n!}2^n}H_n(x) \psi_0(x)\\
\end{align}
where $H_n(x)$ are the Hermite polynoms.

\section{The hamiltonian of the hydrogen atom}

The hydrogen atom plays at central role in atomic physics as it is \textit{the} basic ingredient of atomic structures. It describes a single \textit{electron}, which is bound to the nucleus of a single \textit{proton}. As such it is the simplest of all atoms and can be described analytically within high precision. This has motivated an enormous body of literature on the problem, which derives all imaginable properties in nauseating detail. Therefore, we will focus here on the main properties and only sketch the derivations, while we will reference to the more technical details. 

For the hydrogen atom as shown in \ref{261310}, we can write down the Hamiltonian 
\begin{align}
\hat{H}=\frac{{{\hat{\vec{p}}}^2_\text{p}}}{2m_\text{p}} + \frac{{\hat{\vec{p}}}^2_\text{e}}{2m_\text{e}} - \frac{Ze^2}{4\pi\epsilon_0 r},
\end{align}
where $Ze$ is the nuclear charge. To solve the problem, we have to find the right Hilbert space. We can not solve the problem of the electron alone. If we do a separation of coordinates, i.e., we separate the Hamiltonian into the the center of mass and the relative motion, we get
\begin{align}
\hat{H} = \underbrace{\frac{{\hat{\vec{p}}}^2_{\textrm{cm}}}{2M}}_{\hat{H}_{\textrm{cm}}} + \underbrace{\frac{{\hat{\vec{p}}}^2_\text{r}}{2\mu}- \frac{Ze^2}{4\pi\epsilon_0r}}_{\hat{H}_{\text{atom}}} \label{eq:hydrogencmatomsplit}
\end{align}
with the reduced mass $1/\mu=1/m_\text{e}+1/m_\text{p}$.
If the state of the hydrogen atom $\ket{\psi}$ is an eigenstate of $\hat{H}$, we can write
\begin{align}
\hat{H}\ket{\psi}=&\left( \hat{H}_\textrm{cm}+\hat{H}_{\text{atom}} \right)\ket{\psi_\textrm{cm}}\otimes \ket{\psi_\text{atom}} \label{eq:hydrogencmatom}\\
=& \left( E_{\text{kin}} + E_\text{atom} \right) \ket{\psi}.
\end{align}
Both states in \eqref{eq:hydrogencmatom} are eigenstates of the system. %, e.g. particle being in momentum eigenstate
The state $\ket{\psi}$ can be split up as shown since the two degrees of freedom are generally not entangled.% Hilbert space\selectlanguage{english}
\begin{figure}[h!]
\begin{center}
\includegraphics[width=0.70\columnwidth]{figures/Bildschirmfoto-2018-09-28-um-16-07-07/Bildschirmfoto-2018-09-28-um-16-07-07}
\caption{{Sketch of the hydrogen atom with the relative coordinate and the
coordinates of the proton and the electron.
{\label{261310}}%
}}
\end{center}
\end{figure}



The wave function of the system then reads:
\begin{align}
\psi(\vec{R},\vec{r}) =& \left( \bra{R} \otimes \bra{r}\right)\left( \ket{\psi_\textrm{cm}} \otimes \ket{\psi_{\text{atom}}}\right)\\
=& \psi(\vec{R}) \cdot \psi (\vec{r})
\end{align}
%
Our goal is now to find the eigenfunctions and eigenenergies of $\hat{H}_\text{atom}$. In order to further divide the Hilbert space, we can use the symmetries.

\section{Conservation of orbital angular momentum}

$\hat{H}_\text{atom}$  possesses spherical symmetry, which implies that \textbf{orbital angular momentum} $\hat{\vec{L}}$ is conserved. It is defined as:
\begin{align}
\hat{\vec{L}}=\hat{\vec{r}} \times \hat{\vec{p}}
\end{align}
In other words, we have:
\begin{align}
[\hat{H}_\text{atom}, \hat{\vec{L}}] = 0
\end{align}
Let us show first that the kinetic term commutes with the angular momentum operator,
We will employ the commutator relationships for position and momentum $[x_i, p_j]=i\hbar$ and the relationship $[A,BC] = [A,B]C+B[A,C]$ and $[f(x), p_x] = [x,p_x]\frac{\partial f(x)}{\partial x}$. So we obtain:
\begin{align}
[p_x^2+p_y^2+p_z^2, xp_y - yp_x]&= [p_x^2,xp_y]-[p_y^2,yp_x] \\
 &= [p_x^2,x]p_y-[p_y^2,y] p_x\\
 &=i\hbar 2 p_xp_y-2i\hbar p_y p_x\\
 &= 0
\end{align}
Analog calculations show that $L_y$ and $L_z$ commute. In a similiar fashion we can verify that the potential term commutes with the different components of $\hat{\vec{L}}$
\begin{align}
[\frac{1}{r}, xp_y -yp_x] &= [\frac{1}{r}, xp_y]-[\frac{1}{r}, yp_x]\\
&= x[\frac{1}{r}, p_y]-y[\frac{1}{r}, p_x]\\
&= -x \frac{yi\hbar}{2r^{3/2}}+y\frac{xi\hbar}{2r^{3/2}}\\
&=0
\end{align}
We can therefore decompose the eigenfunctions of the hydrogen atom over the eigenbasis of the angular momentum operator. A detailled discussion of the properties of $\vec{L}$ can be found in Appendix B of \cite{Hertel_2015}. To find the eigenbasis, we first need to identify the commutation relationships between the components of $\hat{\vec{L}}$. We can calculate them following commutation relationships:
\begin{align}
[L_x, L_y] &= [yp_z - zp_y, zp_x - xp_z]\\
&=[yp_z, zp_x]-[yp_z,xp_z]- [zp_y, zp_x] + [zp_y,xp_z]\\
&=[yp_z, zp_x] + [zp_y,xp_z]\\
&=[yp_z, z]p_x +x[zp_y,p_z]\\
&=-i\hbar yp_x +i\hbar xp_y\\
&= i\hbar L_z
\end{align}
This relationship holds for all the other components too and we have in general:
\begin{align}
[L_i, L_j] = i\hbar \epsilon_{ijk}L_k
\end{align}
The orbital angular momentum is therefore part of the large family of angular momentum operators, which also comprises spin etc. In particular the different components are not independent, and therefore we cannot form a basis out the three components. A suitable choice is actually to use the following combinations: 
\begin{align}
\hat{\vec{L}}^2\ket{l,m_l} =& \hbar^2 l (l+1)\ket{l,m_l}\\
\hat{L}_z\ket{l,m_l} =& \hbar m_l \ket{l,m_l}
\end{align}
\begin{itemize}
\item $l$ is a non-negative integer and it is called the \textbf{orbital angular momentum quantum number}.
\item $m_l$ takes values $-l, -l+1, ..., l-1, l$ and it is sometimes called the \textbf{projection of the angular momentum}.
\end{itemize}

\subsection{Eigenfunction of the angular momentum operators}

Having identified the relevant operators it would be nice to obtain a space representation of them. This works especially nicely in spherical coordinates. There, we get
\begin{align}
\hat{L}_z&= - i \hbar \partial_{\phi}\\
\hat{\vec{L}}^2 &= - \hbar^2 \left[\frac{1}{\sin(\theta)}\partial_{\theta} \left( \sin(\theta) \partial_\theta\right) + \frac{1}{\sin^2(\theta)} \partial_{\phi\phi} \right].
\end{align}
The corresponding wave functions are
\begin{align}
\braket{\theta, \phi | l,m_l} = Y_{lm}(\theta,\phi).
\end{align}

Where $Y_{lm}(\theta, \phi)$ are the \textbf{spherical harmonics}.

\section{The radial wave equation}

Given that we now know that the angular momentum is conserved for the hydrogen atom, we can actually rewrite the Hamltonian \ref{eq:hydrogencmatomsplit} in terms of the angular momentum as we find:
\begin{align}
\hat{H}_\text{atom} = \hat{H}_r + \frac{\hat{L}}{2\mu r^2}+V(r) \\
\hat{H}_r = -\frac{\hbar^2}{2\mu}\frac{1}{r^2}\frac{\partial}{\partial r}\left(r^2\frac{\partial}{\partial r}\right)
\end{align}
We can now separate out the angular part and decompose it over the eigenfunctions of $\hat{\vec{L}}$, such that we make the ansatz \footnote{ Only if the system is in a well-defined angular momentum state, we can write it down like this.}:
\begin{align}
\psi (r,\theta,\phi) = R(r) Y_{lm}(\theta,\phi)
\end{align}

We can plug this separated ansatz in the Schr\selectlanguage{ngerman}ödinger equation. We already solved the angular in the discussion of the angular momentum and for the radial part we obtain:
\begin{align}
-\frac{\hbar^2}{2\mu}\frac{1}{r}\frac{d^2(rR(r))}{dr^2} - \frac{Ze^2}{4\pi\epsilon_0 r} R(r) + \frac{\hbar^2}{2\mu}\frac{l(l+1)}{r^2}R(r) = ER(r)
\end{align}
%
Substituting $R(r)=u(r)/r$ leads to
\begin{align}
-\frac{\hbar^2}{2\mu}\frac{d^2}{dr^2}u(r) +\underbrace{ \left( -\frac{Ze^2}{4\pi\epsilon_0 r} + \frac{\hbar^2}{2\mu} \frac{l(l+1)}{r^2} \right)}_{V_{\text{eff}}} u(r) = E \, u(r),\end{align}
which is known as the ``radial wave equation''. It is a very general result for \emph{any} central potential. It can also be used to describe unbound states ($E>0$) that occur during scattering.

In the \href{https://www.authorea.com/users/143341/articles/326674-lecture-6-the-dipole-approximation-in-the-hydrogen-atom}{next lecture} we will look into the energy scales of the hydrogen atom and then start coupling different levels.

\selectlanguage{english}
\FloatBarrier
\nocite{*}

\bibliographystyle{plainnat}
\bibliography{bibliography/converted_to_latex.bib%
}

\end{document}

